\section{Теоретические задачи}

Здесь представлены теоретические задачи Тараса Евгеньевича Панова, к некоторым также написаны решения (решали вместе с Рамилем Хакамовым). Рекомендую проверять их на адекватность.

\begin{problem}
    Докажите, что пространство, двойственное к пространству $\R[t]$ всех многочленов от одной переменной над $\R$, не изоморфно пространству $\R[t]$.
\end{problem}

\begin{solution}
    В пространстве $\R[t]$ есть счётный базис: $1, x^1, x^2, x^3, \ldots$ Докажем, что в пространстве $(\R[t])^\ast$ счётного базиса нет. Для этого найдём в нём несчётную линейно независимую систему. Положим $\xi_x: \R[t] \to \F$, $\xi_x(p) = p(x)$ и рассмотрим систему векторов $\{\xi_x(p) : x \in \R\}$; докажем, что она линейно независима:
    \[
        \sum_{i \in I}\lambda_i\xi_{x_i} = 0.
    \]

    Подставим поочерёдно многочлены $p_j$, корнями которого являются все $x_i$, кроме $j$-го (можно, т.\,к. почти все $\lambda_i$ нулевые). Итак для каждого $j$ получим $\lambda_j = 0$. Значит, наша система линейно независима, и в $(\R[t])^\ast$ нет счётного базиса.
\end{solution}

\begin{problem}
    $(V_\R)_\C$ канонически изоморфно $V \oplus \overline{V}$, где $\overline{V}$ --- комплексно сопряжённое пространство, в котором сложение то же, что и в $V$, а умножение $\ast$ на скаляры определяется как $\lambda \ast v \vcentcolon = \overline{\lambda}v$.
\end{problem}

\begin{solution}
    Так как $\C$ --- алгебраически замкнутое поле, комплексная структура $\mathcal{J}$ должна иметь $n = \dim (V_\R)_\C$ собственных значений, каждое из которых удовлетворяет $\lambda^2 = -1$, т.\,е. $\lambda = \pm 1$. Значит, можем записать $(V_\R)_\C = V^+ \oplus V^-$, где $V^+$ и $V^-$ --- собственные подпространства для $+i$ и $-i$ соответственно. Первое из них изоморфно $V$, а второе --- $\overline{V}$.
\end{solution}

\begin{problem}
    Найдите все инвариантные подпространства оператора, матрица которого есть жорданова клетка $J_\lambda$.
\end{problem}

\begin{solution}
    Для удобства обозначим $e_0 \vcentcolon = \bs{0}$, а через $\A$ --- данный оператор, который в базисе $e_1, \ldots, e_n$ имеет матрицу $J_\lambda$. Тогда $\forall i = 1, \ldots, n$ имеем
    \[
        J_\lambda \cdot e_i =
        \begin{pmatrix}
            \lambda & 1 &  &  &  \\
             & \lambda & 1 &  &  \\
            &  &  \lambda & \ddots &  \\
            &  &  & \ddots & 1 \\
            &  &  &  & \lambda
        \end{pmatrix} \cdot e_i = 
        \lambda e_i + e_{i - 1}.
    \]

    Отсюда следует, что подпространства $\langle e_0, e_1, \ldots, e_i\rangle$ $\forall i = 0, 1, \ldots, n$ являются инвариантными для оператора с матрицей $J_\lambda$.

    Докажем, что других инвариантных подпространств нет. Пусть $V_k$ --- инвариантное подпространство размерности $k \in \{1, \ldots, n - 1\}$. Рассмотрим ограничение оператора $\A$ на это инвариантное подпространство и выберем в нём жорданов базис по алгоритму Антона Александровича, чтобы в нём матрица имела вид
    \[
        \underbrace{\begin{pmatrix}
            \lambda & &  &  &  \\
            1 & \lambda & &  &  \\
            & 1 & \lambda & &  \\
            &  & \ddots & \ddots & \\
            &  &  & 1 & \lambda
        \end{pmatrix}}_{k \times k}.
    \]

    Как я уже писал, то, что единицы стоят снизу, а не сверху, имеет значение, и вот какое. Если единицы стоят сверху, то собственным вектором является $e_1$. А если снизу, то $e_n$. В алгоритме Антона Александровича мы из более высоких векторов получаем более низкие, поэтому хотим, чтобы первым стоял вектор максимальной высоты, а собственный --- последним. Поэтому мы и единицы ставим внизу. А до этого я их ставил вверху, чтобы было поэстетичнее. Правда, теперь у нас будет <<перевёрнутая>> нумерация, но в целом такие вещи полезны для тренировки понимания того, что происходит.

    Итак, в качестве $e_1^\prime$ берём случайный вектор, например, $e_k$. Конечно, мы взяли не совсем случайный вектор, а заведомо хороший, но будем считать, что нам просто повезло. Следующий вектор получаем следующим образом:
    \[
        (J_\lambda - \lambda E)e_1^\prime = J_\lambda e_k - \lambda e_k = \cancel{\lambda e_k} + e_{k - 1} - \cancel{\lambda e_k} = e_{k - 1}.
    \]

    И дальше так можем продолжать, пока не дойдём до вектора $e_k^\prime = e_1$. Таким образом, $V_k = \langle e_1^\prime, \ldots, e_k^\prime\rangle = \langle e_k, \ldots, e_1\rangle$.
\end{solution}

\begin{problem}
    Докажите, что для любого оператора $\A$ существует оператор $\B$ такой, что $\A\B\A = \A$.
\end{problem}

\begin{solution}
    Выберем базис $e_1, \ldots, e_k$ в $\Im\A$ и дополним его до базиса $e_1, \ldots, e_k, e_{k + 1}, \ldots, e_n$ пространства $V$. Существуют векторы $f_1, \ldots, f_k$ такие, что $\A f_i = e_i$ $\forall i = 1, \ldots, k$. Определим оператор $\B$ на базисных векторах $e_1, \ldots e_k$ так, чтобы $\B e_i = f_i$ $\forall i = 1, \ldots, k$, а на остальных базисных векторах можно его доопределить произвольным образом. Получаем
    \[
        \A(\B(\A v)) = \A(\B \lambda^ie_i) = \A(\lambda^if_i) = \lambda^ie_i = \A v
    \]
    $\forall v \in V$, так что $\A\B\A = \A$. Заметим, что если оператор $\A$ невырожденный, то сконструированный нами оператор $\B$ есть в точности $\A^{-1}$.
\end{solution}

\begin{problem}
    Докажите, что в конечномерном пространстве над полем нулевой характеристики не существует операторов $\A$ и $\B$, удовлетворяющий соотношению $\A\B - \B\A = \id$. Существуют ли такие операторы над полем положительной характеристики?
\end{problem}

\begin{solution}
    В конечномерных пространствах над полем нулевой характеристики такого не бывает. Действительно, взяв след от левой части равенства, получим $\tr(\B\A - \A\B) = \tr\B\tr\A - \tr\A\tr\B = 0$. С другой, стороны, берём след от правой части равенства, получаем $\tr\id = n$.

    Ясно, что равенство $n = 0$ может выполняться при $\operatorname{char}\F = \dim V = n$. Чтобы сильно не задумываться, возьмём поле $\Z_2$ и небольшим перебором найдём матрицы операторов, удовлетворяющие требуемому свойству:
    \[
        \begin{pmatrix}
            0 & 1\\
            1 & 0
        \end{pmatrix} \cdot
        \begin{pmatrix}
            1 & 1\\
            0 & 1
        \end{pmatrix} -
        \begin{pmatrix}
            1 & 1\\
            0 & 1
        \end{pmatrix} \cdot
        \begin{pmatrix}
            0 & 1\\
            1 & 0
        \end{pmatrix} =
        \begin{pmatrix}
            0 & 1\\
            1 & 1
        \end{pmatrix} -
        \begin{pmatrix}
            1 & 1\\
            1 & 0
        \end{pmatrix} =
        \begin{pmatrix}
            1 & 0\\
            0 & 1
        \end{pmatrix}.
    \]

    Так что над полями положительной характеристики такое бывает, а нулевой --- не бывает.
\end{solution}

\begin{problem}
    Постройте пример операторов в бесконечномерном пространстве над $\R$, удовлетворяющий соотношению $\A\B - \B\A = \id$.
\end{problem}

\begin{solution}
    Рассмотрим векторное пространство $\R[t]$ (оно счётномерно). Обозначим через $\A$ оператор дифференцирования, а через $\B$ оператор повышения степени $x^k \mapsto x^{k + 1}$, $k = 0, 1, 2, \ldots$ В силу линейности достаточно проверять требуемое тождество на базисных векторах $x^k$, где $k \in \N$:
    \[
        \A(\B x^k) - \B(\A x^k) = \A(x^{k + 1}) - \B(kx^{k - 1}) = (k + 1)x^k - kx^k = x^k
    \]
    и отдельно для $1$ ($= x^0$):
    \[
        \A(\B 1) - \B(\A 1) = \A x - \B 0 = 1.
    \]
\end{solution}

\begin{problem}
    Пусть $\A$, $\B$ --- операторы. Докажите, что характеристические многочлены операторов $\A\B$ и $\B\A$ совпадают.
\end{problem}

\begin{solution}
    Докажем более сильное утверждение: если $A \in \underset{m \times n}{\Mat}(\F)$ и $B \in \underset{n \times m}{\Mat}(\F)$, то характеристические многочлены матриц $AB$ и $BA$ удовлетворяют следующему условию:
    \[
        t^n\det(t \cdot \underset{m \times m}{E} - AB) = t^m\det(t \cdot \underset{n \times n}{E} - BA).
    \]
    Определим 
    \[
        C \vcentcolon =
        \begin{pmatrix}
            t \cdot \underset{m \times m}{E} & A\\
            B & \underset{n \times n}{E}
        \end{pmatrix},\quad
        D \vcentcolon = 
        \begin{pmatrix}
            \underset{m \times m}{E} & 0\\
            -B & t \cdot \underset{n \times n}{E}
        \end{pmatrix}.
    \]
    Тогда
    \begin{gather*}
        \det CD = t^n\det(t \cdot \underset{m \times m}{E} - AB),\\
        \det DC = t^m\det(t \cdot \underset{n \times n}{E} - BA).
    \end{gather*}

    С учётом $\det CD = \det DC$ получаем требуемое.
\end{solution}

\begin{problem}
    Докажите, что оператор $\id + \A\B$ обратим тогда и только тогда, когда $\id + \B\A$ обратим.
\end{problem}

\begin{solution}
    Пусть оператор $\id + \A\B$ вырожденный, тогда найдётся $x \in V \setminus \{\bs{0}\}$ такой, что $(\id + \A\B)x = \bs{0}$. Заметим, что $\B x \ne \bs{0}$ для такого, $x$, потому что иначе
    \[
        (\id + \A\B)x = \id x + \A\underbrace{(\B x)}_{\bs{0}} = x \ne \bs{0}.
    \]
    Так что $y = \B x \ne \bs{0}$. Причём,
    \[
        (\id + \B\A)y = y + \B\underbrace{(\A\B x)}_{-x} = y - \B x = y - y = \bs{0}.
    \]

    Таким образом, оператор $\id + \B\A$ тоже вырожден. Аналогично доказывается обратное утверждение. Итак, оператор $\id + \A\B$ вырожден тогда и только тогда, когда оператор $\id + \B\A$ вырожден. Последнее равносильно утверждению задачи.
\end{solution}

\begin{problem}
    Докажите, что если $\tr\A^k = 0$ для любого $k > 0$ над полем нулевой характеристики, то оператор $\A$ нильпотентен.
\end{problem}

\begin{solution}
    Будем доказывать индукцией по размерности $n$ векторного пространства. Для $n = 1$ $\tr\A = 0$ $\Leftrightarrow$ $\A = \O$ --- нильпотентный. Теперь предположим, что $n > 1$ и для всех размерностей, меньших $n$, утверждение доказано. Рассмотрим минимальный многочлен $\mu_\A(t) = p_0 + p_1t + p_2t^2 + \ldots + p_mt^m$ оператора $\A$, тогда в силу линейности следа выполнено
    \[
        0 = \tr\mu_\A(\A) = p_0\tr E + p_1\underbrace{\tr\A}_{\bs{0}} + p_2\underbrace{\tr\A^2}_{\bs{0}} + \ldots + p_m\underbrace{\tr\A^m}_{\bs{0}},
    \]
    отсюда $p_0 = 0$. А значит, $0$ является корнем минимально (а значит, и характеристического) многочлена, т.\,е. собственным значением оператора $\A$. Пусть $R_0$ --- соответствующее корневое подпространство. На нём оператор $\A\big|_{R_0}$ нильпотентен (по лемме 1 в вопросе 20). Пусть $e_1, \ldots, e_k$ --- базис $R_0$, дополним его до базиса всего пространства $V$ векторами $e_{k + 1}, \ldots, e_n$. Матрица оператора $\A$ в этом базисе имеет вид
    \[
        A =
        \begin{pmatrix}
            \widehat{A} & \ast\\
            0 & \widetilde{A}
        \end{pmatrix},\quad\underbrace{\tr A}_0 = \underbrace{\tr\widehat{A}}_0 + \tr\widetilde{A}
    \]
    (в силу предложения 1 в вопросе 12), где $\widehat{A}$ --- матрица ограничения, а $\widetilde{A}$ --- матрица фактор-оператора. Как мы уже поняли, матрица $\widehat{A}$ нильпотентна, а матрица $\widetilde{A}$ нильпотентна в силу предположения индукции, т.\,к. $\tr\widetilde{A} = 0$ и $\codim R_0 < n$. 
    
    \begin{lemma}
        Если матрица имеет вид
        \[
            M =
            \begin{pmatrix}
                A & B\\
                0 & C
            \end{pmatrix},
        \]
        где $A$ и $C$ нильпотентны, то $M$ нильпотентна.
    \end{lemma}

    \begin{proof}
        Докажем по индукции, что
        $
        M^l =
        \begin{pmatrix}
            A^l & \ast\\
            0 & C^l
        \end{pmatrix}
        $ для любого $l \in \N$. При $l = 1$ доказывать нечего. Теперь пусть доказано для $l$, докажем для $l + 1$:
        \[
            M^{l + 1} = M^l \cdot M =
            \begin{pmatrix}
                A^l & \ast\\
                0 & C^l
            \end{pmatrix} \cdot
            \begin{pmatrix}
                A & B\\
                0 & C
            \end{pmatrix} =
            \begin{pmatrix}
                A^{l + 1} & A^lB + \ast C\\
                0 & C^{l + 1}
            \end{pmatrix}.
        \]

        Теперь пусть $A$ размера $m \times m$, $C$ размера $k \times k$, тогда $A^m = \O$, $C^k = \O$ (степень нильпотентности не превосходит размерности), $\forall v \in V$ получаем:
        \[
            M^{m + k}v = M^m \cdot
            \begin{pmatrix}
                A^k & \ast\\
                0 & 0
            \end{pmatrix}v.
        \]
        По сути, мы умножаем матрицу $M^m$ на вектор, у которого последние $k$ координат равны нулю. Поэтому по сути там только матрица $A^m = 0$ умножается на вектор $v$ высоты $m$, получается $\bs{0}$.
    \end{proof}

    Наша матрица $A$ именно такая, так что решение можно на этом завершить.
\end{solution}

\begin{problem}
    Докажите, что если операторы $\A$ и $\B$ над полем нулевой характеристики удовлетворяют соотношению $\A = \A\B - \B\A$, то оператор $\A$ нильпотентен.
\end{problem}

\begin{solution}
    Возведём данное равенство в степень $k \in \N$:
    \begin{gather*}
        \A^k = (\A\B - \B\A)^k\\
        \tr\A^k = \tr(\A\B - \B\A)^k\\
        \tr\A^k = \sum_{k = 0}^n(-1)^kC_n^k\tr((\A\B)^{n - k}(\B\A)^k)\\
        \tr\A^k = \sum_{k = 0}^n(-1)^kC_n^k\tr((\A\B)^n) = \tr((\A\B)^n)\sum_{k = 0}^n(-1)^kC_n^k = \tr((\A\B)^n)\underbrace{(1 - 1)^n}_{0} = 0.
    \end{gather*}
    Равенство верно в силу того, что $\tr\A\B = \tr\B\A$ и под операцией следа можно считать, что операторы $\A$ и $\B$ коммутируют. Теперь $\A$ нильпотентен по предыдущей задаче.
\end{solution}

\begin{problem}
    Пусть операторы $\A$ и $\B$ в пространстве над $\C$ коммутируют, т.\,е. $\A\B = \B\A$. Докажите, что они имеют общий собственный вектор.
\end{problem}

\begin{solution}
    Пусть $\lambda$ --- собственное значение оператора $\A$, а $V_\lambda$ --- соответствующее ему собственное подпространство. Как известно, оно инвариантно относительно $\A$. Тогда оно инвариантно и относительно оператора $\B$. Действительно, пусть $v \in V_\lambda$, тогда
    \[
        \A(\B v) = \B(\A v) = \B(\lambda x) = \lambda(\B v),
    \]
    следовательно, $\B v \in V_\lambda$. Оператор $\B\big|_{V_\lambda}$ имеет собственный вектор $u \in V_\lambda$: $\B u = \mu u$, $\mu$ --- собственное значение $\B$. Таким образом, $\A u = \lambda u$, $\B u = \mu u$, т.\,е. $u$ --- общий собственный вектор.
\end{solution}

\begin{problem}
    Докажите, что проектор на корневое подпространство $R_\lambda$ оператора $\A$ является многочленом от оператора $\A$.
\end{problem}

\renewcommand{\N}{\mathcal{N}}

\begin{problem}
    Докажите, что любой оператор $\A$ над полем $\C$ можно представить в виде $\A = \D + \N$, где $\D$ --- диагонализируемый оператор, а $\N$ --- нильпотентный оператор, причём операторы $\D$ и $\N$ коммутируют.
\end{problem}

\begin{solution}
    Рассмотрим жорданову форму $J_\A$ оператора $\A$. Она очевидным образом раскладывается в сумму $J_\A = D + N$ диагонали и нильпотента. Пусть $\D$ --- оператор, соответствующий матрице $D$ в жордановом базисе оператора $\A$, $\N$ --- оператор, соответствующий матрице $N$ в том же базисе. Тогда $\A = \D + \N$. Операторы $\D$ и $\N$ коммутируют в силу леммы 3 из билета 21.
\end{solution}

\begin{problem}
    Докажите, что операторы $\D$ и $\N$ из предыдущей задачи являются многочленами от оператора $\A$.
\end{problem}

\begin{solution}
    В жордановом базисе оператора $\A$ матрица оператора $\D$ есть диагональ матрицы $J_\A$ оператора $\A$ в этом базисе, а по лемме 2 из билета 21 диагональ является многочленом от матрицы. Отсюда, $\N = \A - \D$ --- тоже многочлен от $\A$, т.\,к. это есть разность двух многочленов.
\end{solution}

\renewcommand{\N}{\mathbb{N}}

\begin{problem}
    Докажите, что если оператор $\B$ коммутирует с любым оператором, коммутирующим с $\A$, то $\B$ есть многочлен от $\A$.
\end{problem}

\begin{solution}
\end{solution}

\begin{problem}
    Докажите, что $QR$-разложение имеет место для любых прямоугольных матриц. А именно, докажите, что любую вещественную матрицу $A$ размера $m \times n$ можно представить в виде $A = QR$, где $Q$ --- ортогональная матрица размера $m \times m$, а $R = (r^i_j)$ --- верхнетреугольная матрица размера $m \times n$ с неотрицательными числами на <<диагонали>> (т.\,е. $r^i_j = 0$ при $i > j$ и $r^i_i \geqslant 0$).
\end{problem}

\begin{problem}
    Докажите, что для любых линейно независимых векторов $a_1, \ldots, a_k$ евклидова пространства найдётся вектор $b$, что $(a_i, b) > 0$ при $i = 1, \ldots, k$.
\end{problem}

\begin{solution}
    Этот вектор мы будем искать в подпространстве $W = \langle a_1, \ldots, a_k\rangle$. Вектора $a_1, \ldots, a_k$ образуют базис этого пространства, так что $\forall x \in W$ имеем $b = x^ia_i$ для некоторых $x^i \in \R$. Пусть $(a_i, x) = \vcentcolon y^i$, $i = 1, \ldots, k$. Получаем систему линейных уравнений:
    \[
        \begin{cases}
            (a_1, x) = y_1,\\
            (a_2, x) = y_2,\\
            \dotfill\\
            (a_k, x) = y_k
        \end{cases}\quad
        \begin{cases}
            (a_1, a_1)x^1 + (a_1, a_2)x^2 + \ldots + (a_1, a_k)x^k = y^1,\\
            (a_2, a_1)x^1 + (a_2, a_2)x^2 + \ldots + (a_2, a_k)x^k = y^2,\\
            \dotfill\\
            (a_k, a_1)x^1 + (a_k, a_2)x^2 + \ldots + (a_k, a_k)x^k = y^k.
        \end{cases}
    \]

    Получили систему $G(a_1, \ldots, a_k)x^i = y^i$. Известно, что матрица Грама линейно независимой системы векторов невырожденна, а потому система имеет решение. Заметим, что мы доказали намного более сильное утверждение. Из рассуждений выше следует, что можно добиться вообще любых скалярных произведений.
\end{solution}

\begin{problem}
    Пусть $V$ --- пространство и $\gamma_1, \ldots, \gamma_m \in V^\ast$ --- набор линейных функций. Докажите, что отображение
    \[
        V \to V^\ast,\quad v \mapsto \langle\gamma_1, v\rangle\gamma_1 + \ldots + \langle\gamma_m, v\rangle\gamma_m
    \]
    является мономорфизмом тогда и только тогда, когда $\gamma_1, \ldots, \gamma_m$ порождают пространство $V^\ast$.
\end{problem}

\begin{problem}
    Пусть подпространство $W$ в $\R^n$ задано как линейная оболочка: $W = \langle a_1, \ldots, a_k\rangle$, причём вектор-столбцы $a_1, \ldots, a_k$ линейно независимы. Найдите матрицу ортогонального проектора $\pr_W$ в стандартном базисе.
\end{problem}

\begin{problem}
    Докажите, что симметричная положительно определённая матрица $A$ представляется в виде произведения $A = LR$, где $L$ и $R$ --- соответственно нижнетреугольная и верхнетреугольная матрицы.
\end{problem}

\begin{problem}
    Докажите, что для любого (не обязательно невырожденного) оператора $\A$ имеет место полярное разложение $\A = \P\U$, где $\P$ --- неотрицательный самосопряжённый оператор, а $\U$ --- ортогональный (унитарный) оператор.
\end{problem}

\begin{problem}
    Докажите, что в полярных разложениях $\A = \P_1\U_1 = \U_2\P_2$ невырожденного оператора $\A$ ортогональные (унитарные) операторы $\U_1$ и $\U_2$ совпадают.
\end{problem}

\begin{problem}
    Пусть $\A: V \to V$ нормальный оператор (т.\,е. $\A^\ast\A = \A\A^\ast$) и $W$ --- его инвариантное подпространство. Верно ли, что $W^\perp$ также инвариантно относительно $\A$?
\end{problem}

\begin{problem}
    Докажите, что для нормального оператора евклидова пространства существует ортонормированный базис, в котором его матрица состоит из блоков размера $1$ или $2$, причём блоки размера $2$ имеют вид
    $
        \begin{pmatrix}
            a & -b\\
            b & a
        \end{pmatrix}
    $.
\end{problem}

\begin{problem}
    Пусть билинейная функция $\B$ удовлетворяет условию: $\B(x, y) = 0$ тогда и только тогда когда $\B(y, x) = 0$. Докажите, что $\B$ либо симметрична, либо кососимметрична.
\end{problem}

\begin{problem}
    Приведите пример симметрической билинейной функции над полем $\Z_2$, которая не приводится к диагональному виду заменой базиса.
\end{problem}

\begin{solution}
    См. задачу 7.
\end{solution}

\begin{problem}
    Приведите пример пары симметрических билинейных форм, которые нельзя привести к диагональному виду одним преобразованием.
\end{problem}

\begin{problem}
    Пусть $\xi, \eta$ --- ненулевые линейные функции. Найдите ранг билинейной формы $\xi \otimes \eta$.
\end{problem}

\begin{problem}
    Найдите размерность пространства симметрических тензоров $\mathrm{S}^p(V)$.
\end{problem}

\begin{problem}
    Пусть $\xi^1, \ldots, \xi^p \in \T_1^0(V) = \Lambda^1(V)$. Докажите, что $\xi^1 \wedge \ldots \wedge \xi^p = 0$ тогда и только тогда, когда ковекторы $\xi^1, \ldots, \xi^p$ линейно зависимы.
\end{problem}

\begin{solution}
    $\Rightarrow$. Если ковекторы $\xi^1, \ldots, \xi^p$ образуют линейно зависимую систему, то один из них, есть линейная комбинация всех остальных. При его замене этой внешней комбинацией внешнее произведение $\xi^1 \wedge \ldots \wedge \xi^p$ раскладывается в сумму внешних произведений, каждое из которых содержит по два одинаковых множителя. Значит, каждое из них нулевое, в итоге получаем $0$.

    $\Leftarrow$. Если, напротив, $\xi^1, \ldots, \xi^p$ линейно независимы, то мы можем из принять за первые $p$ векторов некоторого базиса. Тогда $p$-форма $\xi^1 \wedge \ldots \wedge \xi^p$ будет элементом соответствующего базиса $\Lambda^p(V)$, и поэтому не равен нулю.
\end{solution}

\begin{problem}
    Внешняя 2-форма $T = \sum\limits_{i < j}T_{ij}\varepsilon^i \wedge \varepsilon^j$ называется \textit{разложимой}, если $T = \xi \wedge \eta$ для некоторых линейных функций (1-форм) $\xi$ и $\eta$. Докажите, что внешняя 2-форма разложим тогда и только тогда, когда ранг соответствующий кососимметрической билинейной функции не превосходит $2$.
\end{problem}

\begin{solution}
    Сразу заметим, чо ранг кососимметрической билинейной функции всегда чётный. А потому из условия это всегда либо $0$, либо $2$. Ясно, что $0$ --- неинтересный случай, так что далее работаем с рангом $2$. Ранг симметрической билинейной формы равен $2$ тогда и только тогда, когда в некотором базисе она имеет вид
    \[
        \begin{pmatrix}
            0 & 1 & & \\
            -1 & 0 & & \scalebox{2}{$0$} \\
             & & \ddots & \\
             \scalebox{2}{$0$} & & & 0
        \end{pmatrix}.
    \]
    Отсюда видно, что в этом базисе наша билинейная форма раскладывается в произведение 1-форм $\varepsilon^1 \wedge \varepsilon^2$.
\end{solution}

\begin{problem}
    Докажите, что определитель $\det B$ кососимметрической матрицы $B = (b_{ij})$ как многочлен от её элементов $b_{ij}$ является квадратом другого многочлена $\pf B$ с целыми коэффициентами: $\det B = (\pf B)^2$. Многочлен $\pf B$ называется \textit{пфаффианом} матрицы $B$.
\end{problem}

\begin{problem}
    Докажите формулу изменения пфаффиана при замене базиса: если $B^\prime = C^tBC$, то $\pf B^\prime = \det C\pf B$.
\end{problem}

\begin{problem}
    Найдите явный вид многочлена $\pf B$ от элементов $b_{ij}$.
\end{problem}

\begin{problem}
    Пусть $\B: V \times V \to \R$ --- невырожденная кососимметрическая билинейная функция, а $\A: V \to V$ --- оператор, сохраняющий эту функцию, т.\,е. $\B(\A u, \A v) = \B(u, v)$. Докажите, что $\det\A = 1$.
\end{problem}

