\section{Теоретические задачи}

Здесь представлены теоретические задачи Тараса Евгеньевича Панова, к некоторым также написаны решения. Рекомендую проверять их на адекватность.

\begin{problem}
    Докажите, что пространство, двойственное к пространству $\R[t]$ всех многочленов от одной переменной над $\R$, не изоморфно пространству $\R[t]$.
\end{problem}

\begin{solution}
    В пространстве $\R[t]$ есть счётный базис: $1, x^1, x^2, x^3, \ldots$ Докажем, что в пространстве $(\R[t])^\ast$ счётного базиса нет. Для этого найдём в нём несчётную линейно независимую систему. Положим $\xi_x: \R[t] \to \F$, $\xi_x(p) = p(x)$ и рассмотрим систему векторов $\{\xi_x(p) : x \in \R\}$; докажем, что она линейно независима:
    \[
        \sum_{i \in I}\lambda_i\xi_{x_i} = 0.
    \]

    Подставим поочерёдно многочлены $p_j$, корнями которого являются все $x_i$, кроме $j$-го (можно, т.\,к. почти все $\lambda_i$ нулевые). Итак для каждого $j$ получим $\lambda_j = 0$. Значит, наша система линейно независима, и в $(\R[t])^\ast$ нет счётного базиса.
\end{solution}

\begin{problem}
    $(V_\R)_\C$ канонически изоморфно $V \oplus \overline{V}$, где $\overline{V}$ --- комплексно сопряжённое пространство, в котором сложение то же, что и в $V$, а умножение $\ast$ на скаляры определяется как $\lambda \ast v \vcentcolon = \overline{\lambda}v$.
\end{problem}

\begin{solution}
    Так как $\C$ --- алгебраически замкнутое поле, комплексная структура $\mathcal{J}$ должна иметь $n = \dim (V_\R)_\C$ собственных значений, каждое из которых удовлетворяет $\lambda^2 = -1$, т.\,е. $\lambda = \pm 1$. Значит, можем записать $(V_\R)_\C = V^+ \oplus V^-$, где $V^+$ и $V^-$ --- собственные подпространства для $+i$ и $-i$ соответственно. Первое из них изоморфно $V$, а второе --- $\overline{V}$.
\end{solution}

\begin{problem}
    Найдите все инвариантные подпространства оператора, матрица которого есть жорданова клетка $J_\lambda$.
\end{problem}

\begin{solution}
    Для удобства обозначим $e_0 \vcentcolon = \bs{0}$, а через $\A$ --- данный оператор, который в базисе $e_1, \ldots, e_n$ имеет матрицу $J_\lambda$. Тогда $\forall i = 1, \ldots, n$ имеем
    \[
        J_\lambda \cdot e_i =
        \begin{pmatrix}
            \lambda & 1 &  &  &  \\
             & \lambda & 1 &  &  \\
            &  &  \lambda & \ddots &  \\
            &  &  & \ddots & 1 \\
            &  &  &  & \lambda
        \end{pmatrix} \cdot e_i = 
        \lambda e_i + e_{i - 1}.
    \]

    Отсюда следует, что подпространства $\langle e_0, e_1, \ldots, e_i\rangle$ $\forall i = 0, 1, \ldots, n$ являются инвариантными для оператора с матрицей $J_\lambda$.

    Докажем, что других инвариантных подпространств нет. Пусть $V_k$ --- инвариантное подпространство размерности $k \in \{1, \ldots, n - 1\}$. Рассмотрим ограничение оператора $\A$ на это инвариантное подпространство и выберем в нём жорданов базис по алгоритму Антона Александровича, чтобы в нём матрица имела вид
    \[
        \underbrace{\begin{pmatrix}
            \lambda & &  &  &  \\
            1 & \lambda & &  &  \\
            & 1 & \lambda & &  \\
            &  & \ddots & \ddots & \\
            &  &  & 1 & \lambda
        \end{pmatrix}}_{k \times k}.
    \]

    Как я уже писал, то, что единицы стоят снизу, а не сверху, имеет значение, и вот какое. Если единицы стоят сверху, то собственным вектором является $e_1$. А если снизу, то $e_n$. В алгоритме Антона Александровича мы из более высоких векторов получаем более низкие, поэтому хотим, чтобы первым стоял вектор максимальной высоты, а собственный --- последним. Поэтому мы и единицы ставим внизу. А до этого я их ставил вверху, чтобы было поэстетичнее. Правда, теперь у нас будет <<перевёрнутая>> нумерация, но в целом такие вещи полезны для тренировки понимания того, что происходит.

    Итак, в качестве $e_1^\prime$ берём случайный вектор, например, $e_k$. Конечно, мы взяли не совсем случайный вектор, а заведомо хороший, но будем считать, что нам просто повезло. Следующий вектор получаем следующим образом:
    \[
        (J_\lambda - \lambda E)e_1^\prime = J_\lambda e_k - \lambda e_k = \cancel{\lambda e_k} + e_{k - 1} - \cancel{\lambda e_k} = e_{k - 1}.
    \]

    И дальше так можем продолжать, пока не дойдём до вектора $e_k^\prime = e_1$. Таким образом, $V_k = \langle e_1^\prime, \ldots, e_k^\prime\rangle = \langle e_k, \ldots, e_1\rangle$.
\end{solution}

\begin{problem}
    Докажите, что для любого оператора $\A$ существует оператор $\B$ такой, что $\A\B\A = \A$.
\end{problem}

\begin{solution}
    Выберем базис $e_1, \ldots, e_k$ в $\Im\A$ и дополним его до базиса $e_1, \ldots, e_k, e_{k + 1}, \ldots, e_n$ пространства $V$. Существуют векторы $f_1, \ldots, f_k$ такие, что $\A f_i = e_i$ $\forall i = 1, \ldots, k$. Определим оператор $\B$ на базисных векторах $e_1, \ldots e_k$ так, чтобы $\B e_i = f_i$ $\forall i = 1, \ldots, k$, а на остальных базисных векторах можно его доопределить произвольным образом. Получаем
    \[
        \A(\B(\A v)) = \A(\B \lambda^ie_i) = \A(\lambda^if_i) = \lambda^ie_i = \A v
    \]
    $\forall v \in V$, так что $\A\B\A = \A$. Заметим, что если оператор $\A$ невырожденный, то сконструированный нами оператор $\B$ есть в точности $\A^{-1}$.
\end{solution}

\begin{problem}
    Докажите, что в конечномерном пространстве над полем нулевой характеристики не существует операторов $\A$ и $\B$, удовлетворяющий соотношению $\A\B - \B\A = \id$. Существуют ли такие операторы над полем положительной характеристики?
\end{problem}

\begin{solution}
    В конечномерных пространствах над полем нулевой характеристики такого не бывает. Действительно, взяв след от левой части равенства, получим $\tr(\B\A - \A\B) = \tr\B\tr\A - \tr\A\tr\B = 0$. С другой, стороны, берём след от правой части равенства, получаем $\tr\id = n$.

    Ясно, что равенство $n = 0$ может выполняться при $\operatorname{char}\F = \dim V = n$. Чтобы сильно не задумываться, возьмём поле $\Z_2$ и небольшим перебором найдём матрицы операторов, удовлетворяющие требуемому свойству:
    \[
        \begin{pmatrix}
            0 & 1\\
            1 & 0
        \end{pmatrix} \cdot
        \begin{pmatrix}
            1 & 1\\
            0 & 1
        \end{pmatrix} -
        \begin{pmatrix}
            1 & 1\\
            0 & 1
        \end{pmatrix} \cdot
        \begin{pmatrix}
            0 & 1\\
            1 & 0
        \end{pmatrix} =
        \begin{pmatrix}
            0 & 1\\
            1 & 1
        \end{pmatrix} -
        \begin{pmatrix}
            1 & 1\\
            1 & 0
        \end{pmatrix} =
        \begin{pmatrix}
            1 & 0\\
            0 & 1
        \end{pmatrix}.
    \]

    Так что над полями положительной характеристики такое бывает, а нулевой --- не бывает.
\end{solution}

\begin{problem}
    Постройте пример операторов в бесконечномерном пространстве над $\R$, удовлетворяющий соотношению $\A\B - \B\A = \id$.
\end{problem}

\begin{solution}
    Я не особо знаю бесконечномерных пространств над $\R$, кроме $\R[t]$, так что искомый пример лежит где-то там.
\end{solution}

\begin{problem}
    Пусть $\A$, $\B$ --- операторы. Докажите, что характеристические многочлены операторов $\A\B$ и $\B\A$ совпадают.
\end{problem}

\begin{solution}
    Докажем более сильное утверждение: если $A \in \underset{m \times n}{\Mat}(\F)$ и $B \in \underset{n \times m}{\Mat}(\F)$, то характеристические многочлены матриц $AB$ и $BA$ удовлетворяют следующему условию:
    \[
        t^n\det(t \cdot \underset{m \times m}{E} - AB) = t^m\det(t \cdot \underset{n \times n}{E} - BA).
    \]
    Определим 
    \[
        C \vcentcolon =
        \begin{pmatrix}
            t \cdot \underset{m \times m}{E} & A\\
            B & \underset{n \times n}{E}
        \end{pmatrix},\quad
        D \vcentcolon = 
        \begin{pmatrix}
            \underset{m \times m}{E} & 0\\
            -B & t \cdot \underset{n \times n}{E}
        \end{pmatrix}.
    \]
    Тогда
    \begin{gather*}
        \det CD = t^n\det(t \cdot \underset{m \times m}{E} - AB),\\
        \det DC = t^m\det(t \cdot \underset{n \times n}{E} - BA).
    \end{gather*}

    С учётом $\det CD = \det DC$ получаем требуемое.
\end{solution}

\begin{problem}
    Докажите, что оператор $\id + \A\B$ обратим тогда и только тогда, когда $\id + \B\A$ обратим.
\end{problem}

\begin{solution}
\end{solution}

