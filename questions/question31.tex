\section{Вычисление скалярного произведения в координатах. Матрица Грама и её свойства}

\begin{definition}
    \textit{Матрицей Грама} системы векторов $a_1, \ldots, a_k$ называется матрица
    \[
        G = G(a_1, \ldots, a_k) \vcentcolon =
        \begin{pmatrix}
            (a_1, a_1) & (a_1, a_2) & \cdots & (a_1, a_k)\\
            (a_2, a_1) & (a_2, a_2) & \ldots & (a_2, a_k)\\
            \vdots & \vdots & \ddots & \vdots\\
            (a_k, a_1) & (a_k, a_2) & \cdots & (a_k, a_k)
        \end{pmatrix}.
    \]
\end{definition}

Пусть в евклидовом пространстве в некотором базисе заданы два вектора $u = (u^1, \ldots, u^n)$ и $v = (v^1, \ldots, v^n)$. Тогда
\[
    (u, v) = (u^ie_i, v^je_j) = u^iv^j(e_i, e_j) = u^iv^jg_{ij} = (u^1, \ldots, u^n) \cdot G \cdot (v^1, \ldots, v^n)^t,
\]
где $G = G(e_1, \ldots, e_n)$ --- матрица Грама базисных векторов.

\begin{definition}
    Часто матрицу $G = G(e_1, \ldots, e_n)$ называют \textit{матрицей Грама скалярного произведения} $(\bs{\cdot}, \bs{\cdot})$.
\end{definition}

\begin{theorem}
    Матрица $G$ является матрицей Грама линейно независимой системы векторов в евклидовом пространстве тогда и только тогда, когда она
    \begin{enumerate}[nolistsep]
        \item симметричная;
        \item положительно определённая.
    \end{enumerate}
\end{theorem}

\begin{proof}
    $\Rightarrow$. Если $G = (g_{ij})$ --- матрица Грама скалярного произведения $(\bs{\cdot}, \bs{\cdot})$, то $g_{ij} = (e_i, e_j) = (e_j, e_i) = g_{ji}$, т.\,е. $G = G^t$. Так же, $(x, x) = x^tGx > 0$ $\forall x \in V$, поэтому $G$ задаёт положительно определённую квадратичную форму, следовательно по критерию Сильвестра все её угловые миноры положительные.

    $\Leftarrow$. Если $G$ симметричная и положительно определённая, то она задаёт функцию $f: V \times V \mapsto x^tGy$, удовлетворяющую аксиомам скалярного произведения, значит, $G$ --- матрица Грама этого скалярного произведения.
\end{proof}

\begin{proposal}
    Пусть $G$ --- матрица Грама системы векторов $a_1, \ldots, a_k$, а $A = (a^i_j)$ --- матрица, в столбцы которой записаны координаты векторов $a_1, \ldots, a_k$ в некотором ортонормированном базисе. Тогда имеет место соотношение
    $G = \overline{A}^tA$ ($G = A^tA$ в евклидовом пространстве).
\end{proposal}

\begin{proof}
    Это вытекат из закона умножения матриц и формулы для скалярного произведения в ортонормированном базисе.
\end{proof}

