\section{Ортогональные матрицы как матрицы перехода от одного ортонормированного базиса
к другому. Группа $O(n)$ ортогональных матриц порядка $n$.}

\begin{definition}
    Матрица перехода от одного ортонормированного базиса евклидова (соответственно, эрмитова) пространства к ортонормированному базису называется \textit{ортогональной} (соответственно, \textit{унитарной}).
\end{definition}

\begin{proposal}
    Следующие условия эквивалентны:
    \begin{enumerate}[nolistsep]
        \item Матрица $C$ ортогональна (соответственно, унитарна);
        \item $C^tC = E$ (соответственно, $\overline{C}^tC = E$);
        \item Столбцы матрицы $C$ образуют ортонормированный базис $\R^n$ (соответственно, $\C^n$);
        \item $CC^t = E$ (соответственно, $\overline{C}C^t = E$);
        \item Строки матрицы $C$ образуют ортонормированный базис $\R^n$ (соответственно, $\C^n$).
    \end{enumerate}
\end{proposal}

\begin{proof}
    Условия $2$ и $4$ эквивалентны, т.\,к. каждое из них эквивалентно равенству $C^t = C^{-1}$ (соответственно, $\overline{C}^t = C^{-1}$). Эквивалентности $2 \Leftrightarrow 3$ и $4 \Leftrightarrow 5$ вытекают из правила умножения матриц и формул скалярного произведения в ортонормированных базисах в $\R^n$ и $\C^n$.

    Докажем импликацию $1 \Rightarrow 2$. Пусть $e_1, \ldots, e_n$ и $e_{1^\prime}, \ldots, e_{n^\prime}$ --- два ортонормированных базиса в эрмитовом пространстве и $C = (c^i_{i^\prime})$ --- матрица перехода, т.\,е. $e_{i^\prime} = c^i_{i^\prime}e_i$. Тогда $(e_i, e_j) = \delta_{ij}$ и $(e_{i^\prime}, e_{j^\prime}) = \delta_{i^\prime j^\prime}$, откуда
    \[
        \delta_{i^\prime j^\prime} = (e_{i^\prime}, e_{j^\prime}) = (c^i_{i^\prime}e_i, c^j_{j^\prime}e_j) = \overline{c^i_{i^\prime}}c^j_{j^\prime}(e_i, e_j) = \overline{c^i_{i^\prime}}c^j_{j^\prime}\delta_{ij} = \overline{c^i_{i^\prime}}\delta_{ij}c^j_{j^\prime}.
    \]

    Согласно правилу умножения матриц, это эквивалентно матричному соотношению $E = \overline{C}^tEC$ или $\overline{C}^tC = E$. Случай евклидова пространства рассматривается аналогично (убрать чёрточки над буквами).

    Осталось доказать импликацию $2 \Rightarrow 1$. Пусть имеет место тождество $\overline{C}^tC = E$ или, в обозначениях Эйнтшейна, $\overline{c^i_{i^\prime}}\delta_{ij}c^j_{j^\prime}$. Возьмём произвольный ортонормированный базис $e_1, \ldots, e_n$. Из соотношения $\overline{C}^tC = E$ вытекает, что матрица $C$ невырождена, и поэтому можно рассмотреть новый базис $e_{1^\prime}, \ldots, e_{n^\prime}$, где $e_{i^\prime} \vcentcolon = c^i_{i^\prime}e_i$. Тогда аналогично выкладке выше мы получаем $(e_{i^\prime}, e_{j^\prime}) = \overline{c^i_{i^\prime}}\delta_{ij}c^j_{j^\prime} = \delta_{i^\prime j^\prime}$, т.\,е. базис $e_{1^\prime}, \ldots, e_{n^\prime}$ тоже ортонормирован, и $C$ --- матрица перехода от одного ортонормированного базиса к другому.
\end{proof}

\begin{definition}
    \textit{Подгруппа ортогональных матриц порядка $n$} в группе $\underset{n \times n}{\Mat}(\R)$ обозначается через $O(n)$.
\end{definition}

