\section{Симметрические и кососимметрические тензоры типов $(p, 0)$ или $(0, q)$. Симметризация и альтернирование}

Здесь мы будем рассматривать только тензоры с нижними индексами.

\begin{definition}
    Полилинейная функция $\T \in \mathbb{P}_p^0(V)$ называется \textit{симметрической}, если
    \[
        \T(v_{\sigma(1)}, \ldots, v_{\sigma(p)}) = \T(v_1, \ldots, v_q),
    \]
    и \textit{кососимметрической}, если
    \[
        \T(v_{\sigma(1)}, \ldots, v_{\sigma(p)}) = \sgn\sigma\T(v_1, \ldots, v_q)
    \]
    $\forall v_1, \ldots, v_p \in V$, $\forall \sigma \in S_p$.
\end{definition}

Компоненты тензора, соответствующео симметрической полилинейной функции, удовлетворяют соотношениям
\[
    T_{i_1, \ldots, i_p} = T_{i_{\sigma(1)}, \ldots, i_{\sigma_p}},
\]
а кососимметрической ---
\[
    T_{i_1, \ldots, i_p} = \sgn\sigma T_{i_{\sigma(1)}, \ldots, i_{\sigma_p}}.
\]

Такие тензоры называются, соответственно, \textit{симметрическими} и \textit{кососимметрическими}. В частности, у кососимметрического тензора отличны от нуля могут быть лишь те компоненты $T_{i_1, \ldots, i_p}$, у которых все индексы различны.

Симметрические и кососимметрические тензоры образуют подпространства в пространстве тензоров $\Ten_p^0(V)$, которые обозначаются $S^p(V)$ и $\Lambda^p(V)$ соответственно.

\begin{definition}
    \textit{Симметризацией} называется линейный оператор $\Sym: \Ten_p^0(V) \to \Ten_p^0(V)$, который тензору $T \in \Ten_p^0(V)$ ставит в соответствие тензор $\Sym T$ с компонентами
    \[
        (\Sym T)_{i_1, \ldots, i_p} \vcentcolon = \frac{1}{p!}\sum_{\sigma \in S_p}T_{i_{\sigma(1)}, \ldots, i_{\sigma(p)}}.
    \]
\end{definition}

\begin{definition}
    \textit{Альтернированием} называется линейный оператор $\Alt: \Ten_p^0(V) \to \Ten_p^0(V)$, который тензору $T \in \Ten_p^0(V)$ ставит в соответствие тензор $\Alt T$ с компонентами
    \[
        (\Alt T)_{i_1, \ldots, i_p} \vcentcolon = \frac{1}{p!}\sum_{\sigma \in S_p}\sgn\sigma T_{i_{\sigma(1)}, \ldots, i_{\sigma(p)}}.
    \]
\end{definition}

Легко видеть, что $\Sym T$ --- симметрический тензор, а $\Alt T$ --- кососимметрический $\forall T \in \Ten_p^0(V)$.

\begin{proposal}
    Операторы $\Sym$ и $\Alt$ являются проекторами на подпространства $S^p(V)$ и $\Lambda^p(V)$.
\end{proposal}

\begin{proof}
    Оба утверждения доказываются аналогично. Докажем второе. Достаточно показать, чо $\Alt T = T$ для любого кососимметрического тензора $T$. Мы имеем:
    \[
        (\Alt T)_{i_1, \ldots, i_p} = \frac{1}{p!}\sum_{\sigma \in S_p}\sgn\sigma T_{i_{\sigma(1)}, \ldots, i_{\sigma(p)}} = \frac{1}{p!}\sum_{\sigma \in S_p}T_{i_1, \ldots, i_p} = T_{i_1, \ldots, i_p},
    \]
    где второе равенство выполнено в силу кососимметричности $T$.
\end{proof}

