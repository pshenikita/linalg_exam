\section{Векторные подпространства, равносильность двух способов их задания. Сумма и пересечение подпространств. Формула Грассмана}

\begin{definition}
    Непустое подмножество $W \subseteq V$ линейного пространства $V$ называется \textit{подпространством}, если $\forall u, v \in W$, $\forall \lambda \in \F$ выполнено $(u + v) \in W$ и $\lambda u \in W$.
\end{definition}

\begin{lemma}[Свойство монотонности размерности]
    Подпространство $W$ конечномерного пространства $V$ конечномерно, причём $\dim W \leqslant \dim V$ и равенство достигается только при $W = V$.
\end{lemma}

\begin{proof}
    Пусть $\dim V = m$ и $e_1, \ldots, e_m$ --- базис пространства $V$. Если $\dim W > m$, то в $W$ найдётся линейно независимая система $f_1, \ldots, f_n$ с $n > m$. Причём $\{f_1, \ldots, f_n\} \subseteq \langle e_1, \ldots, e_m\rangle = V$, что противоречит лемме о линейной зависимости. Следовательно, $\dim W \leqslant \dim V$.

    Пусть $\dim W = \dim V = m$ и пусть $f_1, \ldots, f_m$ --- базис в $W$. Тогда каждый вектор $v \in V$ линейно выражается через $f_1, \ldots, f_m$, так как иначе получили бы линейно независимую систему $f_1, \ldots, f_m, v$ из $m + 1$ векторов в $V$, что противоречит теореме 1. Следовательно, любой вектор из $V$ лежит в $\langle f_1, \ldots, f_m\rangle = W$, т.\,е. $V \subseteq W$, а обратное включение верно из условия. Итак, получаем $V = W$.
\end{proof}

\begin{proposal}
    Множество всех решений системы однородных линейных уравнений с $n$ неизвестными является подпространством координатного пространства $\F^n$.
\end{proposal}

\begin{proof}
    Рассмотрим произвольную систему однородных линейных уравнений:
    \[
        \begin{cases}
            a_{11}x_1 + a_{12}x_2 + \ldots + a_{1n}x_n = 0,\\
            a_{21}x_1 + a_{22}x_2 + \ldots + a_{2n}x_n = 0,\\
            \dotfill\\
            a_{m1}x_1 + a_{m2}x_2 + \ldots + a_{mn}x_n = 0.\\
        \end{cases}
    \]

    Очевидно, что нулевой столбец является её решением и что произведение любого решения на число также является решением. Докажем, что сумма решений $(u_1, \ldots, u_n)^t$ и $(v_1, \ldots, v_n)^t$ является решением. Подставляя её компоненты в $i$-ое уравнение системы, получаем
    \[
        a_{i1}(u_1 + v_1) + \ldots + a_{in}(u_n + v_n) = \underbrace{a_{i1}u_1 + \ldots + a_{in}u_n}_{= 0} + \underbrace{a_{i1}v_1 + \ldots + a_{in}v_n}_{= 0} = 0.
    \]
\end{proof}

\begin{definition}
    \textit{Фундаментальная система решений} --- это базис подпространства решений однородной СЛУ.
\end{definition}

\begin{proposal}
    Линейная оболочка $\langle v_i : i \in I \rangle$ является линейным подпространством в $V$. Более того, она является наименьшим по включению линейным подпространством, содержащим все векторы системы $\{v_i : i \in I\}$.
\end{proposal}

\begin{proof}
    Сумма векторов системы и результат умножения вектора системы на скаляр представляются линейными комбинациями и потому принадлежат линейной оболочке. Следовательно, $\langle v_i : i \in I\rangle$ --- подпространство. Если $W$ --- подпространство, содержащее все векторы из системы $\{v_i : i \in I\}$, то $W$ также содержит все векторы, представляющиеся их линейными комбинациями, а значит, $W \supseteq \langle v_i : i \in I\rangle$.
\end{proof}

Очевидно, что верно и обратное --- любое подпространство является линейной оболочкой (например, своих базисных векторов).

\begin{theorem}
    Способы задания подпространства с помощью однородной системы линейных уравнений и линейной оболочки равносильны.
\end{theorem}

Нам понадобится следующая лемма.

\begin{lemma}
    Пусть матрица $B$ состоит из столбцов фундаментальной системы решений системы $Ax = 0$ (где $x$ --- вектор). Тогда линейная система $B^ty = 0$ задаёт линейную оболочку строк матрицы $A$.
\end{lemma}

\begin{proof}
    Поскольку каждый столбец матрицы $B$ является решением системы $Ax = 0$, имеет место матричное равенство $AB = 0$, которое эквивалентно $B^tA^t = 0$. Таким образом, если матрицу $B^t$ интерпретировать как матрицу коэффициентов некоторой линейной системы, все столбцы матрицы $A^t$ (строки $A$) будут ей удовлетворять.

    Допустим, что некоторый столбец, не принадлежащий линейной оболочке столбцов матрицы $A^t$, тоже удовлетворяет системе $B^ty = 0$. Тогда рассмотрим матрицу $C^t$, которая получается дописыванием к матрице $A$ этого столбца справа; полученная матрица будет удовлетворять соотношению $B^tC^t = 0$, а следовательно, и соотношению $CB = 0$. Это означает, что столбцы матрицы $B$ являются решениями не только линейной системы $Ax = 0$, но и системы $Cx = 0$, отличающейся от системы $Ax = 0$ одним добавленным уравнением, которое по предположению линейно не выражается через исходные уравнения. Это означает, что ранг матрицы $C$ на единицу больше ранга матрицы $A$, т.\,е. количество свободных неизвестных у системы $Cx = 0$ на единицу меньше, чем у системы $Ax = 0$. Значит, все столбцы матрицы $B$ не могут быть решениями системы $Cx = 0$ --- противоречие. Таким образом, системе $B^ty = 0$ удовлетворяют все линейные комбинации строк матрицы $A$, и притом только они.
\end{proof}

Теперь докажем теорему 1:

\begin{proof}
    Пусть подпространство задано однородной системой линейных уравнений
    \[
        \begin{cases}
            a_{11}x_1 + a_{12}x_2 + \ldots + a_{1n}x_n = 0,\\
            a_{21}x_1 + a_{22}x_2 + \ldots + a_{2n}x_n = 0,\\
            \dotfill\\
            a_{m1}x_1 + a_{m2}x_2 + \ldots + a_{mn}x_n = 0.
        \end{cases}
    \]
    Тогда задать его линейной оболочкой можно, найдя ФСР, здесь приведён алгоритм, как это делать. С помощью элементарных преобразований приведём систему к улучшенному ступенчатому виду. Число ненулевых уравнений в этом ступенчатом виде равно $r = \rk A$. Поэтому общее решение будет содержать $r$ главных неизвестных и с точностью до перенумерации неизвестных будет иметь вид
    \[
        \begin{cases}
            x_1 = c_{11}x_{r + 1} + c_{12}x_{r + 2} + \ldots + c_{1, n - r}x_n,\\
            x_2 = c_{21}x_{r + 1} + c_{22}x_{r + 2} + \ldots + c_{2, n - r}x_n,\\
            \dotfill\\
            x_r = c_{r1}x_{r + 1} + c_{r2}x_{r + 2} + \ldots + c_{r, n - r}x_n.\\
        \end{cases}
    \]
    Придавая поочерёдно одному из свободных неизвестных $x_{r + 1}, x_{r + 2}, \ldots, x_n$ значение $1$, а остальным --- $0$, получим следующие решения системы:
    \[
        u_1 = 
        \begin{pmatrix}
            c_{11}\\
            \vdots\\
            c_{r1}\\
            1\\
            0\\
            \vdots\\
            0
        \end{pmatrix},\quad
        u_2 = 
        \begin{pmatrix}
            c_{12}\\
            \vdots\\
            c_{r2}\\
            0\\
            1\\
            \vdots\\
            0
        \end{pmatrix},\quad\ldots,\quad
        u_{n - r} = 
        \begin{pmatrix}
            c_{1, n - r}\\
            \vdots\\
            c_{r, n - r}\\
            0\\
            0\\
            \vdots\\
            1
        \end{pmatrix}.
    \]
    Ранг системы векторов $\{u_1, u_2, \ldots, u_{n - r}\}$ равен рангу матрицы
    \[
        \br{
            \begin{array}{c c c c | c c c c}
                c_{11} & c_{21} & \ldots & c_{r1} & 1 & 0 & \ldots & 0\\
                c_{12} & c_{22} & \ldots & c_{r2} & 0 & 1 & \ldots & 0\\
                \vdots & \vdots & \ddots & \vdots & \vdots & \vdots & \ddots & \vdots\\
                c_{1, n - r} & c_{2, n - r} & \ldots & c_{r, n - r} & 0 & 0 & \ldots & 1
            \end{array}
        }
    \]

    Если поменять местами блоки, отделённые чертой, то получится улучшенный ступенчатый вид с количеством ступенек, равным $n - r$. Так что ранг системы векторов $\{u_1, u_2, \ldots, u_{n - r}\}$ равен количеству векторов в этой системе, поэтому она линейно независима. Эта система также порождает всё подпространство решений, т.\,к. любая линейная комбинация вида
    \[
        \lambda_1u_1 + \ldots + \lambda_{n - r}u_{n - r}
    \]
    является решением, в котором свободные неизвестные имеют значения $\lambda_1, \ldots, \lambda_{n - r}$.

    Теперь пусть подпространство задано линейной оболочкой
    \[
        \left\langle
        u_1 = 
        \begin{pmatrix}
            u_{11}\\
            \vdots\\
            u_{1n}
        \end{pmatrix},\quad
        u_2 = 
        \begin{pmatrix}
            u_{21}\\
            \vdots\\
            u_{2n}
        \end{pmatrix},\quad\ldots,\quad
        u_m = 
        \begin{pmatrix}
            u_{m1}\\
            \vdots\\
            u_{mn}
        \end{pmatrix}
        \right\rangle
    \]

    Составим матрицу
    \[
        U = 
        \begin{pmatrix}
            u_{11} & u_{12} & \ldots & u_{1n}\\
            \vdots & \vdots & \ddots & \vdots\\
            u_{m1} & u_{m2} & \ldots & u_{mn}\\
        \end{pmatrix}
    \]
    из строк $u_1^t, u_2^t, \ldots, u_n^t$. Найдём ФСР системы $Ux = 0$ (указанным выше способом), и запишем её векторы по строкам в матрицу $U^\prime$. По лемме 1 пространство решений однородной системы линейных уравнений $U^\prime y = 0$ есть линейная оболочка строк матрицы $U$, а это и есть данные нам векторы.
\end{proof}

\begin{proposal}
    Пересечение $V_1 \cap V_2$ подпространств в $V$ является подпространством в $V$.
\end{proposal}

\begin{proof}
    Во-первых, $\bs{0} \in V_1$ и $\bs{0} \in V_2$, поэтому $\bs{0} \in V_1 \cap V_2 \ne \varnothing$. Во-вторых, $\forall u, v \in V_1 \cap V_2$, $\forall \lambda \in \F$ сумма $u + v$ и произведение $\lambda v$ также лежат в $V_1$ и в $V_2$, а значит, и в $V_1 \cap V_2$.
\end{proof}

\begin{remark}
    Аналогично доказывается, что для любого семейства подпространств $\{U_i : i \in I\}$ их пересечение $\bigcap\limits_{i \in I}U_i$ тоже подпространство.
\end{remark}

\begin{definition}[Сумма подпространств]
    $V_1 + V_2 \vcentcolon = \{v_1 + v_2 : v_1 \in V_1, v_2 \in V_2\}$.
\end{definition}

\begin{proposal}
    $V_1 + V_2 = \langle V_1 \cup V_2\rangle$.
\end{proposal}

\begin{proof}
    Включение $V_1 + V_2 \subseteq \langle V_1 \cup V_2\rangle$ следует из того, что вектор $v_1 + v_2$ является линейной комбинацией векторов $v_1, v_2 \in V_1 \cup V_2$. Докажем обратное включение. Для этого рассмотрим линейную комбинацию $v = \lambda_1u_1 + \ldots + \lambda_nu_n$ векторов $u_1, \ldots, u_n \in V_1 \cup V_2$. Можно считать, что $u_1, \ldots, u_k \in V_1$ и $u_{k + 1}, \ldots, u_n \in V_2$. Тогда мы имеем $v = v_1 + v_2$, где $v_1 = \lambda_1u_1 + \ldots + \lambda_nu_n \in V_1$ и $v_2 = \lambda_{k + 1}u_{k + 1} + \ldots + \lambda_nu_n \in V_2$. Следовательно, $v \in V_1 + V_2$.
\end{proof}

\begin{remark}
    Само объединение $V_1 \cup V_2$ подпространств в общем случае не является подпространством. Примером служит объединение двух прямых на плоскости.
\end{remark}

\begin{theorem}[Формула Грассмана]
    $\dim V_1 + \dim V_2 = \dim(V_1 + V_2) + \dim(V_1 \cap V_2)$.
\end{theorem}

\begin{proof}
    Выберем базис $e_1, \ldots, e_k$ пространства $V_1 \cap V_2$. Воспользовавшись леммой 3 из первого вопроса, можем дополнить его до базиса $e_1, \ldots, e_k, f_1, \ldots, f_l$ пространства $V_1$ и до базиса $e_1, \ldots, e_k, g_1, \ldots, g_m$ пространства $V_2$. Тогда мы имеем $\dim(V_1 \cap V_2) = k$, $\dim V_1 = k + l$, $\dim V_2 = k + m$. Докажем, что $e_1, \ldots, e_k, f_1, \ldots, f_l, g_1, \ldots, g_m$ --- базис пространства $V_1 + V_2$. Заметим, что т.\,к. $V_1 + V_2 = \langle V_1 \cup V_2\rangle$, то любой вектор из $V_1 + V_2$ выражается через эту систему векторов. Остаётся проверить, что эта система линейно независима. Пусть имеет место равенство
    \[
        \lambda_1e_1 + \ldots + \lambda_ke_k + \mu_1f_1 + \ldots + \mu_lf_l + \nu_1g_1 + \ldots + \nu_mg_m = \bs{0}.
    \]
    Перепишем его в виде
    \[
        \lambda_1e_1 + \ldots + \lambda_ke_k + \mu_1f_1 + \ldots + \mu_lf_l = -\nu_1g_1 - \ldots - \nu_mg_m.
    \]
    Вектор, стоящий в обеих частях этого равенства, лежит и в $V_1$, и в $V_2$, а значит, и в $V_1 \cap V_2$, а потому линейно выражается через $e_1, \ldots, e_k$. Т.\,к. векторы $e_1, \ldots, e_k, f_1, \ldots, f_l$ линейно независимы по построению, получаем $\mu_1 = \ldots = \mu_l = 0$. Аналогично, $\nu_1 = \ldots = \nu_m = 0$. Тогда из линейной независимости $e_1, \ldots, e_k$ следует $\lambda_1 = \ldots = \lambda_k = 0$. Итак, $\dim(V_1 + V_2) = k + l + m$, отсюда вытекает требуемое.
\end{proof}

\textbf{Алгоритм вычисления базисов} в $V_1 + V_2$ и $V_1 \cap V_2$ ($V_1$ и $V_2$ --- подпространства конечномерного пространства $V$). При доказательстве формулы Грассмана мы попутно показали, что объединение базисов $V_1$ и $V_2$ --- полная подсистема в $U_1 + U_2$, значит, из неё можно выделить базис (выше в вопросе 1, как). Отметим, что в случае, когда сумма $V_1 + V_2$ прямая, пересечение этих базисов пустое, и ничего дополнительно делать не нужно.

Чтобы найти базис пересечения, зададим оба подпространства в виде системы линейных уравнений. Система, состоящая из всех уравнений обоих систем, будет задавать пересечение этих подпространств. Для нахождения базиса ищем ФСР.

