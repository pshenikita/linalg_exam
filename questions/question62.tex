\section{Тензоры на евклидовом пространстве}

Пусть $V$ --- евклидово пространство. Тогда соответствие $v \mapsto (v, \bs{\cdot})$ задаёт канонический изоморфизм $V \to V^\ast$, т.\,е. позволяет отождествить тензоры типа $(0, 1)$ с тензорами типа $(1, 0)$. В координатах это выглядит следующим образом. Пусть $G = (g_{ij})$ --- матрица Грама скалярного произведения. Т.\,к. скалярное произведение --- это билинейная функция, $G$ является тензором типа $(2, 0)$. Тогда при изоморфизме $V \to V^\ast$ вектор $T$ с координатами $T^i$ переходит в ковектор с координатами $T_j = g_{ij}T^i$. Таким образом, мы <<опустили индекс>> у тензора $T$ при помощи фиксированного тензора типа $(2, 0)$. Эта операция обобщается следующим образом.

\begin{definition}
    \textit{Опускание индекса} --- это линейное отображение $\Ten_p^q(V) \to \Ten_{p + 1}^{q - 1}(V)$ тензоров в евклидовом пространстве $V$, которое тензору $T = \{T_{i_1, \ldots, i_p}^{j_1, \ldots, j_q}\}$ типа $(p, q)$ ставит в соответствие тензор $\{g_{ij}T_{i_1, \ldots, i_p}^{j, j_2, \ldots, j_q}\}$ типа $(p + 1, q - 1)$.
\end{definition}

\begin{example}
    Рассмотрим изоморфизм $\psi: \End(V) \to \mathrm{B}(V)$ между пространством операторов и пространством билинейных функций, сопоставляющий оператору $\A$ билинейную функцию $\B_\A = (\A\bs{\cdot}, \bs{\cdot})$. В координатах это выглядит так: оператору с матрицей $A = (a^i_k)$ сопоставляется билинейная функция с матрицей $B_A = (g_{ij}a^i_k)$. Таким образом, $B_A$ --- это тензор типа $(2, 0)$, получаемый в результате опускания индекса у тензора $A$ типа $(1, 1)$.
\end{example}

Для того, чтобы определить операцию, обратную к опусканию индекса, рассмотрим набора $\{g^{kl}\}$, состоящий из элементов обратной матрицы к матрице Грама $G = (g_{ij})$, т.\,е. $g^{kl}g_{lj} = \delta^k_j$.

\begin{proposal}
    $\{g^{kl}\}$ является тензором типа $(0, 2)$.
\end{proposal}

\begin{proof}
    Необходимо поверить тензорный закон. В новом базисе мы имеем $g^{k^\prime i^\prime}g_{i^\prime j^\prime} = \delta^{k^\prime}_{j^\prime}$. Т.\,к. $\{g_{ij}\}$ --- тензор типа $(2, 0)$, мы имеем $g_{i^\prime j^\prime} = c^i_{i^\prime}c^j_{j^\prime}g_{ij}$. Подставив это в предыдущую формулу, получим $g^{k^\prime i^\prime}c^i_{i^\prime}c^j_{j^\prime}g_{ij} = \delta^{k^\prime}_{j^\prime}$. Теперь умножим обе части этого равенства на компоненты обратной матрицы $c^{j^\prime}_k$ (и просуммируем по $j^\prime$): $g^{k^\prime i^\prime}c^i_{i^\prime}c^j_{j^\prime}c^{j^\prime}_kg_{ij} = \delta^{k^\prime}_{j^\prime}c^{j^\prime}_k$. Т.\,к. $c^j_{j^\prime}c^{j^\prime}_k = \delta^j_k$, отсюда получаем $g^{k^\prime i^\prime}c^i_{i^\prime}g_{ik} = c^{k^\prime}_k$. Далее умножим обе части на $g^{kl}$: $g^{k^\prime i^\prime}c^l_{i^\prime}c^{l^\prime}_l = c^{k^\prime}_kc^{l^\prime}_lg^{kl}$. Т.\,к. $c^l_{i^\prime}c^{l^\prime}_l = \delta^{l^\prime}_{i^\prime}$, окончательно получаем $g^{k^\prime l^\prime} = c^{k^\prime}_kc^{l^\prime}_lg^{kl}$. Это и есть тензорный закон преобразования для тензора типа $(0, 2)$.
\end{proof}

\begin{definition}
    \textit{Поднятие индекса} --- это линейное отображение $\Ten_p^q(V) \to \Ten_{p - 1}^{q + 1}(V)$ тензоров в евклидовом пространстве $V$, которое тензору $T = \{T_{i_1, \ldots, i_p}^{j_1, \ldots, j_q}\}$ типа $(p, q)$ ставит в соответствие тензор $\{g^{ij}T_{i, i_1, \ldots, i_p}^{j_2, \ldots, j_q}\}$ типа $(p - 1, q + 1)$.
\end{definition}

Операция опускания (или поднятия) индекса является композицией тензорного произведения с тензором $\{g_{ij}\}$ (или $\{g^{ij}\}$) и свёртки.

