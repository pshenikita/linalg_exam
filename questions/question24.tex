\section{Единственность жордановой нормальной формы}

\begin{theorem}
    Жорданова нормальная форма оператора единственна с точностью до перестановки клеток.
\end{theorem}

\begin{proof}
    Надо показать, что количество жордановых клеток фиксированного размера с одним и тем же $\lambda$ не зависит от выбора жорданова базиса. Выберем произвольный жорданов базис. Пусть $W_{\lambda_i}$ --- линейная оболочка части этого базиса, отвечающая всем клеткам с $\lambda_i$ на диагонали. Тогда ограничение оператора $(\A - \lambda_i \cdot \id)$ на $W_{\lambda_i}$ --- нильпотентный оператор, а значит, $W_{\lambda_i} \subseteq R_{\lambda_i}$. Кроме того, $W_{\lambda_1} \oplus \ldots \oplus W_{\lambda_k}$ по определению жорданова базиса и $V = R_{\lambda_1} \oplus \ldots \oplus R_{\lambda_k}$ (корневое разложение), отсюда $\dim W_{\lambda_i} = \dim R_{\lambda_i}$ и $W_{\lambda_i} = R_{\lambda_i}$. Итак, подпространства, отвечающие клеткам с собственным значением $\lambda_i$, не зависят от способа приведения к жордановой форме и равны $R_{\lambda_i}$.

    Таким образом, мы свели доказательство единственности жордановой формы к случаю, когда оператор $\A$ имеет одно собственное значение $\lambda$. Любой жорданов базис для такого оператора будет также нормальным базисом для нильпотентного оператора $\A - \lambda \cdot \id$. Для нильпотентных операторов мы уже доказали единственость нормального вида (т.\,е. жордановой формы).
\end{proof}

