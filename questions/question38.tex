\section{Самосопряжённый линейный оператор, его свойства и матрица. Существование ортонормированного базиса из собственных векторов}

\begin{definition}
    Оператор $\A: V \to V$ в евклидовом или эрмитовом пространстве называется \textit{самосопряжённым}, если $\A^\ast = \A$, т.\,е. $\forall u, v \in V$ выполнено соотношение
    \[
        (\A u, v) = (u, \A v).
    \]
\end{definition}

\begin{proposal}
    Матрица $A$ самосопряжённого оператора $\A$ в ортонормированном базисе евклидова (эрмитова) пространства симметрична (эрмитова), т.\,е. $A^t = A$ (соответственно, $\overline{A}^t = A$).

    Если матрица оператора $\A$ в некотором ортонормированном базисе симметрична (эрмитова), то оператор $\A$ самосопряжён.
\end{proposal}

\begin{proof}
    Первое утверждение вытекает из предложения 1 в вопросе 37: т.\,к. матрица оператора $\A^\ast$ есть $\overline{A}^t$ и $\A^\ast = \A$, мы получаем $\overline{A}^t = A$.

    Докажем второе утверждение. Пусть $e_1, \ldots, e_n$ --- ортонормированный базис, в котором матрица $A$ оператора $\A$ эрмитова, т.\,е. $\overline{A}^t = A$. Тогда из предложения 1 в вопросе 37 следует, что матрица $\overline{A}^t$ оператора $\A^\ast$ в том же базисе совпадает с $A$. Значит, $\A^\ast = \A$ и оператор $\A$ самосопряжён.
\end{proof}

В связи с с этим самосопряжённые операторы в евклидовом пространстве также называют \textit{симметрическими}, а в эрмитовом пространстве --- \textit{эрмитовыми}.

\begin{theorem}
    Самосопряжённый оператор диагонализируем в ортонормированном базисе.
\end{theorem}

Доказательство будет опираться на лемму, которая важна сама по себе.

\begin{lemma}
    Все корни характеристического многочлена самосопряжённого оператора $\A$ вещественны.
\end{lemma}

\begin{proof}
    Вначале докажем лемму для эрмитова пространства. В этом случае корни характеристического многочлена суть собственные значения оператора $\A$. Пусть $\lambda \in \C$ --- такой корень и $v \ne \bs{0}$ --- соответствующий собственный вектор, т.\,е. $\A v = \lambda v$. Тогда
    \[
        \overline{\lambda}(v, v) = (\lambda v, v) = (\A v, v) = (v, \A v) = (v, \lambda v) = \lambda(v, v).
    \]
    Т.\,к. $(v, v) \ne 0$, то $\overline{\lambda} = \lambda$, т.\,е. $\lambda \in \R$.

    Случай евклидова пространства сводится к эрмитовому случаю при помощи комплексификации. Пусть $A$ --- матрица самосопряжённого оператора $\A$ в ортонормированном базисе евклидова пространства $V$. Тогда матрица $A$ вещественна и симметрична. Та же матрица $A$ будет матрицей комплексифицированного оператора $\A_\C$ в соответствующем базисе эрмитова пространства $V_\C$. Этот базис также ортонормирован, а матрица $A$, будучи вещественной и симметричной, является эрмитовой. Следовательно, оператор $\A_\C$ самосопряжён, а корни его характеристического многочлена вещественны и совпадают с корнями характеристического многочлена $\A$.
\end{proof}

Теперь докажем теорему 1.

\begin{proof}
    Используем индукцию по размерности пространства $V$. При $\dim V = 1$ доказывать нечего. Предположим, что утверждение доказано для операторов в пространствах размерности $n - 1$, и докажем его для пространства $V$ размерности $n$.

    В силу предыдущей леммы у самосопряжённого оператора $\A$ имеется собственный вектор $v$, т.\,е. одномерное инвариантное подпространство $W = \langle v\rangle$. В силу важной леммы, ортогональное дополнение $W^\perp$ инвариантно относительно оператора $\A^\ast = \A$. Т.\,к. $\dim W^\perp = n - 1$, в пространстве $W^\perp$ имеется ортонормированный базис $e_1, \ldots, e_{n - 1}$ из собственный векторов оператора $\A\big|_{W^\perp}$. Тогда $e_1, \ldots, e_{n - 1}, \frac{v}{\abs{v}}$ --- ортонормированный базис из собственных векторов оператора $\A$.
\end{proof}

Диагональный вид матрицы самосопряжённого оператора $\A$ в ортонормированном базисе из собственных векторов называется \textit{каноническим видом} самосопряжённого оператора.

Практический метод нахождения ортонормированного базиса из собственный векторов основан на следующей лемме.

\begin{lemma}
    Собственные векторы, отвечающие различным собственным значениям самосопряжённого оператора $\A$, взаимно ортогональны.
\end{lemma}

\begin{proof}
    Пусть $\A u = \lambda u$ и $\A v = \mu v$, где $\lambda \ne \mu$ --- вещественные собственные значения. Тогда
    \[
        \lambda(u, v) = (\lambda u, v) = (\A u, v) = (u, \A v) = (u, \mu v) = \mu(u, v),
    \]
    откуда $(u, v) = 0$, т.\,к. $\lambda \ne \mu$.
\end{proof}

Для нахождения ортонормированного базиса из собственных векторов самосопряжённого оператора $\A$ находятся все его собственные подпространства, а затем в каждом из них выбирается ортонормированный базис. Объединение этих базисов и будет нужным базисом для $\A$. Про это хорошо \href{http://halgebra.math.msu.su/staff/klyachko/teaching/lin.al/SY.PDF}{написал} Антон Александрович Клячко.

В евклидовом пространстве верно и утверждение, обратное к предыдущей теореме.

\begin{proposal}
    Если оператор $\A$ в евклидовом пространстве диагонализируем в ортонормированном базисе, то $\A$ самосопряжён.
\end{proposal}

\begin{proof}
    Действительно, диагональная матрица симметрична, а оператор, имеющий симметричную матрицу в ортонормированном базисе евклидова пространства самосопряжён согласно предложению 1.
\end{proof}

В эрмитовом пространстве класс операторов, имеющих симметричную матрицу в ортонормированном базисе, шире, чем самосопряжённые (т.\,к. на диагонали могут стоять не вещественные числа).

