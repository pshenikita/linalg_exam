\section{Существование одномерного или двумерного инвариантного подпространства для любого линейного оператора над полем действительных чисел}

Б\'{о}льшая часть этого вопроса --- рассказ про овеществление и комплексификацию линейных пространств (взято у Панова и Винберга).

При работе с линейными операторами часто бывает удобно изменить поле скаляров. Здесь рассмотрим две такие операции: переход от $\R$ к $\C$ (комплексификация) и переход от $\C$ к $\R$ (овеществление).

\begin{definition}
    Пусть $V$ --- линейное пространство над полем $\C$. Рассмотрим множество $V_\R$, состоящее из тех же векторов, что и $V$. На $V_\R$ имеется операция сложения (та же, что и на $V$), а вместо операции умножения на все комплексные числа оставим лишь умножение на все вещественные числа. Тогда $V_\R$ --- линейное пространство над полем $\R$, которое называется \textit{овеществлением пространства} $V$.
\end{definition}

\begin{proposal}
    Пусть $e_1, \ldots, e_n$ --- базис пространства $V$. Тогда $e_1, \ldots, e_n, ie_1, \ldots, ie_n$ --- базис пространства $V_\R$.
\end{proposal}

\begin{proof}
    Проверим, что векторы $e_1, \ldots, e_n, ie_1, \ldots, ie_n$ линейно независимы в $V_\R$. Пусть $\lambda_1e_1 + \ldots + \lambda_ne_n + \mu_1ie_1 + \ldots + \mu_nie_n = \bs{0}$ в пространстве $V_\R$, где $\lambda_k, \mu_k \in \R$, $k = 1, \ldots, n$. Тогда в пространстве $V$ мы имеем $(\lambda_1 + i\mu_1)e_1 + \ldots + (\lambda_n + i\mu_n)e_n = \bs{0}$. Т.\,к. векторы $e_1, \ldots, e_n$ линейно независимы в $V$, то $\lambda_k + i\mu_k = 0 \Rightarrow \lambda_k = \mu_k = 0$, $k = 1, \ldots, n$. Следовательно, векторы $e_1, \ldots, e_n, ie_1, \ldots, ie_n$ линейно независимы в $V_\R$. Теперь проверим, что эти векторы порождают всё пространство $V_\R$. Возьмём $v \in V_\R$ и рассмотрим его как вектор из $V$. Т.\,к. $e_1, \ldots, e_n$ --- базис в $V$, то $v = \alpha_1e_1 + \ldots + \alpha_ne_n$, где $(\alpha_k = \lambda_k + i\mu_k) \in \C$ ($k = 1, \ldots, n$), где $\lambda_k, \mu_k \in \R$. Тогда $v = \lambda_1e_1 + \ldots + \lambda_ne_n + \mu_1ie_1 + \ldots + \mu_nie_n$.
\end{proof}

\begin{corollary}
    $\dim V_\R = 2\dim V$.
\end{corollary}

\begin{definition}
    Пусть $V$ --- комплексное пространство и $\A: V \to V$ --- оператор. Тогда тот же оператор, рассматриваемый в пространстве $V_\R$ называется \textit{овеществлением оператора} $\A$ и обозначается $\A_\R$.
\end{definition}

\begin{proposal}
    Запишем матрицу оператора $\A$ в базисе $e_1, \ldots, e_n$ пространства $V$ в виде $A + iB$, где $A$ и $B$ --- вещественные матрицы. Тогда
    \begin{enumerate}[nolistsep]
        \item матрица оператора $\A_\R$ в базисе $e_1, \ldots, e_n, ie_1, \ldots, ie_n$ есть
            $
            \left(
            \begin{array}{c | c}
                A & -B\\
                \hline
                B & A
            \end{array}
            \right)
            $;
        \item $\det\A_\R = \abs{\det\A}^2$.
    \end{enumerate}
\end{proposal}

\begin{proof}
    \begin{enumerate}
        \item Пусть $A = (a^l_k)$ и $B = (b^l_k)$. Тогда $\A_\R(e_k) = \A(e_k) = (a^l_k + ib^l_k)e_l = a^l_ke_l + ib^l_ke_l$, $\A_\R(ie_k) = \A(ie_k) = i\A(e_k) = i(a^l_k + ib^l_k)e_l = -b^l_ke_l + a^l_kie_l$.
        \item Заметим, что
            \begin{multline*}
                \left(
                \begin{array}{c | c}
                    A & -B\\
                    \hline
                    B & A
                \end{array}
                \right) \leadsto
                \left(
                \begin{array}{c | c}
                    A - iB & -B - iA\\
                    \hline
                    B & A
                \end{array}
                \right) \leadsto\\
                \left(
                \begin{array}{c | c}
                    A - iB & -B - iA + i(A - iB)\\
                    \hline
                    B & A + iB
                \end{array}
                \right) \leadsto
                \left(
                \begin{array}{c | c}
                    A - iB & 0\\
                    \hline
                    B & A + iB
                \end{array}
                \right).
            \end{multline*}

            Отсюда получаем $\det\A_\R = \det(A - iB)\det(A + iB) = \overline{\det\A}\det\A = \abs{\det\A}^2$.
    \end{enumerate}
\end{proof}

\begin{definition}
    Пусть $V$ --- вещественное пространство. \textit{Комплексной структурой} на $V$ называется такой оператор $\mathcal{J}: V \to V$, что $\mathcal{J}^2 = -\id$.
\end{definition}

Пусть $V$ --- вещественное пространство с комплексной структурой $\mathcal{J}$. Введём на $V$ операцию умножения на комплексные числа по правилу $(\lambda + i\mu)v = \lambda v + \mu\mathcal{J}v$. Тогда $V$ превращается в комплексное пространство $\widetilde{V}$, для которого $\widetilde{V}_\R = V$, а овеществление оператора умножения на $i$ есть $\mathcal{J}$. Проверка выполняется непосредственно.

\begin{proposal}
    Пусть $\mathcal{J}$ -- комплексная структура на $V$. Тогда:
    \begin{enumerate}[nolistsep]
        \item размерность вещественного пространства $V$ чётна;
        \item в подходящем базисе матрица оператора $\mathcal{J}$ имеет вид
            $
            \left(
            \begin{array}{c | c}
                0 & -E\\
                \hline
                E & 0
            \end{array}
            \right)
            $.
    \end{enumerate}
\end{proposal}

\begin{proof}
    Т.\,к. любой базис $V$ порождает $\widetilde{V}$, то это пространство конечномерно. А т.\,к. $V = \widetilde{V}_\R$, то $\dim V = 2\dim\widetilde{V}$ --- чётно. Далее, если $e_1, \ldots, e_n$ --- базис комплексного пространства $\widetilde{V}$, то $e_1, \ldots, e_n, ie_1, \ldots, ie_n$ --- базис пространства $\widetilde{V}_\R = V$. В этом базисе оператор $\mathcal{J}$ (овеществление оператора умножения на $i$) имеет указанный вид. Это прямое следствие записей сверху.
\end{proof}

\begin{definition}
    Пусть $V$ --- линейное пространство над полем $\R$. Рассмотрим пространство $V \oplus V$, состоящее из пар $(u, v)$, где $u, v \in V$, и введём на нём комплексную структуру следующим образом: $\mathcal{J}(u, v) \vcentcolon = (-v, u)$. Получаемое пространство $\widetilde{V \oplus V}$ над полем $\C$ называется \textit{комплексификацией} пространства $V$ и обозначается $V_\C$.
\end{definition}

\begin{proposal}
    Пусть $e_1, \ldots, e_n$ --- базис пространства $V$. Тогда векторы $(e_1, \bs{0}), \ldots, (e_n, \bs{0})$ образуют базис пространства $V_\C$.
\end{proposal}

\begin{proof}
    Проверим линейную независимость: пусть $\alpha_1(e_1, \bs{0}) + \ldots + \alpha_n(e_n, \bs{0}) = (\bs{0}, \bs{0})$ для некоторых $\alpha_k = \lambda_k + i\mu_k \in \C$, $k = 1, \ldots, n$. Выкладка:
    \[
        (\lambda_k + i\mu_k)(e_k, \bs{0}) = \lambda_k(e_k, \bs{0}) + \mu_k\mathcal{J}(e_k, \bs{0}) = (\lambda_ke_k, \mu_ke_k).
    \]
    Подставляя это в линейную комбинацию выше, получаем \[(\lambda_1e_1 + \ldots + \lambda_ne_n, \mu_1e_1 + \ldots + \mu_ne_n) = (\bs{0}, \bs{0}).\]
    Из линейной независимости векторов $e_1, \ldots, e_n$ в $V$, получаем $\lambda_k = \mu_k = 0$ для всех $k$. А значит, $\alpha_1 = \ldots = \alpha_n = 0$. Итак, $(e_1, \bs{0}), \ldots, (e_n, \bs{0})$ линейно независимы. То, что они порождают всё пространство, проверяется аналогично.
\end{proof}

\begin{corollary}
    $\dim V_\C = \dim V$.
\end{corollary}

\begin{definition}
    Пусть $V$ --- пространство над $\R$ и $\A: V \to V$ --- оператор. Оператор $\A_\C: V_\C \to V_\C$, заданный формулой $\A_\C(u, v) \vcentcolon = (\A u, \A v)$, называется \textit{комплексификацией оператора} $\A$.
\end{definition}

\begin{proposal}
    Пусть $A$ --- матрица оператора $\A$ в базисе $e_1, \ldots, e_n$. Тогда оператор $\A_\C$ в базисе $(e_1, \bs{0}), \ldots, (e_n, \bs{0})$ задаётся той же матрицей $A$.
\end{proposal}

\begin{proof}
    $\A_\C(e_k, \bs{0}) = (\A e_k, \bs{0}) = (a^l_ke_l, \bs{0}) = a^l_k(e_l, \bs{0})$.
\end{proof}

Отметим, что при работе с комплексифицированным пространством $V_\C$ удобно записывать векторы $(u, v) \in V_\C$ в виде $u + iv$. Тогда действие комплексифицированного оператора $\A_\C$ записывается как $\A_\C(u + iv) = \A u + i\A v$.

\begin{proposal}
    Пространство $(V_\C)_\R$ канонически изоморфно $V \oplus V$.
\end{proposal}

\begin{proof}
    Действительно, $V_\C = \widetilde{V \oplus V}$, а $(\widetilde{V \oplus V})_\R = V \oplus V$ (мы просто сначала добавили, а потом убрали домножение на комплексные скаляры).
\end{proof}

Рассказ про овеществление и комплексификацию завершён.

Оператор $\A: V \to V$ в нетривиальном пространстве над полем $\C$ имеет инвариантное подпространство размерности $1$. Действительно, т.\,к. поле $\C$ алгебраически замкнуто, характеристический многочлен $\chi_\A(t)$ имеет корень $\lambda$ и собственный вектор $v \in V_\lambda$. Подпространство $\langle v\rangle$ собственное размерности $1$.

\begin{theorem}
    Оператор $\A: V \to V$ в нетривиальном пространстве над полем $\R$ имеет инвариантное подпространство размерности $1$ или $2$.
\end{theorem}

\begin{proof}
    Если характеристический многочлен $\chi_\A(t)$ имеет вещественный корень, то (аналогично) мы получаем одномерное инвариантное подпространство. Предположим, что $\chi_\A(t)$ не имеет вещественных корней. Пусть $\lambda + i\mu$ --- комплексный корень, $\mu \ne 0$. Тогда $\lambda + i\mu$ --- собственное значение комплексифицированного оператора $\A_\C$ (напомним, что в подходящих базисах матрицы операторов $\A$ и $\A_\C$ совпадают). Возьмём соответствующий собственный вектор $u + iv \in V_\C$. Тогда $\A u + i\A v = \A_\C(u + iv) = (\lambda + i\mu)(u + iv) = (\lambda u - \mu v) + i(\mu u + \lambda v)$. Следовательно, $\A u = \lambda u - \mu v$, а $\A v = \mu u + \lambda v$, и линейная оболочка $\langle u, v\rangle$ является инвариантным подпространством для $\A$.
\end{proof}

\begin{corollary}
    Над полем $\R$ любой линейный оператор приводим к блочно-диагональному виду, причём блоки имеют порядок не выше двух.
\end{corollary}

