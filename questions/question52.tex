\section{Аффинное евклидово пространство. Теорема об изоморфизме. Расстояние между двумя
плоскостями}

Здесь мы будем рассматривать аффинные пространства $(\A, V)$, ассоциированные с евклидовым пространством $V$, которые будем называть \textit{аффинными евклидовыми пространствами}.

Среди всех аффинных систем координат в аффинном евклидовом пространстве выделяются системы координат, определяемые ортонормированными реперами. Они называются \textit{прямоугольными системами координат}.

\begin{definition}
    \textit{Расстояние} $\rho$ между точками $p, q \in \A$ аффинного евклидова пространства $(\A, V)$ определяется по формуле
    \[
        \rho(p, q) \vcentcolon = \abs{\overline{pq}}.
    \]
\end{definition}

\begin{definition}
    \textit{Аффинные евклидовы пространства} $(\A, V)$ и $(\A^\prime, V^\prime)$ называются \textit{изоморфными}, если между соответствующими аффинными пространствами существует \textit{изометрический изоморфизм} $\Phi$, т.\,е. такой, который сохраняет длины векторов:
    \[
        \abs{\overline{\Phi(p)\Phi(q)}} = \abs{\overline{pq}}\quad\forall p, q \in \A.
    \]
\end{definition}

Здесь верно утверждение, аналогичное уже доказанному про аффинные пространства.

\begin{theorem}[Об изоморфизме]
    Аффинные евклидовы пространства изоморфны тогда и только тогда, когда они имеют одинаковую размерность.
\end{theorem}

\begin{proof}
    Чтобы доказать эту теорему, нужно перестать думать. Если размерности одинаковые, то берём любое ортогональное отображение $V \to V^\prime$ и принимаем его за дифференциал нашего отображения, можно к нему любой вектор ещё добавить, если вдруг хочется. Вот и искомый изометрический изоморфизм. А если они изоморфны, то конечно же размерности должны быть равны, иначе между ними нельзя установить биекцию.
\end{proof}

Нахождение расстояния от точки $p \in \A$ до плоскости $P = p_0 + W$ с помощью векторизации сводится к нахождению расстояния от вектора $x \vcentcolon = \overline{p_0p} \in V$ до подпространства $W$. А именно, пусть $x = \pr_Wx + \ort_Wx$. Приняв точку $p_0$ за начало отсчёта, мы получаем
\[
    \rho(p, P) = \rho(x, W) = \abs{\ort_Wx}.
\]

Точка $q \vcentcolon = p_0 + \pr_Wx$ является <<основанием перпендикуляра, опущенного из точки $p$ на плоскость $P$>>.

\begin{theorem}
    Расстояние между плоскостями $P_1 = p_1 + W_1$ и $P_2 = p_2 + W_2$ аффинного евклидова пространства находится по формуле
    \[
        \rho(P_1, P_2) = \abs{\ort_{W_1 + W_2}\overline{p_1p_2}}.
    \]
\end{theorem}

\begin{proof}
    Разложим $V$ в прямую сумму $V = (W_1 + W_2) \oplus (U + W)^\perp$ и представим вектор $\overline{p_1p_2}$ в виде
    \[
        \overline{p_1p_2} = \pr_{W_1 + W_2}\overline{p_1p_2} + \ort_{W_1 + W_2}\overline{p_1p_2}.
    \]
    Выберем любые две точки $q_1 = (p_1 + w_1) \in P_1$, $q_2 = (p_2 + w_2) \in P_2$. Тогда по теореме Пифагора получаем, что
    \begin{multline*}
        \rho(q_1, q_2) = \abs{\overline{p_1p_2} + w_2 - w_1}^2 = \abs{(\pr_{W_1 + W_2}\overline{p_1p_2} + w_2 - w_1) + \ort_{W_1 + W_2}\overline{p_1p_2}}^2 =\\ = \abs{\pr_{W_1 + W_2}\overline{p_1p_2} + w_2 - w_1}^2 + \abs{\ort_{W_1 + W_2}\overline{p_1p_2}}^2 \geqslant \abs{\ort_{W_1 + W_2}\overline{p_1p_2}}^2.
    \end{multline*}

    Таким образом, $\rho(q_1, q_2) \geqslant \abs{\ort_{W_1 + W_2}\overline{p_1p_2}}$, причём равенство достигается для таких точек $q_1$ и $q_2$, что
    \[
        \pr_{W_1 + W_2}\overline{p_1q_1} = w_1 - w_2.
    \]

    Т.\,к. $\pr_{W_1 + W_2}\overline{p_1p_2} \in W_1 + W_2$, такие векторы $w_1 \in W_1$, $w_2 \in W_2$ существуют.
\end{proof}

