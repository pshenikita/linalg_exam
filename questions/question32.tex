\section{Ортогональность векторов. Линейная независимость системы ненулевых ортогональных векторов. Ортогональный и ортонормированный базис. Алгоритм ортогонализации Грама "---Шмидта}

\begin{proposal}
    Пусть $v_1, \ldots, v_k$ --- набор попарно ортогональных ненулевых векторов евклидова или эрмитова пространства. Тогда эти векторы линейно независимы.
\end{proposal}

\begin{proof}
    Пусть некоторая линейная комбинация данных векторов равна нулю:
    \[
        \sum_{i = 1}^k\lambda_iv_i = \bs{0}.
    \]

    Умножим обе части этого равенства скалярно на $v_j$ и воспользуемся линейностью скалярного произведения по второму аргументу:
    \[
        0 = \br{v_j, \sum_{i = 1}^k\lambda_iv_i} = \sum_{i = 1}^k\lambda_i(v_j, v_i) = \lambda_j(v_j, v_j),
    \]
    т.\,к. по условию остальные слагаемые в этой сумме равны нулю. Посколько $v_j \ne \bs{0}$, из положительной определённости скалярного произведения следует, что $(v_j, v_j) \ne \bs{0}$, а значит, $\lambda_j = 0$. Это выполнено $\forall j = 1, \ldots, k$, следовательно, линейная комбинация $\sum\limits_{i = 1}^k\lambda_iv_i$ тривиальна.
\end{proof}

\begin{definition}
    Базис $e_1, \ldots, e_n$ евклидова или эрмитова пространства называется \textit{ортогональным}, если его векторы попарно ортогональны. Если при этом длина каждого вектора равна $1$, то базис называется \textit{ортогормированным}.
\end{definition}

\begin{theorem}
    Пусть $a_1, \ldots, a_k$ --- набор линейно независимых векторов пространства $V$. Тогда существует такой набор попарно ортогональных векторов $b_1, \ldots, b_k$, что для каждого $i = 1, \ldots, k$ линейная оболочка $\langle b_1, \ldots, b_i\rangle$ совпадает с $\langle a_1, \ldots, a_i\rangle$.
\end{theorem}

\begin{proof}
    При $k = 1$ утверждение очевидно: можно взять $b_1 \vcentcolon = a_1$. Предположим, что утверждение верно для наборов из $i$ векторов, и докажем его для наборов из $i + 1$ вектора. Пусть $b_1, \ldots, b_i$ --- ортогональный набор, построенный по набору $a_1, \ldots, a_i$. Мы хотим, чтобы для нового вектора $b_{i + 1}$ линейная оболочка $\langle b_1, \ldots, b_i, b_{i + 1}\rangle$ совпадала с $\langle a_1, \ldots, a_i, a_{i + 1}\rangle = \langle b_1, \ldots, b_i, a_{i + 1}\rangle$, и поэтому будем искать $b_{i + 1}$ в виде
    \[
        b_{i + 1} = a_{i + 1} + \lambda_1b_1 + \ldots + \lambda_ib_i.
    \]

    Коэффициенты $\lambda_1, \ldots, \lambda_i$ будем подбирать так, чтобы вектор $b_{i + 1}$ был ортогонален всем предыдущим векторам $b_1, \ldots, b_i$. Умножив скалярно предыдущее равенство слева на $b_j$ и использовав то, что $(b_i, b_l) = 0$ при $j \ne l$, получаем
    \[
        0 = (b_j, b_{i + 1}) = (b_j, a_{i + 1}) + \lambda_j(b_j, b_j),
    \]
    откуда $\ds\lambda_j = -\frac{(b_j, a_{j + 1})}{(b_j, b_j)}$ $\forall j = 1, \ldots, i$. Окончательно для вектора $b_{i + 1}$ получаем
    \begin{multline*}
        b_{i + 1} = a_{i + 1} - \frac{(b_1, a_{i + 1})}{(b_1, b_1)}b_1 - \frac{(b_2, a_{i + 1})}{(b_2, b_2)}b_2 - \ldots - \frac{(b_i, a_{i + 1})}{(b_i, b_i)}b_i =\\ = a_{i + 1} - \pr_{b_1}a_{i + 1} - \pr_{b_2}{a_{i + 1}} - \ldots - \pr_{b_i}{a_{i + 1}}.
    \end{multline*}
\end{proof}

\begin{definition}
    Индуктивная процедура перехода от набора $a_1, \ldots, a_k$ к ортогональному набору $b_1, \ldots, b_k$ называется \textit{процессом ортогонализации Грама "---Шмидта}.
\end{definition}

Условие $\langle b_1, \ldots, b_i\rangle = \langle a_1, \ldots, a_i\rangle$ при $i = 1, \ldots, k$ означает, что матрица перехода от $a_1, \ldots, a_k$ к $b_1, \ldots, b_k$ является верхнетреугольной.

\begin{corollary}
    В евклидовом или эрмитовом пространстве $V$ существуют ортонормированные базисы.
\end{corollary}

\begin{proof}
    Возьмём произвольные базис и ортогонализируем его методом Грама "---Шмидта, получив при этом новый базис $b_1, \ldots, b_n$. Тогда базис, состоящий из векторов $\frac{b_1}{\abs{b_1}}, \ldots, \frac{b_n}{\abs{b_n}}$ будет ортонормированным.
\end{proof}

\begin{proposal}
    Пусть векторы $u$ и $v$ имеют координаты $u^1, \ldots, u^n$ и $v^1, \ldots, v^n$ в некотором ортонормированном базисе евклидова или эрмитова пространства $V$. Тогда их скалярное произведение вычисляется по формуле
    \[
        (u, v) = \overline{u^1}v^1 + \overline{u^2}v^2 + \ldots + \overline{u^n}v^n.
    \]
\end{proposal}

\begin{proof}
    Пусть $e_1, \ldots, e_n$ --- ортонормированный базис. Тогда
    \[
        (u, v) = (u^ie_i, v^je_j) = \overline{u^i}v^j(e_i, e_j) = \overline{u^i}v^j\delta_{ij} = \overline{u^1}v^1 + \overline{u^2}v^2 + \ldots + \overline{u^n}v^n.
    \]
\end{proof}

