\section{Изоморфизм векторных пространств одинаковой размерности}

\begin{definition}
    Пусть $V$ и $W$ --- линейные пространства над полем $\F$. Отображение $\A: V \to W$ называется \textit{линейным}, если $\forall u, v \in V$, $\forall \lambda \in \F$ выполнено $\A(u + v) = \A u + \A v$, $\A(\lambda v) = \lambda\A v$.
\end{definition}

\begin{definition}
    Биективное линейное отображение $\A: V \to W$ называется \textit{изоморфизмом}, а пространства $V$ и $W$, между которыми есть изоморфизм, называются \textit{изоморфными}.
\end{definition}

\begin{theorem}
    Два конечномерных пространства $V$ и $W$ над полем $\F$ изоморфны тогда и только тогда, когда они имеют одинаковые размерности.
\end{theorem}

\begin{proof}
    Из определения изоморфизма вытекает, что свойство системы векторов быть линейно независимой и порождать всё пространство сохраняются при изоморфизмах, т.\,е. при изоморфизме базис переходит в базис\footnote{Это частный случай того, что <<образ базиса является базисом образа>> при $\Im \A = W$.}. Следовательно, если $\A: V \to W$ --- изоморфизм, то $\dim V = \dim W$. Пусть теперь $\dim V = \dim W = n$. Выберем базисы $e_1, \ldots, e_n$ и $f_1, \ldots, f_n$ соответственно. Тогда формула $\A(x^ie_i) = x^if_i$ определяет линейное отображение $\A: V \to W$. Оно является биекцией, т.\,к. формула $\A^{-1}(x^if_i) = x^ie_i$ определяет обратное отображение.
\end{proof}

