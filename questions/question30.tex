\section{Евклидово пространство. Неравенство Коши "---Буняковского. Угол между векторами и длина вектора в евклидовом пространстве. Неравенство треугольника}

Здесь я сразу напишу и про эрмитовы пространства, чтобы лишний раз не дублировать записи.

\begin{definition}
    Линейное пространство над полем $\R$ называется \textit{евклидовым}, если на парах его векторов определена функция $f: V \times V \to \R$ (обозначаемая $(a, b) \vcentcolon = f(a, b)$ и называемая \textit{скалярным произведением}), удовлетворяющая следующим аксиомам:
    \begin{enumerate}[nolistsep]
        \item \textit{билинейность}:
            \[
                (\lambda_1u_1 + \lambda_2u_2, v) = \lambda_1(u_1, v) + \lambda_2(u_2, v),\quad(u, \mu_1v_1 + \mu_2v_2) = \mu_1(u, v_1) + \mu_2(u, v_2)
            \]
            $\forall \lambda_1, \lambda_2, \mu_1, \mu_2 \in \R$, $\forall u, u_1, u_2, v, v_1, v_2 \in V$;
        \item \textit{симметричность}: $(u, v) = (v, u)$ $\forall u, v \in V$;
        \item \textit{положительная определённость}: $(v, v) \geqslant 0$ $\forall v \in V$, причём $(v, v) = 0$ только при $v = \bs{0}$.
    \end{enumerate}
\end{definition}

Заметим, что из симметричности и линейности по одному аргументу сразу следует линейность и по другому аргументу.

\begin{remark}
    В комплексном пространстве функции с такими свойствами не бывает. Действительно, если $(v, v)$ --- положительное вещественное число, то $(iv, iv) = i^2(v, v) = -(v, v)$ --- отрицательное. Таким образом, функция $V \times V \to \R$ ($V$ --- линейное пространство над $\C$) не может быть билинейной и положительно определённой.
\end{remark}

\begin{definition}
    Линейное пространство над полем $\C$ называется \textit{эрмитовым} (или \textit{унитарным}), если на парах его векторов определена функция $f: V \times V \to \C$ (обозначаемая $(a, b) \vcentcolon = f(a, b)$ и называемая \textit{скалярным произведением}), удовлетворяющая следующим свойствам:
    \begin{enumerate}[nolistsep]
        \item \textit{полуторалинейность}:
            \[
                (\lambda_1u_1 + \lambda_2u_2, v) = \overline{\lambda_1}(u_1, v) + \overline{\lambda_2}(u_2, v),\quad(u, \mu_1v_1 + \mu_2v_2) = \mu_1(u, v_1) + \mu_2(u, v_2)
            \]
            $\forall \lambda_1, \lambda_2, \mu_1, \mu_2 \in \C$, $\forall u, u_1, u_2, v, v_1, v_2 \in V$;
        \item \textit{эрмитовость}: $(u, v) = \overline{(v, u)}$ $\forall u, v \in V$; в частности $(v, v) \in \R$ $\forall v \in V$;
        \item \textit{положительная определённость}: $(v, v) \geqslant 0$ $\forall v \in V$, причём $(v, v) = 0$ только при $v = \bs{0}$.
    \end{enumerate}
\end{definition}

Свойство полуторалинейности выражает линейность скалярного произведения по втором аргументу и \textit{антилинейность} по первому. Ввиду наличия свойства эрмитовости, полуторалинейность очевидно вытекает из линейности по второму аргументу.

\begin{example}
    \begin{enumerate}[nolistsep]
        \item Скалярное произведение векторов $u = (u^1, \ldots, u^n)$ и $v = (v^1, \ldots, v^n)$ в пространстве $\R^n$ можно задать формулой
            \[
                (u, v) \vcentcolon = u^1v^1 + u^2v^2 + \ldots + u^nv^n,
            \]
            а скалярное произведение в $\C^n$ --- формулой
            \[
                (u, v) \vcentcolon = \overline{u^1}v^1 + \overline{u^2}v^2 + \ldots + \overline{u^n}v^n.
            \]
            Это называется \textit{стандартным} скалярным произведением.
        \item Скалярное произведение в пространстве $\underset{n \times n}{\Mat}(\C)$ квадратных комплексных матриц размера $n$ задаётся с помощью формулы
            \[
                (A, B) \vcentcolon = \tr(\overline{A}^tB).
            \]
        \item Рассмотрим пространство $C[a; b]$ вещественнозначных функций, непрерывных на отрезке $[a; b]$. Зададим скалярное произведение функций $f$ и $g$ по формуле
            \[
                (f, g) \vcentcolon = \int\limits_a^bf(x)g(x)dx.
            \]
    \end{enumerate}
\end{example}

\begin{definition}
    Пусть $V$ --- евклидово или эрмитово пространство. Для $v \in V$ величина $\sqrt{(v, v)}$ называется \textit{длиной} вектора $v$ и обозначается $\abs{v}$.
    Векторы $u, v \in V$ такие, что $(u, v) = 0$, называются \textit{ортогональными}, обозначается $u \perp v$.
\end{definition}

\begin{proposal}
    Пусть $u$ --- ненулевой вектор евклидова или эрмитова пространства $V$. Тогда для любого вектора $v \in V$ существует единственное разложение $v = v_1 + v_2$, где вектор $v_1$ коллинеарен вектору $u$, а вектор $v_2$ ортогонален $u$.
\end{proposal}

\begin{proof}
    Сначала докажем единственность. Пусть $v = v_1 + v_2$ --- такое разложение. Тогда имеем $v_1 = \lambda u$, $v_2 = v - \lambda u$. Условие $u \perp v_2$ влечёт
    \[
        \bs{0} = (u, v_2) = (u, v - \lambda u) = (u, v) - \lambda(u, u).
    \]

    Отсюда $\lambda = (u, v) / (u, u)$ и
    \[
        v_1 = \frac{(u, v)}{(u, u)}u.\eqno(\ast)
    \]

    Тем самым векторы $v_1$ и $v_2 = v - v_1$ определены однозначно. С другой стороны, определив $v_1$ по этой формуле, мы получим $v_2 = (u - v_1) \perp u$.
\end{proof}

\begin{definition}
    Вектор $(\ast)$ называется \textit{ортогональной проекцией} вектора $v$ на направление вектора $u$ и обозначается $\pr_uv$, а вектор $v - \pr_uv$ называется \textit{ортогональной составляющей} вектора $v$ относительно $u$ и обозначается $\ort_uv$.
\end{definition}

\begin{theorem}[Неравенство Коши "---Буняковского]
    Для любых двух векторов $u$, $v$ евклидова или эрмитова пространства имеет место неравенство
    \[
        \abs{(u, v)} \leqslant \abs{u} \cdot \abs{v},
    \]
    причём равенство имеет место в том и только в том случае, когда $u$ и $v$ коллинеарны.
\end{theorem}

\begin{proof}
    Если $u = 0$, утверждение очевидно. Пусть $u \ne \bs{0}$. Запишем $v = v_1 + v_2$, где $v_1 = \pr_uv$ и $v_2 = \ort_uv$. Тогда $(v_1, v_2) = 0$, и мы имеем
    \[
        \abs{v}^2 = (v, v) = (v_1 + v_2, v_1 + v_2) = (v_1, v_1) + (v_1, v_2) + (v_2, v_1) + (v_2, v_2) = \abs{v_1}^2 + \abs{v_2}^2.
    \]

    Отсюда $\abs{v_1} \leqslant \abs{v}$, причём равенство достигается только при $v_2 = \bs{0}$, т.\,е. когда вектор $v$ коллинеарен вектору $u$. Осталось заменить, что $\abs{v_1} = \abs{\pr_uv} = \frac{\abs{(u, v)}}{\abs{u}}$, так что неравенство $\abs{v_1} \leqslant \abs{v}$ эквивалентно требуемому.
\end{proof}

\begin{definition}
    \textit{Углом} между двумя ненулевыми векторами $u$, $v$ евклидова пространства называется величина
    \[
        \angle(u, v) \vcentcolon = \arccos\frac{(u, v)}{\abs{u}\abs{v}} \in [0; \pi];
    \]
\end{definition}

Неравенство Коши "---Буняковского гарантирует, что угол между ненулевыми векторами всегда определён.

\begin{corollary}[Неравенство треугольника]
    Для любых двух векторов $u$, $v$ евклидова или эрмитова пространства выполнено неравенство
    \[
        \abs{u + v} \leqslant \abs{u} + \abs{v}.
    \]
\end{corollary}

\begin{proof}
    В обеих частях неравенства стоят неотрицательные величины, поэтому при возведении в квадрат получается равносильное неравенство
    \[
        (u + v, u + v) \leqslant (u, u) + (v, v) + 2\abs{u}\abs{v}.
    \]

    После раскрытия скобок в левой части и сокращения подобных членов мы получаем неравенство:
    \[
        (u, v) + (v, u) \leqslant 2\abs{u}\abs{v},
    \]
    которое следует из неравенства Коши "---Буняковского.
\end{proof}

