\section{Собственные векторы и собственные значения линейного оператора. Линейная независимость собственных векторов линейного оператора, отвечающих попарно различным собственным значениям}

\begin{definition}
    Ненулевой вектор $v \in V$ называется \textit{собственным} для оператора $\A$, если $\A v = \lambda v$ для некоторого $\lambda \in \F$. Число $\lambda \in \F$ называется \textit{собственным значением}, если существует собственный вектор $v$, для которого $\A v = \lambda v$.
\end{definition}

\begin{proposal}
    Собственные векторы, отвечающие попарно различным собственным значениям $\lambda_1, \ldots, \lambda_k$ оператора $\A$, линейно независимы.
\end{proposal}

\begin{proof}
    Докажем утверждение теоремы индукцией по $k$. При $k = 1$ доказывать нечего. Пусть $k > 1$ и
    \[
        \mu_1v_1 + \ldots + \mu_{k - 1}v_{k - 1} + \mu_kv_k = \bs{0}\quad\br{v_i \in V_{\lambda_i}, (\mu_1, \ldots, \mu_k) \ne (0, \ldots, 0)}
    \]
    Применяя оператор $\A$, получаем
    \[
        \lambda_1\mu_1v_1 + \ldots + \lambda_{k - 1}\mu_{k - 1}v_{k - 1} + \lambda_k\mu_kv_k = \bs{0}.
    \]
    Вычитая отсюда исходное неравенство, умноженное на $\lambda_k$, получаем
    \[
        \mu_1(\lambda_1 - \lambda_k)v_1 + \ldots + (\lambda_{k - 1} - \lambda_k)v_{k - 1} = \bs{0},
    \]
    откуда в силу предположения индукции следует равенство нулю всех коэффициентов. Если найдётся $\mu_j \ne 0$, то $\lambda_j - \lambda_{k - 1} = 0$, однако из условия собственные значения попарно различны. Противоречие. Значит, все $\mu_j$ равны нулю, и система $v_1, \ldots, v_k$ линейно независима.
\end{proof}

\begin{corollary}
    Собственные подпространства $V_{\lambda_1}, \ldots, V_{\lambda_k}$, соответствующие попарно различным собственным значениям $\lambda_1, \ldots, \lambda_k$ оператора $\A$, образуют прямую сумму $V_{\lambda_1} \oplus \ldots \oplus V_{\lambda_k}$.
\end{corollary}

