\section{Поверхности второго порядка (квадрики). Пересечение квадрики с прямой. Центр и вершина квадрики}

\begin{definition}
    Множество вида $\Gamma(Q) \vcentcolon = \{p \in \A : Q(p) = 0\}$, где $Q$ --- аффинно-квадратичная функция, если только оно не пусто и не является плоскостью, называется \textit{квадрикой} или \textit{гиперповерхностью второго порядка}.
\end{definition}

\begin{definition}
    Точка $o \in \A$ называется \textit{центром} квадрики, если эта квадрика симметрична относительно $o$, т.\,е. вместе со всякой точкой $o + x$ ($x \in V$) содержит точку $o - x$. Центр квадрики, лежащий на ней самой, называется её \textit{вершиной}. Квадрика, имеющая (хотя бы один) центр, называется \textit{центральной}.
\end{definition}

Очевидно, что всякий центр аффинно-квадратичной функции $Q$ является центром квадрики $\Gamma(Q)$. Как будет показано ниже, верно и обратное.

\begin{proposal}
    Любая прямая либо целиком лежит на квадрике, либо пересекается с ней не более чем в двух точках.
\end{proposal}

\begin{proof}
    Т.\,к. начало отсчёта $o$ может быть выбрано в любой точке, то без ограничения общности можно считать, что прямая проходит через $o$. Пусть функция $Q$ в векторизованной форме имеет вид $Q(x) = q(x) + l(x) + c$. Тогда пересечение прямой $L = o + \langle x\rangle = \{o + tx : t \in \F\}$ ($x \in V$) с квадрикой $\Gamma(Q)$ определяется условием
    \[
        Q(tx) = t^2q(x) + tl(x) + c,
    \]
    представляющим собой квадратное уравнение относительно $t$. Если все коэффициенты этого уравнения равны $0$, то $L \subset \Gamma(Q)$; в противном случае оно имеет не более двух корней, а это означает, что пересечение $L \cap \Gamma(Q)$ содержит не более двух точек.
\end{proof}

\begin{proposal}
    Если $o$ --- вершина квадрики $\Gamma$, то вместе с любой точкой $p \ne o$ квадрика $\Gamma$ содержит всю прямую $op$.
\end{proposal}

\begin{proof}
    Пусть $p = o + x$ ($x \in V$); тогда $\Gamma$ содержит три различные точки $o$, $o + x$, $o - x$ прямой $op$ и, следовательно, --- всю прямую.
\end{proof}

\begin{definition}
    Всякое подмножество аффинного пространства, содержащее точку $o$ и вместе с любой точкой $p \ne o$ всю прямую $op$ называется \textit{конусом} с вершиной в точке $o$. Квадрика называется \textit{конической}, если она имеет (хотя бы одну) вершину.
\end{definition}

\begin{proposal}
    Всякая квадрика содержит точку, не являющуюся её вершиной.
\end{proposal}

\begin{proof}
    Если бы все точки квадрики были её вершинами, то в силу предложения 2 вместе с любыми двумя точками она содержала бы проходящую через них прямую и, согласно задаче 13, была бы плоскостью, а это противоречит определению квадрики.
\end{proof}

Очевидно, что пропорциональные аффинно-квадратичные функции определяют одну и ту же квадрику. Обратное утверждение не столь очевидно.

\begin{theorem}
    Пусть $\Gamma$ --- квадрика в аффинном пространстве над бесконечным полем $\F$. Если $\Gamma = \Gamma(Q_1) = \Gamma(Q_2)$ для каких-то аффинно-квадратичных функций $Q_1$, $Q_2$, то эти функции пропорциональны.
\end{theorem}

\begin{proof}
    Возьмём в качестве начала отсчёта какую-нибудь точку $o$ квадрики $\Gamma$, не являющуюся её вершиной. Тогда в векторизованной форме
    \[
        Q_1(x) = q_1(x) + l_1(x),\quad Q_2(x) = q_2(x) + l_2(x),
    \]
    где $l_1, l_2 \ne \O$. Точки пересечения прямой $\{o + tx : t \in \F\}$ с квадрикой $\Gamma$ определяются любым из уравнений
    \[
        t^2q_1(x) + tl_1(x) = 0,\quad t^2q_2(x) + tl_2(x) = 0.
    \]

    Т.\,к. эти уравнения должны иметь одинаковые решения (относительно $t$), то при $l_1(x), l_2(x) \ne 0$ мы получаем
    \[
        \frac{q_1(x)}{l_1(x)} = \frac{q_2(x)}{l_2(x)},
    \]
    откуда
    \[
        q_1(x)l_2(x) = q_2(x)l_1(x).\eqno(\ast)
    \]
    Как следствие\footnotemark, это верно при всех $x$.

    \footnotetext{
        Пусть поле $\F$ бесконечно и $h \in \F[x_1, x_2, \ldots, x_n]$ --- какой-либо ненулевой многочлен. Если многочлены $f, g \in \F[x_1, \ldots, x_n]$ принимают одинаковые значения при всех значениях переменных $x_1, \ldots, x_n$, при которых многочлен $h$ не обращается в нуль, то они равны.

        \begin{proof}
            При указанных условиях многочлены $fh$ и $gh$ принимают одинаковые значения вообще при всех значениях переменных, и, значит, $fh = gh$ (теорема из первого семестра). Т.\,к. в алгебре многочленов нет делителей нуля, то отсюда следует, что $f = g$.
        \end{proof}

        Если поле $\F$ конечно, то теорема и её доказательство тем не менее остаются в силе для многочленов, степень которых по каждому из переменных меньше числа элементов поля $\F$.\hfill\textit{Из Винберга}
    }

    Предположим, что линейные функции $l_1$ и $l_2$ не пропорциональны. Тогда в подходящем базисе $l_1(x) = x_1$, $l_2(x) = x_2$ и равенство записывается в виде
    \[
        q_1(x)x_2 = q_2(x)x_1.
    \]

    Рассматривая члены в левой и правой части, мы видим, что должно быть
    \[
        q_1(x) = l(x)x_1,\quad q_2(x) = l(x)x_2,
    \]
    где $l(x)$ --- какая-то линейная функция, и, значит,
    \[
        Q_1(x) = (l(x) + 1)x_1,\quad Q_2(x) = (l(x) + 1)x_2.
    \]

    Т.\,к. $\Gamma = \Gamma(Q_1)$, то $\Gamma$ содержит гиперплоскость $x_1 = 0$. Т.\,к. в то же время $\Gamma = \Gamma(Q_2)$, то $Q_2$ должна тождественно обращаться в нуль на этой гиперплоскости. Однако ни один из её множителей $l(x) + 1$ и $x_2$ не обращается на ней в нуль тождествено (первых из них не обращается в нуль уже в точке $o$). Поскольку в алгебре многочленов нет делителей нуля, мы приходим к противоречию.

    Итак, $l_2 = \lambda l_1$ ($\lambda \in \F^\ast$). Из $(\ast)$ получаем тогда, что и $q_2 = \lambda q_1$, и, значит, $Q_2 = \lambda Q_1$.
\end{proof}

\begin{corollary}
    Всякий центр квадрики $\Gamma(Q)$ является также центром функции $Q$.
\end{corollary}

\begin{proof}
    Если $o$ --- центр квадрики $\Gamma(Q)$, то $\Gamma(Q) = \Gamma(\overline{Q})$, где
    \[
        \overline{Q}(o + x) = Q(o - x).
    \]

    Следовательно, $\overline{Q} = \lambda Q$ ($\lambda \in \F^\ast$). Сравнивая члены второй степени в выражениях $\overline{Q}$ и $Q$, мы видим, что должно быть $\lambda = 1$, т.\,е. $\overline{Q} = Q$, а это и означает, что $o$ --- центр функции $Q$.
\end{proof}

\begin{corollary}
    Если квадрика $\Gamma(Q)$ инвариантна относительно некоторого параллельного переноса, то и функция $Q$ тоже инвариантна относительно этого переноса.
\end{corollary}

\begin{proof}
    Если квадрика переходит в себя при параллельном переносе на вектор $a \in V$, то $\Gamma(Q) = \Gamma(\overline{Q})$, где
    \[
        \overline{Q}(p) = Q(p + a).
    \]
    Далее рассуждаем так же, как в доказательстве следствия 1.
\end{proof}

\begin{remark}
    Анализируя приведённого выше доказательства с учётом замечания в последней сноске показывает, что она верна и для конечных полей, исключая только поле $\Z_3$ (напомним, что мы считаем, что $\operatorname{char}\F \ne 2$). Над полем $\Z_3$ можно привести следующий контрпример:
    \[
        Q_1 = x_1^2 + x_1x_2 + 1 = 0,\quad Q_2 = x_2^2 + x_1x_2 + 1 = 0.
    \]
    задают одну и ту же конику в $\Z_3^2$, состоящую из точек $(1, 1)$ и $(-1, -1)$. Однако оба следствия верны и для поля $\Z_3$.
\end{remark}

\begin{definition}
    $\Ker Q \vcentcolon = \Ker \widehat{q} \cap \Ker l$.
\end{definition}

\begin{proposal}
    Функция $Q$ инвариантна относительно параллельного переноса на вектор $a$ тогда и только тогда, когда $a \in \Ker Q$.
\end{proposal}

\begin{proof}
    Инвариантность функции $Q$ относительно параллельного переноса на вектор $a$ равносильна тому, что она сохраняет свой вид при переносе начала отсчёта в точку $o^\prime = o + a$. Ввиду леммы 1 из предыдущего вопроса это происходит тогда и только тогда, когда $a \in \Ker Q$.
\end{proof}

\begin{corollary}
    $\Ker \widehat{q} \cap \Ker l$ не зависит от выбора системы координат.
\end{corollary}

Таким образом, если $\Ker Q \ne \{\bs{0}\}$, то квадрика $\Gamma = \Gamma(Q)$ вместе с каждой точкой $p \in \A$ содержит целую плоскость $p + \Ker Q$.

\begin{definition}
    Такая квадрика называется \textit{цилиндрической} с \textit{касательным пространством} $\Ker Q$.
\end{definition}

Выберем базис пространства $V$ так, чтобы последние $d \vcentcolon = \dim W$ векторов были базисом подпространства $W$. Тогда выражение $Q$ не будет содержать последних $d$ координат. Пусть $\A_0 \subset \A$ --- любая плоскость, касательное пространство которой натянуто на первые $n - d$ базисных векторов, и $\Gamma_0$ --- квадрика, задаваемая в $\A_0$ уравнением $Q = 0$ (т.\,е. $\Gamma_0 = \Gamma \cap \A_0$). Тогда $\Gamma = \Gamma_0 + W$.

\begin{definition}
    Квадрика, не являющаяся цилиндрической, называется \textit{невырожденной}.
\end{definition}

\begin{proposal}
    Невырожденная квадрика имеет не более одного центра.
\end{proposal}

\begin{proof}
    Пусть $o$ и $o^\prime$ --- центры квадрики $\Gamma$. Обозначим через $s$ и $s^\prime$ центральные симметрии относительно $o$ и $o^\prime$ соответственно. Тогда $s\Gamma = s^\prime\Gamma = \Gamma$ и, следовательно, $ss^\prime \Gamma = \Gamma$. Т.\,к.
    \[
        d(ss^\prime) = ds \cdot ds^\prime = (-\id)^2 = \id,
    \]
    то $ss^\prime$ --- (нетривиальный) параллельный перенос и, значит, квадрика $\Gamma$ цилиндрическая.
\end{proof}

