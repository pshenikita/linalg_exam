\section{Аннулирующие многочлены линейного оператора (матрицы). Минимальный многочлен. Критерий диагонализируемости в терминах минимального многочлена}

Пусть $\A: V \to V$ --- оператор. Каждому многочлену $P(t) = p_o + p_1t + \ldots + p_nt^n \in \F[t]$ можно сопоставить оператор $P(\A) \vcentcolon = p_0 \cdot \id + p_1 \cdot \A + \ldots + p_n\A^n$.

\begin{definition}
    Такой оператор называется \textit{многочленом от оператора} $\A$.
\end{definition}

\begin{definition}
    Многочлен $P(t)$ называют \textit{аннулирующим} оператор $\A$, если $P(\A) = \O$.
\end{definition}

\begin{example}
    \begin{enumerate}[nolistsep]
        \item Многочлен $t - 1$ аннулирует оператор $\id$.
        \item Многочлен $t^2 - t$ аннулирует любой проектор (см. приложение про проекторы).
    \end{enumerate}
\end{example}

\begin{proposal}
    У любого оператора существует ненулевой аннулирующий многочлен.
\end{proposal}

\begin{proof}
    Пусть $\dim V = n$. Рассмотрим $n^2 + 1$ операторов: $\A^0 = \id, \A^1 = \A, \A^2, \ldots, \A^{n^2}$. Т.\,к. размерность пространства операторов равна $n^2$, эти операторы линейно зависимы, т.\,е. существуют числа $p_0, p_1, p_2, \ldots, p_{n^2}$, не все равные нулю, такие, что $p_o \cdot \id + p_1 \cdot \A + p_2 \cdot \A^2 + \ldots + p_{n^2} \cdot \A^{n^2} = 0$. Тогда ненулевой многочлен $P(t) = p_0 + p_1t + \ldots + p_{n^2}t^{n^2}$ аннулирует $\A$.
\end{proof}

\begin{definition}
    Ненулевой многочлен со старшим коэффициентом $1$ минимальной возможной степени среди аннулирующих многочленов оператора $\A$ называется \textit{минимальным многочленом} для оператора $\A$.
\end{definition}

\begin{lemma}
    Пусть $\A: V \to V$ --- линейный оператор конечномерного векторного пространства $V$. Тогда
    \begin{enumerate}[nolistsep]
        \item минимальный многочлен линейного оператора равен минимальному многочлену его матрицы в любом базисе;
        \item минимальный многочлен является делителем любого аннулирующего многочлена;
        \item минимальный многочлен единственный;
        \item наборы корней у минимального многочлена и характеристического многочлена совпадают.
    \end{enumerate}
\end{lemma}

\begin{proof}
    \begin{enumerate}
        \item Из доказательства теоремы 1 из вопроса 11 следует, что если $f(t) \in \F[t]$ --- произвольный многочлен с коэффициентами из $\F$, то для матрицы $A$ линейного оператора $\A$ в произвольном базисе выполняется $f(A) = 0 \Leftrightarrow f(\A) = \O$. Откуда и получаем первое утверждение.
        \item Пусть $f(t)$ --- произвольный аннулирующий многочлен для $\A$. Поделим многочлен $f(t)$ на минимальный многочлен $\mu_\A(t)$ линейного оператора $\A$ с остатком: $f(t) = \mu_\A(t)q(t) + r(t)$, при этом для остатка $r(t)$ должно выполняться $\deg r < \deg \mu_\A$. Подставляя $\A$ вместо $t$, получаем
            \[
                r(\A) = f(\A) - \mu_\A(\A)q(\A) = 0 - 0 \cdot q(\A) = 0,
            \]
            т.\,е. $r(t)$ аннулирует $\A$ и $\deg r < \deg \mu_\A$, поэтому $r(t) = 0$, т.\,е. деление происходит нацело.
        \item Любой из двух минимальных многочленов делит другой, а их старшие коэффициенты совпадают.
        \item Т.\,к. характеристический многочлен $\chi_\A(t)$ линейного оператора $\A$ является его аннулирующим многочленом (по теореме Гамильтона "---Кэли, будет позднее), из п.\,1 получаем, что $\mu_\A$ делит $\chi_\A$. Значит, корни минимального многочлена являются корнями характеристического многочлена. Обратно, пусть $\lambda$ --- корень характеристического многочлена. Тогда у $\A$ есть собственный вектор $v$ с собственным значением $\lambda$. Применим к вектору $v$ оператор $\mu_\A$. С одной стороны, расписав минимальный многочлен явно:
            \[
                \mu_\A = t^k + b_{k - 1}t^{k - 1} + \ldots + b_1t + b_0
            \]
            и подставив линейный оператор $\A$ вместо переменной $t$, получаем
            \begin{multline*}
                \mu_\A(\A)v = (\A^k + b_{k - 1}\A^{k - 1} + \ldots + b_1\A + b_0 \cdot \id)v =\\ = \A^kv + b_{k - 1}\A^{k - 1}v + \ldots + b_1\A v + b_0 \cdot \id v = \lambda^kv + b_{k - 1}\lambda^{k - 1}v + \ldots + b_1\lambda v + b_0v =\\ = (\lambda^k + b_{k - 1}\lambda^{k - 1} + \ldots + b_1\lambda + b_0)v = \mu_\A(\lambda) \cdot v.
            \end{multline*}

            С другой стороны, т.\,к. $\mu_\A(\A) = \O$, имеем $\mu_\A(\A)v = 0 \cdot v = \bs{0}$. Откуда $\mu_\A(\lambda)v = \bs{0}$. Т.\,к. $v \ne \bs{0}$, то число $\mu_\A(\lambda)$ равно нулю, т.\,е. $\lambda$ --- корень минимального многочлена.
    \end{enumerate}
\end{proof}

\begin{theorem}[Критерий диагонализируемости в терминах минимального многочлена]
    Линейный оператор $\A: V \to V$ конечномерного векторного пространства $V$ над алгебраически замкнутым полем $\F$ диагонализируем тогда и только тогда, когда его минимальный многочлен не имеет кратных корней.
\end{theorem}

\begin{proof}
    Пусть сначала $e_1, \ldots, e_n$ --- базис из собственных векторов для $\A$, причём $\A e_i = \lambda_ie_i$, т.\,е. матрица оператора $\A$ в выбранном базисе диагональна:
    \[
        A =
        \begin{pmatrix}
            \lambda_1 &  &  & \\
             & \lambda_2 &  &  \\
             &  & \ddots &  \\
             &  &  & \lambda_n \\
        \end{pmatrix}.
    \]

    Т.\,к. минимальный многочлен линейного оператора равен минимальному многочлену его матрицы в произвольном базисе, то $\mu_\A(t) = (t - \alpha_1)\ldots(t - \alpha_p)$, где $\alpha_1, \ldots, \alpha_p$ --- все попарно различные числа $\lambda_1, \ldots, \lambda_n$. Это следует из того, что выписанный многочлен является аннулирующим для матрицы $A$, а минимальный многочлен делит любой аннулирующий многочлен и наборы корней у минимального и характеристического многочлена совпадают.

    Пусть теперь минимальный многочлен не имеет кратных корней. И пусть $J$ --- жорданова нормальная форма оператора $\A$, которая существует в силу алгебраической замкнутости поля $\F$. Позже будет написано, как искать минимальный многочлен для жордановой матрицы, а сейчас воспользуемся этим результатом:
    \[
        \mu_\A(t) = \mu_J(t) = (t - \alpha_1)^{m_1}\ldots(t - \alpha_p)^{m_p},
    \]
    где $\alpha_1, \ldots, \alpha_p$ --- все попарно различные собственные значения линейного оператора $\A$, $m_i$ --- максимальный порядок жордановой клетки с $\alpha_i$ на диагонали в жордановой нормальной форме $J$. Т.\,к. минимальный многочлен не имеет кратных корней, получаем, что $m_i = 1$ для всех $i$, т.\,е. в $J$ все клетки имеют порядок $1$. Это и означает, что жорданова нормальная форма диагональна.
\end{proof}

\begin{problem}[\href{https://math.stackexchange.com/questions/136804/can-a-matrix-in-mathbbr-have-a-minimal-polynomial-that-has-coefficients-in}{math.stackexchange}]
    Пусть $\K$ --- поле и $\F \supseteq \K$ --- его расширение. Пусть также $A \in \underset{n \times n}\Mat(\K)$ и $\mu_{A, \K}$ --- минимальный многочлен $A$. Заметим, что эту матрицу $A$ можно также рассмотреть как матрицу из $\underset{n \times n}\Mat(\F)$, и пусть в этом случае её минимальный многочлен есть $\mu_{A, \F}$. Доказать, что $\mu_{A, \K} = \mu_{A, \F}$.
\end{problem}

Нам понадобится две леммы.

\begin{lemma}
    Если $A \in \underset{n \times n}{\Mat}(\K)$, то $\deg\mu_{A, \K} = d$ тогда и только тогда, когда выполнены следующие два условия:
    \begin{enumerate}[nolistsep]
        \item матрицы $E, A, A^2, \ldots, A^{d - 1}$ линейно независимы над полем $\K$;
        \item матрицы $E, A, A^2, \ldots, A^{d - 1}, A^d$ линейно зависимы над полем $\K$;
    \end{enumerate}
\end{lemma}

\begin{proof}
    $\Rightarrow$. Пусть $\mu_{A, \K}(t) = a_0 + a_1t + a_2t^2 + \ldots + a_dt^d$. Тогда $\mu_{A, \K}(A) = 0$, т.\,е.
    \[
        a_0E + a_1A + a_2A^2 + \ldots + a_dA^d = 0.
    \]
    Причём, не все $a_i$ равны нулю. Значит, выполнено условие 2. А условие 1 выполнено в силу минимальности $\mu_{A, \K}$. Действительно, если $E, A, A^2, \ldots, A^{d - 1}$ линейно зависимы, то коэффициенты этой линейной зависимости можно выбрать в качестве коэффициентов аннулирующего многочлена степени $d - 1$, а такого нет, потому что $\deg\mu_{A, \K}$ минимальна для аннулирующих многочленов.

    $\Leftarrow$. Эти условия ровно то и означают: существует аннулирующий многочлен степени $d$, но не существует аннулирующего многочлена степени $d - 1$.
\end{proof}

\begin{lemma}
    Если $\F \supseteq \K$ --- расширение поля и $v_1, \ldots, v_r \in \K^N \subseteq \F^N$. Тогда $v_1, \ldots, v_r$ линейно зависимы над полем $\K$ тогда и только тогда, когда они линейно зависимы над полем $\F$.
\end{lemma}

\begin{proof}
    $\Rightarrow$. Очевидно, т.\,к. коэффициенты этой линейной зависимости лежат как в $\K$, так и в $\F$ в силу $\K \subseteq \F$.

    $\Leftarrow$. Пусть $v_i = (v^i_1, v^i_2, \ldots, v^i_N)^t$ и пусть $x_1, \ldots, x_r \in \F^N$ --- коэффициенты линейной зависимости $v_1, \ldots, v_r$. Тогда
    \[
        x_1
        \begin{pmatrix}
            v^1_1\\
            \vdots\\
            v^1_N
        \end{pmatrix} +
        x_2
        \begin{pmatrix}
            v^2_1\\
            \vdots\\
            v^2_N
        \end{pmatrix} + \ldots +
        x_r
        \begin{pmatrix}
            v^r_1\\
            \vdots\\
            v^r_N
        \end{pmatrix} + x_{r + 1}v^\prime_{r + 1} + \ldots + x_Nv^\prime_N = \bs{0},
    \]
    где векторы $v^\prime_i$ ($i = r + 1, \ldots, N$) выбираются так, чтобы быть линейно независимыми с векторами $v_1, \ldots, v_r, v^\prime_{r + 1}, \ldots, v^\prime_{i - 1}$. Это можно сделать, т.\,к. при $i \leqslant N$ эти векторы образуют подпространство размерности не более $N$, в качестве следующего вектора можно взять любой вектор не из этого подпространства. Заметим, что в любом решении этой системы $x_{r + 1} = \ldots = x_N = 0$, ведь мы никак не сможем нейтрализовать слагаемое $x_iv^\prime_i$ ($i = r + 1, \ldots, N$), потому что этот вектор линейно независим со всеми остальными векторами. Значит, наша дополненная векторами со штрихами система имеет ненулевое решение тогда и только тогда, когда система без штрихов имеет ненулевое решение, т.\,е. тогда и только тогда, когда $v_1, \ldots, v_r$ линейно независимы.

    Получили СЛУ с матрицей коэффициентов $V = (v^i_j)^{i = 1, \ldots N}_{j = 1, \ldots, N}$. Векторы $v_1, \ldots, v_r$ линейно независимы над $\K$ $\Leftrightarrow$ система имеет ненулевое решение $\Leftrightarrow$ $\det V \ne 0$ над $\K$ $\Leftrightarrow$ $\det V \ne 0$ над $\F$ (тут нам неважно, над каким полем считать, все вычисления всё равно происходят внутри $\K$) $\Leftrightarrow\ldots\Leftrightarrow$ векторы $v_1, \ldots, v_r$ линейно независимы над $\F$.
\end{proof}

\begin{remark}
    Векторы со штрихами нужно было добавлять, чтобы сделать матрицу СЛУ квадратной, чтобы можно было считать её определитель. Сначала решение было написано без этого дополнения, но в таком виде оно, конечно, неверное.
\end{remark}

Теперь можно и задачу решить.

\begin{solution}
    Понятно, что $\mu_{A, \F}$ делит $\mu_{A, \K}$. Так как их старшие коэффициенты равны, достаточно показать, что их степени равны. А это очевидное следствие из двух предыдущих лемм.
\end{solution}

