\section{Аффинно-квадратичные функции. Центр. Приведение аффинно-квадратичной функции к каноническому виду заменой аффинной системы координат}

Будем считать, что $\operatorname{char}\F \ne 2$.

\begin{definition}
    \textit{Аффинно-квадратичной функцией} на аффинном постранстве $(\A, V)$ называется всякая функция $Q: \A \to \F$, имеющая в векторизованной форме вид
    \[
        Q(x) = q(x) + l(x) + c,
    \]
    где $q$ --- квадратичная функция, $\ell$ --- линейная функция, $c \in \F$ --- константа.
\end{definition}

Пусть $\widehat{q}$ --- соответствующая квадратичной функции $q$ билинейная форма.

\begin{lemma}
    При переносе начала отсчёта $o$ в точку $o^\prime = o + a$ ($a \in V$) слагаемые выражения из определения аффинно-квадратичной функции преобразуются следующим образом:
    \[
        q^\prime(x) = q(x),\quad l^\prime(x) = 2\widehat{q}(a, x) + l(x),\quad c^\prime = q(a) + l(a) + c.
    \]
\end{lemma}

\begin{proof}
    Непосредственно проверяем:
    \begin{multline*}
        Q(o^\prime + x) = Q(o + a + x) = q(a + x) + l(a + x) + c =\\ = q(a) + 2\widehat{q}(a, x) + q(x) + l(a) + l(x) + c =\\ = q(x) + (2\widehat{q}(x) + l(x)) + (q(a) + l(a) + c).
    \end{multline*}
\end{proof}

В частности, квадратичная функция $q$ не зависит от выбора начала отсчёта. В координатах выражение аффинно-квадратичной функции принимает вид
\[
    Q(x) = \sum_{i, j}x_ix_ja_{ij} + \sum_ix_ib_i + c\quad(a_{ij} = a_{ji}).
\]

Коэффициентам $b_i$ и $c$ можно придать следующий смысл:
\[
    c = Q(o),\quad b_i = \frac{\partial Q}{\partial x_i}(o).
\]

\begin{definition}
    Линейная функция $l(x) = \sum\limits_ix^ib_i$ называется \textit{дифференциалом} функции $Q$ в точке $o$ и обозначается $d_oQ$.
\end{definition}

В случае $\F = \R$ это согласуется с обычным определением дифференциала.

\begin{definition}
    Точка $o$ называется \textit{центром} аффинно-квадратичной функции $Q$, если
    \[
        Q(o + x) = Q(o - x)\quad\forall x \in V.
    \]
\end{definition}

Ясно, что это имеет место тогда и только тогда, когда $d_oQ = \O$. Поэтому множество всех центров функции $Q$ задаётся системой линейных уравнений
\[
    \frac{\partial Q}{\partial x^1} = \ldots = \frac{\partial Q}{\partial x^n} = 0.
\]

Оно либо является плоскостью какой-то размерности, либо пусто. Легко видеть, что матрица коэффициентов это системы есть удвоенная матрица квадратичной функции $q$. Следовательно, если $q$ невырожденна, то $Q$ имеет единственный центр.

\begin{theorem}
    Для любой аффинно-квадратичной функции $Q: \A \to \F$, не являющеся аффинно-линейной, существует система координат $(o; e_1, \ldots, e_n)$, в которой $Q$ имеет один из следующих видов $(\ast)$:
    \begin{gather*}
        Q(a) = \lambda_1x_1^2 + \ldots + \lambda_rx_r^2 + \lambda_{r + 1},\\
        Q(a) = \lambda_1x_1^2 + \ldots + \lambda_rx_r^2 + 2x_{r + 1},
    \end{gather*}
    где числа $\lambda_1, \ldots, \lambda_r$ отличны от нуля.
\end{theorem}

\begin{proof}
    Существует базис $e_1^\prime, \ldots, e_n^\prime$ векторного пространства $V$, в котором матрица квадратичной функции $q$ диагональна:
    \[
        B =
        \begin{pmatrix}
            \lambda_1 &  &  &  &  &  \\
              & \ddots &  &  &  & \\
              &  & \lambda_r &  &  & \\
              &  &  & 0 &  & \\
              &  &  &  & \ddots & \\
              &  &  &  & & 0\\
        \end{pmatrix},
    \]
    где числа $\lambda_1, \ldots, \lambda_r$ отличны от нуля, т.\,к. функция $Q$ не является аффинно-линейной.

    Фиксируем произвольную точку $o^\prime \in \A$. В системе координат $(o^\prime; e_1^\prime, \ldots, e_n^\prime)$ функция $Q$ примет вид
    \[
        Q(o^\prime + v) = \lambda_1{x_1^\prime}^2 + \ldots + \lambda_r{x_r^\prime}^2 + 2u_1^\prime x_1^\prime + \ldots + 2u_n^\prime x_n^\prime + \alpha^\prime.
    \]

    Перенос начала координат в точку $o^{\prime\prime}$, сводящийся к замене координат
    \begin{align*}
        x_i^{\prime\prime} &= x_i^\prime + \frac{u_i^\prime}{\lambda_i},\quad i = 1, \ldots, r,\\
        x_i^{\prime\prime} &= x_i^\prime,\quad i = r + 1, \ldots, n,
    \end{align*}
    приводит функцию $Q$ к виду
    \[
        Q(o^{\prime\prime} + v) = \lambda_1{x_1^{\prime\prime}}^2 + \ldots + \lambda_r{x_r^{\prime\prime}}^2 + 2u_{r + 1}^\prime x_{r + 1}^{\prime\prime} + \ldots + 2u_n^\prime x_n^{\prime\prime} + \alpha^{\prime\prime}.
    \]

    Если $u^\prime_{r + 1} = \ldots = u_n^\prime = 0$, то $Q$ с точностью до обозначения имеет вид, как в первой строчке $(\ast)$, в этом случае $Q$ имеет центр.

    Если хотя бы один из коэффициентов $u_{r + 1}^\prime$ отличен от нуля (для определённости считаем, что это $u_{r + 1}^\prime$), то ещё одна замена координат
    \begin{align*}
        x_{r + 1} &= u^\prime_{r + 1}x_{r + 1}^{\prime\prime} + \ldots + u_n^\prime x_n^{\prime\prime} + \frac{\alpha^{\prime\prime}}{2},\\
        x_i &= x_i^{\prime\prime},\quad i \ne r + 1
    \end{align*}
    приведёт $Q$ к виду, как во второй строчке $(\ast)$, в этом случае $Q$ не имеет центра.
\end{proof}

