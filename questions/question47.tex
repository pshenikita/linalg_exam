\section{Аффинно независимые системы точек. Барицентрическая комбинация точек. Примеры}

Здесь я в основном основывался на Винберге, а у него хронологически плоскости появляются раньше аффинной независимости, так что сначала лучше прочитать вопрос 48.

\begin{definition}
    Точки $p_0, p_1, \ldots, p_k \in \A$ называются \textit{аффинно независимыми}, если \[\dim\aff\{p_0, p_1, \ldots, p_k\} = k,\] и \textit{аффинно зависимыми} в противном случае.
\end{definition}

Из доказательства теоремы 1 в следующем вопросе видно точки $p_0, p_1, \ldots, p_k$ аффинно зависимы тогда и только тогда, когда векторы $\overline{p_0p_1}, \ldots, \overline{p_0p_k}$ линейно зависимы. В то же время, из определения ясно, что свойство точек быть аффинно зависимыми или независимыми не зависит от их нумерации.

Линейные комбинации точек аффинного пространства, вообще говоря, неопределены. Однако некоторым из них можно придать смысл. А именно,

\begin{definition}
    Назовём \textit{барицентрической комбинацией} точек $p_1, \ldots, p_k \in \A$ аффинного пространства $(\A, V)$ линейную комбинацию вида $\sum\limits_i\lambda_ip_i$, где $\sum\limits_i\lambda_i = 1$, и будем считать её равной точке $p$, определяемой равенством
    \[
        \overline{op} = \sum_{i = 1}^k\lambda_i\overline{op_i},
    \]
    где $o \in \A$.
\end{definition}

Благодаря условию на сумму коэффициентов это определение не зависит от выбора точки $o$. Действительно, пусть $o^\prime$ --- любая другая точка. Тогда
\[
    \overline{o^\prime p} = \overline{o^\prime o} + \overline{op} = \sum_{i = 1}^k\lambda(\overline{o^\prime o} + \overline{op_i}) = \sum_{i = 1}^k\lambda_i\overline{o^\prime p_i}.
\]

В частности,

\begin{definition}
    \textit{Центр тяжести системы точек} $\{p_1, \ldots, p_k\}$ определим как
    \[
        \operatorname{center}(p_1, \ldots, p_k) = \frac{p_1 + \ldots + p_k}{k}.
    \]
\end{definition}

\begin{proposal}
    Барицентрическая комбинация $\lambda p + \mu q$ для точек $p, q \in \A$ аффинного пространства $(\A, V)$ есть точка $r$, лежащая на прямой $pq$ и обладающая тем свойством, что
    \[
        \overline{pr} = \frac{\mu}{\lambda}\overline{rq}
    \]
    (если $\lambda = 0$, $\mu = 1$, то $r = q$).
\end{proposal}

\begin{proof}
    Приняв точку $r$ за опорную точку в определении барицентрической комбинации, мы получаем:
    \[
        0 = \lambda\overline{rp} + \mu\overline{rq},
    \]
    откуда и следует требуемое.
\end{proof}

Вообще, барицентрические координаты --- очень далеко идущая тема. Вокруг них развита довольно мощная техника для решения геометрических задач. Почитать про это можно \href{https://old.mccme.ru//mmmf-lectures//books/books/book.40.pdf}{здесь} и здесь в соответствующем приложении.

