\section{Собственные подпространства. Неравенство между размерностью собственного подпространства и кратностью корня характеристического многочлена}

\begin{proposal}
    Все собственные векторы, отвечающие собственным значениям $\lambda$, и вектор $\bs{0}$ образуют подпространство, которое совпадает с $\Ker(\A - \lambda \cdot \id)$.
\end{proposal}

\begin{proof}
    По определению ядра, равенство $\A v = \lambda v$ имеет место тогда и только тогда, когда $v \in \Ker(\A - \lambda \cdot \id)$.
\end{proof}

\begin{definition}
    Пусть $\lambda$ --- собственное значение для оператора $\A$. Подпространство $V_\lambda \vcentcolon = \Ker(\A - \lambda \cdot \id)$ называется \textit{собственным подпространством}, соответствующим $\lambda$.
\end{definition}

\begin{proposal}
    Каждое собственное подпространство $V_\lambda$ оператора $\A$ является инвариантным относительно него.
\end{proposal}

\begin{proof}
    Действительно, если $v \in V_\lambda$, то $\A v = \lambda v \in V_\lambda$.
\end{proof}

\begin{proposal}
    Размерность собственного подпространства $V_\lambda$ не превосходит кратности $\lambda$ как корня характеристического многочлена.
\end{proposal}

\begin{proof}
    Пусть $\dim V_\lambda = k$. Выберем базис $e_1, \ldots, e_k$ в подпространстве $V_\lambda$ и дополним его до базиса в $V$. Т.\,к. $\A e_i = \lambda e_i$ при $i = 1, \ldots, k$, по предложению 1 из вопроса 12 матрица оператора $\A$ в выбранном базисе имеет вид
    $
    A = 
    \left(
    \begin{array}{c c c | c}
        \lambda & & 0 & \\
         & \ddots & & \ast\\
        0 & & \lambda & \\
        \hline
         & & & \vspace{-4mm}\\
         & 0 & & \widetilde{A}
    \end{array}
    \right)
    $. Тогда \[\chi_\A(t) = \det(A - t \cdot E) = (\lambda - t)^k\det(\widetilde{A} - t \cdot E) = (\lambda - t)^k\chi_{\widetilde{\A}}(t),\] где $\widetilde{\A}$ --- фактор-оператор. Отсюда вытекает, что кратность корня $\lambda$ не меньше $k = \dim V_\lambda$.
\end{proof}

