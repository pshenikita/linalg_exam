\section{Существование жордановой нормальной формы матрицы над алгебраически замкнутым полем}

Посмотрим, как выглядит матрица оператора $\A\big|_{R_\lambda}$ (ограничения оператора $\A$ на корневое подпространства $R_\lambda$). Т.\,к. $(\A - \lambda \cdot \id)\big|_{R_\lambda}$ является нильпотентным оператором, в пространстве $R_\lambda$ можно выбрать нормальный базис для этого оператора. Тогда матрица оператора $(\A - \lambda \cdot \id)\big|_{R_\lambda}$ в этом базисе будет состоять из блоков вида
$
\begin{pmatrix}
    0 & 1 &  &  \\
      & 0 & \ddots & \\
      & & \ddots & 1\\
      & & & 0
\end{pmatrix}
$. А значит, матрица оператора $\A\big|_{R_\lambda}$ состоит из блоков вида
$
\begin{pmatrix}
    \lambda & 1 &  &  \\
      & \lambda & \ddots & \\
      & & \ddots & 1\\
      & & & \lambda
\end{pmatrix}
$, т.\,е. жордановых клеток.

\begin{theorem}
    Для любого оператора $\A$ в пространстве $V$ над алгебраически замкнутым полем существует жорданов базис (в котором оператор имеет жорданову нормальную форму).
\end{theorem}

\begin{proof}
    Существование жордановой формы является прямым следствием теорем о разложении в прямую сумму корневых подпространств и существовании нормального вида для нильпотентных операторов. Действительно, пусть $\lambda_1, \ldots, \lambda_k$ --- все собственные значения $\A$. Выберем в каждом корневом подпространстве $R_{\lambda_i}$ нормальный базис для нильпотентного оператора $(\A - \lambda_i \cdot \id)\big|_{R_{\lambda_i}}$. Тогда объединение этих базисов даст жорданов базис для оператора $\A$ в силу наличия корневого разложения $V = R_{\lambda_1} \oplus \ldots \oplus R_{\lambda_k}$ (здесь мы пользуемся алгебраической замкнутостью поля).
\end{proof}

Этот билет я сдавал на коллоквиуме Антону Александровичу, и он предложил мне следующую интересную задачу.

\begin{problem}[А.\,А. Клячко]
    Верно ли, что если у каждого оператора в линейном пространстве $V$ над полем $\F$ есть жорданова форма, то поле $\F$ алгебраически замкнутое?
\end{problem}

Нам пригодится вспомогательное утверждение (которое интересно и само по себе).

\begin{lemma}
    Пусть имеем $n$-мерное векторное пространство $V$ над полем $\F$. Тогда для любого многочлена $p \in \F[t]$ степени не больше $n$ существует оператор $\A$ такой, что $\chi_\A = p$.
\end{lemma}

\begin{proof}
    Фиксируем некоторый базис $e_1, \ldots, e_n$ пространства $V$ и рассмотрим оператор с матрицей
    \[
        A = \begin{pmatrix}
            0 & 0 & \cdots & 0 & -a_0\\
            1 & 0 & \cdots & 0 & -a_1\\
            0 & 1 & \cdots & 0 & -a_2\\
            \vdots & \vdots & \ddots & \vdots & \vdots\\
            0 & 0 & \cdots & 1 & -a_{n - 1}
        \end{pmatrix}
    \]
    в этом базисе. Здесь $a_1, \ldots, a_n$ --- произвольные элементы $\F$. Тогда можем вычислить его характеристический многочлен (вычисления приведены ниже). Небольшой комметарий к ним: определитель считается с помощью разложения по последней строке. Как известно, в таком разложении знак у слагаемого выбирается <<в шахматном порядке>>, причём в левой верхней клетке матрицы стоит <<$+$>>. На самом деле, <<$+$>> будет во всех клетках на главной диагонали.

    \begin{multline*}
        \chi_\A(t) = \det(A - t \cdot E) = \det
        \begin{pmatrix}
            -t & 0 & \cdots & 0 & -a_0\\
            1 & -t & \cdots & 0 & -a_1\\
            0 & 1 & \cdots & 0 & -a_2\\
            \vdots & \vdots & \ddots & \vdots & \vdots\\
            0 & 0 & \cdots & 1 & -a_{n - 1} - t
        \end{pmatrix} = (-1)^{n + 1} \cdot (-a_0) \cdot \det
        \begin{pmatrix}
            1 & -t & \cdots & 0\\
            0 & 1 & \cdots & 0\\
            \vdots & \vdots & \ddots & \vdots\\
            0 & 0 & \cdots & 1
        \end{pmatrix} + \\
        + (-1)^n \cdot (-a_1) \cdot \det
        \begin{pmatrix}
            -t & 0 & \cdots & 0\\
            0 & 1 & \cdots & 0\\
            \vdots & \vdots & \ddots & \vdots\\
            0 & 0 & \cdots & 1
        \end{pmatrix} + (-1)^{n + 1} \cdot (-a_2) \cdot \det
        \begin{pmatrix}
            -t & 0 & \cdots & 0\\
            1 & -t & \cdots & 0\\
            \vdots & \vdots & \ddots & \vdots\\
            0 & 0 & \cdots & 1
        \end{pmatrix} + \ldots + (-a_{n - 1} - t) \cdot {}\\{} \cdot \det
        \begin{pmatrix}
            -t & 0 & \cdots & 0\\
            1 & -t & \cdots & 0\\
            \vdots & \vdots & \ddots & \vdots\\
            0 & 0 & \cdots & -t
        \end{pmatrix} = (-1)^na_0 + (-1)^na_1 \cdot t + (-1)^na_2 \cdot t^2 + \ldots + (-1)^na_{n - 1} \cdot t^{n - 1} + (-1)^nt^n.
    \end{multline*}
\end{proof}

Теперь приступим к решению задачи:

\begin{solution}
    Возьмём линейный оператор $\A: V \to V$ и рассмотрим его матрицу $A$ в фиксированном базисе $e_1, \ldots, e_n$ $n$-мерного пространства $V$ ($n$ можно брать любым) над полем $\F$. С одной стороны, её характеристический многочлен (из предыдущей леммы) может быть любым многочленом из $\F[t]$ степени не выше $n$. С другой стороны, характеристический многочлен жордановой матрицы $J_A$ раскладывается на линейные множители. А т.\,к. $A \sim J_A$, это тот же самый многочлен. Получаем, что любой многочлен из $\F[t]$ раскладывается на линейные множители, что значит, что поле $\F$ алгебраически замкнуто.
\end{solution}

