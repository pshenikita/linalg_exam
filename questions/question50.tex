\section{Аффинное отображение и его линейная часть. Композиция аффинных отображений. Критерий обратимости в терминах линейной части. Матрица аффинного отображения. Существование и единственность аффинного отображения, действующего на систему из $n + 1$ аффинно независимыx точек $n$-мерного аффинного пространства}

Пусть $(\A, V)$ и $(\A^\prime, V^\prime)$ --- аффинные пространства над одним и тем же полем $\F$.

\begin{definition}
    \textit{Аффинным отображением} пространства $(\A, V)$ в пространство $(\A^\prime, V^\prime)$ называется всякое отображение $f: \A \to \A^\prime$, обладающее свойством
    \[
        f(p + x) = f(p) + \varphi(x),\quad\forall p \in \A, x \in V,
    \]
    где $\varphi$ --- некоторое линейное отображение пространства $V$ в пространство $V^\prime$.
\end{definition}

Из определения вытекает, что $\varphi(\overline{pq}) = \overline{f(p)f(q)}$ $\forall p, q \in \A$. Тем самым, линейное отображение $\varphi$ однозначно задаётся по $f$. Оно называется \textit{дифференциалом} $f$ и обозначается через $df$.

Введём на $(\A, V)$, приняв за начало отсчёта какие-то точки $o \in \A$ и $o^\prime \in \A^\prime$. Полагая в выражении из определения аффинного отображения $p = o$, мы получаем следующее представление аффинного отображения $f$ в векторизованной форме:
\[
    f(x) = \varphi(x) + b,\quad b \in V^\prime,
\]
где $b \vcentcolon = \overline{o^\prime f(o)}$.

Обратно, как легко проверить, для любого линейного отображения $\varphi: V \to V^\prime$ и любого вектора $b \in V^\prime$ отображение $f(x) \vcentcolon = \varphi(x) + b$ аффинно и его дифференциал равен $\varphi$.

Отсюда, в свою очередь, получается запись отображения $f$ в координатах:
\[
    y^i = a^i_jx^j + b^i\quad(i = 1, \ldots, m),
\]
где $x^1, \ldots, x^n$ --- координаты точки $x$, а $y^1, \ldots, y^m$ --- координаты точки $y \vcentcolon = f(x)$.

\begin{definition}
    \textit{Матрицей аффинного отображения} называется матрица $A = (a^i_j)$ из последней записи.
\end{definition}

Пусть $(\A^{\prime\prime}, V^{\prime\prime})$ --- ещё одно аффинное пространство, а $g: \A^\prime \to \A^{\prime\prime}$ --- аффинное отображение.

\begin{proposal}
    Отображение $gf: \A \to \A^{\prime\prime}$ является аффинным, причём
    \[
        d(gf) = dg \cdot df.
    \]
\end{proposal}

\begin{proof}
    При $p \in \A$ и $x \in V$ имеем:
    \[
        (gf)(p + x) = g(f(p + x)) = g(f(p) + df(x)) = g(f(p)) + dg(df(x)) = (gf)(p) + (dg \cdot df)(x).
    \]
\end{proof}

\begin{remark}
    При $\F = \R$ дифференциал аффинного отображения есть частный случай дифференциала произвольного гладкого отображения, рассматриваемого в анализе, а последнее предложение даёт частный случай формулы для дифференциала сложной функции.
\end{remark}

\begin{proposal}
    Аффинное отображение биективно тогда и только тогда, когда его дифференциал биективен.
\end{proposal}

\begin{proof}
    Выберем сначала точки отсчёта $o$ и $o^\prime$ в $\A$ и $\A^\prime$ соответственно таким образом, чтобы $f(o) = o^\prime$. Тогда отображение $f$ в векторизованной форме будет совпадать со своим же дифференциалом, откуда и следует доказываемое утверждение.
\end{proof}

\begin{definition}
    Биективное аффинное отображение называется \textit{изоморфизмом} аффинных пространств. Аффинные пространства, между которыми существует изоморфизм, называются \textit{изоморфными}.
\end{definition}

\begin{corollary}
    Конечномерные аффинные пространства (над одним и тем же полем) изоморфны тогда и только тогда, когда они имеют одинаковую размерность.
\end{corollary}

Очевидно, что при аффинном отображении $f: \A \to \A^\prime$ всякая плоскость $P = p + W$ пространства $(\A, V)$ переходит в плоскость $f(P) = f(p) + df(W)$ пространства $(\A^\prime, V^\prime)$. Если $f$ биективно, то $\dim f(P) = \dim P$.

\begin{theorem}
    Пусть $\{p_0, p_1, \ldots, p_n\}$ и $\{q_0, q_1, \ldots, q_n\}$ --- две системы аффинно-независимых точек в $n$-мерном аффинном пространстве $(\A, V)$. Тогда существует единственное аффинное отображение $f: \A \to \A$, переводящее $p_i$ в $q_i$ $\forall i = 0, 1, \ldots, n$.
\end{theorem}

\begin{proof}
    Существует единственное линейное отображение $\varphi: V \to V$, переводящее базис $\overline{p_0p_1}, \ldots, \overline{p_0p_n}$ в базис $\overline{q_0q_1}, \ldots, \overline{q_0q_n}$. Векторизуем пространство $\A$, приняв за начало отсчёта точку $p_0$. Тогда искомое аффинное преобразование $f$ записывается в виде
    \[
        f(x) \vcentcolon = \varphi(x) + \overline{p_0q_0}.
    \]
\end{proof}

