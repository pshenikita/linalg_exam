\section{Аффинное отображение и его линейная часть. Композиция аффинных отображений. Критерий обратимости в терминах линейной части. Матрица аффинного отображения. Существование и единственность аффинного отображения, действующего на систему из $n + 1$ аффинно независимыx точек $n$-мерного аффинного пространства}

\renewcommand{\A}{\mathfrak{A}}
\renewcommand{\B}{\mathfrak{B}}

\begin{definition}
    Пусть даны два аффинных пространства $(\A, V)$ и $(\B, W)$. Отображение $\Phi: (\A, V) \to (\B, W)$ называется \textit{аффинным} (\textit{аффинно-линейным}), если существует линейное отображение $\varphi: V \to W$ такое, что $\forall a \in \A$ и $v \in V$ выполняется $\Phi(a + v) = \Phi(a) + \varphi(v)$.

    Отображение $\varphi: V \to W$ называется \textit{дифференциалом} отображения $\Phi$.
\end{definition}

\renewcommand{\A}{\mathcal{A}}
\renewcommand{\B}{\mathcal{B}}

