\section{Квадрики в евклидовом (точечном) пространстве. Ортогональная классификация квадрик. Аффинная классификация.}

Нецилиндрические квадрики можно разбить на три типа.

\begin{enumerate}
    \item \textbf{Неконические центральные квадрики}. Выбрав начало отсчёта в центре квадрики и умножив её уравнение на подходящее число, мы приведём его к виду
        \[
            q(x_1, \ldots, x_n) = 1,
        \]
        где $q$ --- невырожденая квадратичная функция.
    \item \textbf{Конические квадрики}. Выбрав начало отсчёта в вершине квадрики, мы приведём её уравнение к виду
        \[
            q(x_1, \ldots, x_n) = 0,
        \]
        где $q$ --- невырожденная квадратичная функция. При этом у нас ещё остаётся возможность умножить уравнение на любое число $\lambda \ne 0$.
    \item \textbf{Нецентральные квадрики}. Т.\,к. $\Ker \widehat{q} \cap \Ker l = \{\bs{0}\}$, но $\Ker \widehat{q} \ne \{\bs{0}\}$ (иначе квадрика была бы центральной), то $\dim \Ker \widehat{q} = 1$ и
        \[
            V = \Ker l \oplus \Ker \widehat{q}.
        \]
        Выбрав начало отсчёта на квадрике и базис пространства $V$, согласованный с последним разложением, мы приведём уравнение квадрики к виду
        \[
            u(x_1, \ldots, x_{n - 1}) = x_n,
        \]
        где $u = q\big|_{\Ker l}$ --- невырожденная квадратичная функция от $n - 1$ переменных. при этом остаётся возможность умножить уравнение на любое число $\lambda \ne 0$, одновременно разделив на $\lambda$ последний базисный вектор.
\end{enumerate}

Возможности дальнейшего упрощения уравнения квадрики за счёт выбора подходящего базиса в пространстве $V$ зависит от поля $\F$. При $\F = \C$ или $\R$ мы можем привести квадратичную функцию $q$ к нормальному виду.

Рассмотрим более подробно случай $\F = \R$. В этом случае уравнение невырожденной квадрики может быть приведено к одному и только одному из следующих видов:
\begin{enumerate}
    \item \textbf{Неконические центральные квадрики}:
        \[
            x_1^2 + \ldots + x_k^2 - x_{k + 1}^2 - \ldots - x_n^2 = 1\quad (0 < k \leqslant n).
        \]
    \item \textbf{Конические квадрики}:
        \[
            x_1^2 + \ldots + x_k^2 - x_{k + 1}^2 - \ldots - x_n^2 = 0\quad\br{\frac{n}{2} \leqslant k < n}.
        \]
    \item \textbf{Нецентральные квадрики}:
        \[
            x_1^2 + \ldots + x_k^2 - x_{k + 1}^2 - \ldots - x_{n - 1}^2 = x_n\quad\br{\frac{n - 1}{2} \leqslant k < n}.
        \]
\end{enumerate}

Полученный результат можно интерпретировать как классификацию вещественных квадрик с точностью до аффинных преобразований. В самом деле, если квадрики $\Gamma_1$ и $\Gamma_2$ задаются одним и тем же уравнением в аффинных системах координат, связанных с реперами $(o; e_1, \ldots, e_n)$ и $(o^\prime; e_1^\prime, \ldots, e_n^\prime)$ соответственно, то $\Gamma_1$ переводится в $\Gamma_2$ аффинным преобразованием, переводящим репер $(o; e_1, \ldots, e_n)$ в репер $(o^\prime; e_1^\prime, \ldots, e_n^\prime)$. Обратно, если квадрика $\Gamma_1$ переводится в квадрику $\Gamma_2$ аффинным преобразованием $f$, то $\Gamma_1$ и $\Gamma_2$ задаются одним и тем же уравнением в аффинных системах координат, связанных с реперами $(o; e_1, \ldots, e_n)$ и $(f(o); df(e_1), \ldots, df(e_n))$ соответственно.

Посмотрим теперь, к какому виду можно привести уравнение квадрики в евклидовом пространстве, если ограничиться прямоугольными системами координат. Как и в аффинной геометрии, задача сводится к случаю невырожденных квадрик. Рассмотрим, как и выше, три типа таких квадрик.

\begin{definition}
    В произвольной размерности квадрики типа 1 при $k = n$ называются \textit{эллипсоидами, а при $k < n$} --- гиперболоидами; квадрики типа 2 называются \textit{квадратичными конусами}; квадрики типа 3 при $k = n - 1$ называются \textit{эллиптическими параболоидами}, а при $k < n - 1$ --- \textit{гиперболическими параболоидами}.
\end{definition}

\begin{definition}
    С каждым параболоидом $\Gamma = \Gamma(Q)$ каноническим образом связано одномерное подпространство $\Ker \widehat{q} \subset V$, называемое \textit{особым направлением} параболоида $\Gamma$.
\end{definition}

Т.\,к. $\Ker \widehat{q} \not\subset \Ker l$ при любом выборе начала отсчёта, то при $x \in \Ker q$ уравнение
\[
    Q(tx) = t^2q(x) + tl(x) + c = 0
\]
имеет ровно одно решение. Следовательно, любая прямая особого направления пересекает параболоид ровно в одной точке. Более того, это пересечение (по той же причине) трансверсально, ведь касание --- это на самом деле кратный корень.

\begin{enumerate}
    \item \textbf{Неконические центральные квадрики}. Из теоремы о приведении квадратичной функции к главным осям следует, что уравнение такой квадрики в прямоугольной системе координат может быть приведено к виду
        \[
            \lambda_1x_1^2 + \ldots + \lambda_nx_n^2 = 1\quad(\lambda_1, \ldots, \lambda_n \ne 0).
        \]
        Числа $\lambda_1, \ldots, \lambda_n$ определены однозначно с точностью до перестановки.
    \item \textbf{Конические квадрики}. Уравнение такой квадрики в прямоугольной системе координат может быть приведено к виду
        \[
            \lambda_1x_1^2 + \ldots + \lambda_nx_n^2 = 0\quad(\lambda_1, \ldots, \lambda_n \ne 0).
        \]
        Числа $\lambda_1, \ldots, \lambda_n$ определены однозначно с точностью до перестановки и одновременного умножения на $\lambda \ne 0$.
    \item \textbf{Нецентральные квадрики (параболоиды)}. Выбрав начало отсчёта произвольно и приведя квадратичную функцию $q$ к главным осям, мы получим прямоугольную систему координат, в котором уравнение параболоида будет иметь вид
        \[
            \lambda_1x_1^2 + \ldots + \lambda_{n - 1}x_{n - 1}^2 + b_1x_1 + \ldots + b_nx_n + c = 0\quad(\lambda_1, \ldots, \lambda_n, b_n \ne 0).
        \]

        За счёт переноса начала отсчёта по координатам $x_1, \ldots, x_{n - 1}$ можно убрать линейные члены, содержащие эти координаты. При этом, вообще говоря, изменится свободный член. После этого за счёт переноса начала отсчёта по координате $x_n$ можно убрать свободный член. Наконец, умножив уравнение на подходящее число, можно привести его к виду
        \[
            \lambda_1x_1^2 + \ldots + \lambda_{n - 1}x_{n - 1}^2 = x_n\quad(\lambda_1, \ldots, \lambda_{n - 1} \ne 0).
        \]
        
        Покажем, что начало отсчёта, при котором уравнение параболоида приводится к такому виду, определено однозначно. Для этого охарактеризуем его в инвариантных терминах.

        Пусть $(o; e_1, \ldots, e_n)$ --- репер, в котором уравнение параболоида имеет такой вид. Тогда особое направление этого парабобоида есть $\langle e_n\rangle$, а его касательная гиперплоскость в точке $o$ задаётся уравнением $x_n = 0$. Следовательно, если базис $(e_1, \ldots, e_n)$ ортонормированный, то касательная гиперплоскость параболоида в точке $o$ ортогональна особому направлению. Такая точка называется \textit{вершиной} параболоида (раньше мы называли вершиной другое, это определение вводится только для параболоида), а проходящая через неё прямая особого направления --- \textit{осью} параболоида.
        \begin{proposal}
            Всякий параболоид в евклидовом пространстве имеет единственную вершину.
        \end{proposal}

        \begin{proof}
            Пусть $p$ --- точка параболоида с координатами $x_1, \ldots, x_n$. Дифференцируя каноническое уравнение параболоида, находим, что координаты нормального вектора параболоида в точке $p$ суть
            \[
                (2\lambda_1x_1, \ldots, 2\lambda_{n - 1}x_{n - 1}, -1).
            \]
            Для того, чтобы точка $p$ была вершиной параболоида, необходимо и достаточно, чтобы этот вектор был пропорционален $e_n$, а это имеет место тогда и только тогда, когда $x_1 = \ldots = x_{n - 1} = 0$, т.\,е. $p = o$.
        \end{proof}

        \begin{corollary}
            Коэффициенты $\lambda_1, \ldots, \lambda_{n - 1}$ в каноническом уравнении параболоида определены однозначно с точностью до перестановки и одновременного умножения на $-1$.
        \end{corollary}

        \begin{proof}
            Как мы показали, начало отсчёта, при котором уравнение параболоида приводится к требуемому виду, определено однозначно. Вектор $e_n$ как единичный вектор особого направления определён однозначно с точностью до умножения на $-1$, приводящего к умножению на $-1$ левой части нашего уравнения. Если вектор $e_n$ фиксирован, то мы уже не можем умножить уравнение на число $\lambda \ne 1$, не изменив его правой части; но тогда числа $\lambda_1, \ldots, \lambda_{n - 1}$ определены однозначно с точностью до перестановки как собственные значения симметрического оператора, соответствующего квадратичной функции $q$.
        \end{proof}
\end{enumerate}

Аналогично тому, как это было сделано применительно к аффинной классификации квадрик, полученные результаты можно интерпретировать как классификацию квадрик в аффинном евклидом пространстве с точностью до движений.

