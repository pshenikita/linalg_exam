\section{Линейные операторы. Изменение матрицы линеного оператора при переходе к новому базису. Подобные матрицы}

\begin{definition}
    Линейное отображение $\A: V \to V$ пространства $V$ в себя называется \textit{линейным оператором}.
\end{definition}

Базис в $V$ выбирается только один, поэтому изменение матрицы линейного оператора при переходе к новому базису выглядит так же, как в теореме 1 из 9 вопроса при $C = D$.

\begin{definition}
    Матрицы $A$ и $B$ называются \textit{подобными} тогда и только тогда, когда существует невырожденная матрица $C$ такая, что
    \[
        B = C^{-1}AC.
    \]
\end{definition}

\begin{proposal}
    Если $A$ и $B$ подобны, то $\det A = \det B$, $\rk A = \rk B$, $\tr A = \tr B$.
\end{proposal}

\begin{proof}
    В обозначениях из определения подобия $\det B = \det(C^{-1}AC) = \det(C^{-1}C) \cdot \det A = \det A$. Второе утверждение следует из того, что ранг не меняется при домножении на невырожденную матрицу.

    Непосредственным вычислением доказывается, что $\tr(AC) = \tr(CA)$ (теорема из первого семестра). Как следствие, $\tr(C^{-1}AC) = \tr A$.
\end{proof}

Это значит, что определитель, ранг и след матрицы линейного оператора не меняются при любой замене базиса, таким образом, можно говорить об \textit{определителе}, \textit{ранге} и \textit{следе линейного оператора}.

