\section{Аффинное подпространство (плоскость), его размерность. Геометрический смысл множества решений неоднородной системы линейных уравнений. Задание плоскости системой линейных уравнений}

\begin{definition}
    \textit{Плоскостью} в аффинном пространстве $(\A, V)$ называется подмножество вида $p_0 + W$, где $p_0 \in \A$, а $W \subseteq V$ --- подпространство. Подпространство $W$ часть называют \textit{направляющим} или \textit{касательным} к плоскости $p_0 + W$. \textit{Размерностью} плоскости $p_0 + W$ называют размерность её касательного пространства $\dim W$.
\end{definition}

Нульмерная плоскость есть точка. Одномерная плоскость называется \textit{прямой}. Плоскость размерности $n - 1$ называется \textit{гиперплоскостью}.

Очевидно, плоскость $p_0 + W$ является аффинным пространством относительно тех же операций, что и $(\A, V)$, причём верно и обратное.

\begin{proposal}
    Если $\B \subseteq \A$ и $W \subseteq V$ --- подпространство, то аффинное пространство $(\B, W)$ является плоскостью.
\end{proposal}

\begin{proof}
    Возьмём произвольную точку $p \in \B$ и докажем, что множество $p + W$ совпадает с $\B$. Из того, что $(\B, W)$ --- аффинное пространство, $\forall q \in \B$ существует $w \in W$ такой, что $q = p + w$. Значит, $\B \subseteq (p + W)$. Теперь пусть $w \in W$. Из того, что $(\B, W)$ --- аффинное пространство, $p + w \in \B$. Значит, $\B \supseteq (p + W)$. Таким образом, $(p + W) = \B$, и $(\B, W)$ есть плоскость $p + W$.
\end{proof}

\begin{proposal}
    Пусть $(\B, W) = p + W$ --- плоскость и $q \in \B$. Тогда $(\B, W) = q + W$.
\end{proposal}

\begin{proof}
    Докажем, что множества $p + W$ и $q + W$ совпадают. Пусть $p_1 \in (p + W)$. Тогда найдётся вектор $w \in W$ такой, что $p_1 = p + w$. Так вот, $p_1 = p + w = q + (\overline{qp} + w) \in (q + W)$. Обратное включение доказывается аналогично.
\end{proof}

\begin{definition}
    Для любого подмножества $\B \subseteq \A$ и любой точки $p_0 \in \B$ плоскость
    \[
        p_0 + \langle\overline{p_0p} : p \in \B\rangle
    \]
    называется \textit{аффинной оболочкой множества $\B$} и обозначается через $\aff\B$.
\end{definition}

\begin{proposal}
    Если $p + W$ --- плоскость, содержащая все точки $\B \subseteq \A$, то $(p + W) \supseteq \aff\B$.
\end{proposal}

\begin{proof}
    Фиксируем точку $q \in \B$. В касательном пространстве $\aff\B$ выберем базис $\overline{qq_1}, \ldots, \overline{qq_k}$. Тогда если $q^\prime \in \aff\B$, то $q^\prime = q + \lambda_1\overline{qq_1} + \ldots + \lambda_k\overline{qq_k}$ для некоторых $\lambda_i \in \F$, причём все векторы $\overline{qq_i}$ содержатся в $W$, т.\,к. $q \in W$ и $q_i \in W$ по условию. Значит, $q^\prime \in W$ ($i = 1, \ldots, k$).
\end{proof}

Хронологически вот здесь появляется определение аффинно независимых точек из предыдудещего вопроса.

\begin{theorem}
    Через любые $k + 1$ точек аффинного пространства проходит плоскость размерности не больше $k$; при этом, если эти точки аффинно независимы, то через них проходит единственная плоскость размерности $k$.
\end{theorem}

\begin{proof}
    Пусть $(\B, W)$ --- плоскость и $p_0, p_1, \ldots, p_k \in \B$. Тогда $\aff\{p_0, p_1, \ldots, p_k\}$ есть плоскость размерности не больше $k$, проходящая через $p_0, p_1, \ldots, p_k$. Если $\dim\aff\{p_0, p_1, \ldots, p_k\} = k$ (точки аффинно независимые), то векторы $\overline{p_0p_1}, \ldots, \overline{p_0p_k}$ линейно независимы в касательном пространстве аффинной оболочки и эта аффинная оболочка является единственной $k$-мерной плоскостью, которая содержит наши точки.
\end{proof}

Выше уже было про то, что множество решений неоднородной СЛУ есть аффинное пространство. Из этого следует, что это плоскость в аффинном пространстве $\mathbb{A}^n$. Оказывается, верно и обратное.

\begin{theorem}
    Всякая плоскость есть множество решений некоторой системы линейных уравнений.
\end{theorem}

\begin{proof}
    Пусть $p_0 + W$ --- некоторая плоскость. Подпространство $W$ может быть задано системой однородных линейных уравнений. Заменив свободные члены этих уравнений значениями, принимаемыми левыми частями в точке $p_0$, мы получим систему линейных уравнений, задающую нашу плоскость.
\end{proof}

Здесь я приведу решения двух очень похожих друг на друга задач. Решение первой есть в Винберге, оно почти дословно переписано оттуда, вторую решал я сам, так что решения отличаются.

\begin{problem}[Из Винберга]
    Если в поле $\F$ более двух элементов, то непустое подмножество $P \subseteq \A$ точек аффинного пространства $(\A, V)$ (над полем $\F$) является плоскостью тогда и только тогда, когда вместе с любыми двумя различными точками оно содержит проходящую через них прямую.
\end{problem}

\begin{solution}
    Необходимость очевидна, докажем достаточность. Пусть $P \subseteq \A$ --- непустое подмножество, обладающее указанным свойством. Фиксируем произвольную точку $p_0 \in P$ и рассмотрим подмножество $W \vcentcolon = \{v \in V : p_0 + v \in P\} \subseteq V$. Нам нужно доказать, что $W$ --- подпространство. Ясно, что оно содержит $\bs{0}$. Далее, если $w \in W$ --- любой ненулевой вектор и $\lambda \in \F$, то точка $p_0 + \lambda w$ лежит на прямой, прохдящий через $p_0$ и $p_0 + w$, следовательно, $\lambda w \in U$. Докажем наконец, что если $w_1, w_2 \in W$ --- непропорциональные векторы, то $w_1 + w_2 \in W$. Пусть $\lambda \in \F \setminus \{0, 1\}$. Легко видеть, что точка $p = p_0 + w_1 + w_2$ лежит на прямой проходящей через точки $p_1 = p_0 + \lambda w_1 \in P$ и $p_2 = p_0 + \frac{\lambda}{\lambda - 1}w_2 \in P$, а именно
    \[
        p = \frac{1}{\lambda}p_1 + \frac{\lambda - 1}{\lambda}p_2.
    \]

    Следовательно, $p \in P$ и $w_1 + w_2 \in W$.
\end{solution}

\begin{problem}[А.\,А. Клячко]
    Если $\operatorname{char}\F \ne 2$, то непустое подмножество $P \subseteq \A$ точек аффинного пространства $(\A, V)$ (над полем $\F$) является плоскостью тогда и только тогда, когда вместе с любыми двумя различными точками оно содержит проходящую через них прямую.
\end{problem}

\begin{solution}
    Необходимость очевидна, докажем достаточность. Пусть $P \subseteq \A$ --- непустое подмножество, обладающее указанным свойством. Фиксируем произвольную точку $p_0 \in P$ и рассмотрим подмножество $W \vcentcolon = \{v \in V : p_0 + v \in P\} \subseteq V$. Нам нужно доказать, что $W$ --- подпространство. Ясно, что оно содержит $\bs{0}$. Далее, если $x, y \in W$, то $\aff(p + x, p + y) \subseteq P$ по условию, так что $\forall \lambda \in \F$
    \[
        (1 - \lambda)(p + x) + \lambda(p + y) = p + (1 - \lambda)x + \lambda y = p + x + \lambda(y - x) \in P.
    \]

    Подставляя $x = \bs{0}$, получаем $p + \lambda y \in P$, значит, $\lambda y \in W$. А т.\,к. $\operatorname{char}\F \ne 2$, то можем подставить $\lambda = \frac{1}{2}$, при этом получим $x + y \in P$.
\end{solution}

Здесь я воспользовался тем, что прямая через две точки есть множество барицентрических комбинаций этих двух точек. Это достаточно очевидно, однако верен и более общий факт.

\begin{problem}[Из Винберга]
    Аффинная оболочка множества $\B \subseteq \A$ аффинного пространства $(\A, V)$ есть совокупность всех барицентрических комбинаций точек из $\B$.
\end{problem}

\begin{solution}
    Если $\B = \varnothing$, доказывать нечего. Иначе можно фиксировать точку $o \in \B$. Обозначим через $\widetilde{\B}$ совокупность барицентрических комбинаций точек из $\B$, а через $W$ --- касательное подпространство к плоскости $\aff\B$.

    Пусть $p \in \widetilde{\B}$, т.\,е. $\ds\overline{op} = \sum_{q \in \B}\lambda_i\overline{oq}$. Все векторы $\overline{oq}$ лежат в $W$, а значит, и $\overline{op} \in W$. Из $o \in \B \subseteq \aff\B$ следует $p \in \aff\B$, таким образом $\widetilde{\B} \subseteq \aff\B$.

    Обратно, пусть $p \in \aff\B$. Фиксируем базис $e_1, \ldots, e_n$ в $W$, тогда $\overline{op} = \sum\limits_{i = 1}^n\lambda_ie_i$, причём можно считать, что $\sum\limits_{i = 1}^n\lambda_i \ne 0$. т.\,к. тогда можно заменить какой-нибудь базисный вектор на пропорциональный ему и сумма изменится. В этом случае заменим $\lambda_i$ на $\ds\frac{\lambda_i}{\sum\limits_i\lambda_i}$, а базисные векторы <<удлиним>> на $\sum\limits_i\lambda_i$, получим барицентрическую комбинацию.
\end{solution}

