\section{Аффинные преобразования. Разложение биективного аффинного преобразования в
композицию параллельного переноса и преобразования с неподвижной точкой}

\begin{definition}
    \textit{Аффинное отображение} аффинного пространства $(\A, V)$ в себя называется \textit{аффинным преобразованием}.
\end{definition}

Биективные аффинные преобразования образуют группу, называемую \textit{полной аффинной группой} пространства $(\A, V)$ и обозначаемую через $\mathrm{GA}(\A)$. В силу предложения 1 из вопроса 50 отображение
\[
    d: \mathrm{GA}(\A) \to \mathrm{GL}(V)
\]
является гомоморфизмом групп. Его ядро есть \textit{группа параллельных переносов}
\[
    t_a: p \mapsto p + a\quad(a \in V).
\]
Обозначим её через $\mathrm{Trans}(V)$.

\begin{proposal}
    Для любых $f \in \mathrm{GA}(\A)$ и $a \in V$ имеем
    \[
        ft_af^{-1} = t_{df(a)}.
    \]
\end{proposal}

\begin{proof}
    Применяя преобразование $ft_af^{-1}$ к точке $q = f(p)$, получаем
    \[
        ft_af^{-1}(q) = ft_a(p) = f(p + a) = f(p) + df(a) = q + df(a).
    \]
\end{proof}

Если фиксировано начало отсчёта $o \in \A$ и тем самым аффинное пространство $(\A, V)$ отождествлено с векторным пространством $V$, то группа $\mathrm{GL}(V)$ становится подгруппой группы $\mathrm{GA}(\A)$. Это не что иное, как стабилизатор точки $o$ в группе $\mathrm{GA}(\A)$. Из записи аффинных преобразований в векторизованной форме
\[
    f(x) = \varphi(x) + b
\]
следует, что всякое аффинное преобразование $f \in \mathrm{GA}(\A)$ единственным образом представляется в виде
\[
    f = t_b\varphi,\quad \varphi \in \mathrm{GL}(V), b \in V.
\]

Ясно, что $\varphi = df$ не зависит от выбора начала отсчёта, а вот вектор $b = \overline{of(o)}$ от этого, вообще говоря, зависит. Вспомнив, что $\varphi(\bs{0}) = \bs{0}$, мы фактически доказали следующее утверждение.

\begin{theorem}
    Любое биективное аффинное преобразование представимо в виде композиции параллельного переноса и преобразования с неподвижной точкой.
\end{theorem}

