\section{Нормальный (канонический) вид квадратичной формы над полями действительных и комплексных чисел. Закон инерции}

Над полем $\R$ квадратичную форму можно далее упростить:

\begin{proposal}
    Для любой симметрической билинейной функции $\B$ в пространстве над полем $\R$ существует базис, в котором её матрица имеет диагональный вид с $1$, $-1$ и $0$ на диагонали.
\end{proposal}

\begin{proof}
    Сначала с помощью теоремы 1 приведём квадратичную форму к виду
    \[
        Q(u) = r_{11}(u^1)^2 + \ldots + r_{nn}(u^n)^2.
    \]

    Если $r_{ii} > 0$, то замена $y^i = \sqrt{r_{ii}}u^i$ приводит слагаемое $r_{ii}(u^i)^2$ к виду $(y^i)^2$. Если же $r_{ii} < 0$, то замена $y^i = \sqrt{-r_{ii}}u^i$ приводит слагаемое $r_{ii}(u^i)^2$ к виду $-(y^i)^2$. В результате получаем требуемый вид квадратичной формы с коэффициентами $1$, $-1$ и $0$.
\end{proof}

\begin{definition}
    Вид, описанный в последнем предложении, называется \textit{нормальным видом} вещественной симметрической билинейной формы (вещественной квадратичной формы).
\end{definition}

Над полем $\C$ квадратичную форму можно ещё больше упростить:

\begin{proposal}
    Для любой симметрической билинейной функции $\B$ над полем $\C$ существует базис, в котором её матрица имеет диагональный вид с $1$ и $0$ на диагонали.
\end{proposal}

\begin{proof}
    Сначала мы с помощью последнего предложения приведём квадратичную форму к виду $(y^1)^2 + \ldots + (y^p)^2 - (y^{p + 1})^2 - \ldots - (y^{p + q})^2$. Затем сделаем замену координат $y^k = z^k$ при $k \leqslant p$ и $y^k = iz^k$ при $k > p$. В результате получим требуемый вид.
\end{proof}

\begin{definition}
    Вид, описанный в последнем предложении, называется \textit{нормальным видом} комплексной симметрической билинейной формы (комплексной квадратичной формы).
\end{definition}

В случае симметрической билинейной формы над $\C$ нормальный вид зависит только от её ранга, и поэтому мы получаем:

\begin{proposal}
    Две комплексные симметрические билинейные формы (комплексные квадратичные формы) получаются друг из друга линейной заменой координат только и только тогда, когда их ранги совпадают.
\end{proposal}

В случае симметрических билинейных форм над $\R$ ситуация сложнее: их нормальный вид не определяется одним лишь рангом, а зависит ещё от количества $1$ и $-1$ на диагонали матрицы. Оказывается, что нормальный вид такой формы не зависит от способа приведения к нормальному виду.

\begin{theorem}[Закон инерции]
    Количество $1$, $-1$ и $0$ на диагонали нормального вида матрицы вещественной симметрической билинейной функции $\B$ не зависит от способа приведения к нормальному виду.
    Другими словами, если квадратичная форма $Q(x)$ вещественной линейной заменой $x = Cy$ приводится к виду
    \[
        (y^1)^2 + \ldots + (y^p)^2 - (y^{p + 1})^2 - \ldots - (y^{p + q})^2,
    \]
    а вещественной линейной заменой $x = C^\prime z$ --- к виду
    \[
        (z^1)^2 + \ldots + (z^{p^\prime}) - (z^{p^\prime + 1})^2 - \ldots - (z^{p^\prime + q^\prime})^2,
    \]
    то мы имеем $p = p^\prime$ и $q = q^\prime$.
\end{theorem}

\begin{proof}
    Пусть $(x^1, \ldots, x^n)$ --- координаты в исходном базисе $e_1, \ldots, e_n$ пространства $V$, $(y^1, \ldots, y^n)$ --- координаты в базисе $f_1, \ldots, f_n$, а $(z^1, \ldots, z^n)$ --- координаты в базисе $g_1, \ldots, g_n$. Рассмотрим подпространства
    \[
        \begin{array}{l l l}
            U_+ \vcentcolon = \langle f_1, \ldots, f_p \rangle, & U_- \vcentcolon = \langle f_{p + 1}, \ldots, f_{p + q} \rangle, & U_0 \vcentcolon = \langle f_{p + q + 1}, \ldots, f_n \rangle,\\
            W_+ \vcentcolon = \langle g_1, \ldots, g_{p^\prime} \rangle, & W_- \vcentcolon = \langle g_{p^\prime + 1}, \ldots, g_{p^\prime + q^\prime} \rangle, & W_0 \vcentcolon = \langle g_{p^\prime + q^\prime + 1}, \ldots, g_n \rangle
        \end{array}
    \]

    Для ненулевого вектора $x \in U_+$ мы имеем $x = y^1f_1 + \ldots + y^pf_p$, поэтому $\B(x, x) = (y^1)^2 + \ldots + (y^p)^2 > 0$. Аналогично, если $x \in U_- \oplus U_0$, то $\B(x, x) \leqslant 0$. Для ненулевого вектора $x \in W_+$ мы имеем $\B(x, x) > 0$, а для $x \in W_- \oplus W_0$ имеем $\B(x, x) \leqslant 0$. Предположим, что $p > p^\prime$. Тогда
    \[
        \dim U_+ + \dim(W_- \oplus W_0) = p + (n - p^\prime) > n = \dim V,
    \]
    значит, $U_+ \cap (W_- \oplus W_0) \ne \{\bs{0}\}$. Возьмём ненулевой вектор $x$ в этом пересечении. Т.\,к. $x \in U_+$, имеем $\B(x, x) > 0$. С другой стороны, $x \in W_- \oplus W_0$ следует, что $\B(x, x) \leqslant 0$ Противоречие. Аналогично приводится к противоречию случай $p < p^\prime$.

    Следовательно, $p = p^\prime$. Кроме того, $p + q = p^\prime + q^\prime = \rk\B$, а значит, и $q = q^\prime$.
\end{proof}

\begin{remark}
    Важно понимать, что человечество на самом деле не умеет приводить квадратичные формы к какому-то адекватному виду. Мы хоть что-то знаем только про очень узкие ситуации --- симметричная (кососимметричная) матрица, только над полями $\R$ или $\C$ и т.\,п. Даже над полем $\Q$ понять, являются две квадратичные формы эквивалентными, сложно. Есть инвариант в виде ранга, есть замечание, что $\det A^\prime = \det (C^tAC) = (\det C)^2\det A$ (т.\,е. отношение определителей должно быть квадратом элемента поля). Но вот примерно на этом какие-то нормальные соображения заканчиваются. Ходят слухи, что в НМУ на <<Алгебре 2>> в 2024 году учили что-то понимать про поля типа $\Z_p$.
\end{remark}

\begin{definition}
    Число $p$ называется \textit{положительным индексом инерции}, а $q$ --- \textit{отрицательным}. А в общем, \textit{индексами инерции} называется пара $(p, q)$.
\end{definition}

\begin{definition}
    Разность $p - q$ между числом положительных и отрицательных диагональных элементов в нормальном виде называется \textit{сигнатурой} вещественной симметрической билинейной функции.
\end{definition}

Из последней теоремы следует, что сигнатура, как и ранг, являются инвариантом вещественной симметрической билинейной функции, т.\,е. не зависит от базиса.

\begin{corollary}
    Две вещественные симметрические билинейные формы получаются друг из друга линейной заменой координат тогда и только тогда, когда их ранги и сигнатуры совпадают.
\end{corollary}

