\section{Полярное разложение невырожденного линейного оператора в евклидовом пространстве. Сингулярное разложение}

\begin{definition}
    Самосопряжённый оператор $\A$ называется \textit{положительным}, если $(\A v, v) > 0$ $\forall v \in V \setminus \{\bs{0}\}$.
\end{definition}

\begin{lemma}
    Самосопряжённый оператор $\A$ положителен тогда и только тогда, когда все его собственные значения положительны. 
\end{lemma}

\begin{proof}
    $\Rightarrow$. Пусть $(\A v, v) > 0$ при $v \ne 0$. Рассмотрим собственный вектор $v$ с собственным значением $\lambda$. Тогда $\lambda(v, v) = (\A v, v) > 0$, откуда $\lambda > 0$.

    $\Leftarrow$. Обратно, пусть все собственные значения $\lambda_i$ положительны. Выберем ортонормированный базис из собственных векторов $e_1, \ldots, e_n$ с собственными значениями $\lambda_1, \ldots, \lambda_n$. Тогда для ненулевого вектора $v = v^ie_i$ мы имеем
    \[
        (\A v, v) = (\A(v^ie_i), v^je_j) = \overline{v}^iv^j(\A e_i, e_j) = \sum_{i, j}\lambda_i\overline{v^i}v^j(e_i, e_j) = \sum_{i = 1}^n\lambda_i\abs{v^i}^2 > 0.
    \]
\end{proof}

\begin{theorem}
    Для положительного оператора $\A$ существует единственный положительный оператор $\P$, удовлетворяющий соотношению $\P^2 = \A$.
\end{theorem}

\begin{proof}
    Пусть $\lambda_1, \ldots, \lambda_k$ --- различные собственые значения оператора $\A$ и $V_1, \ldots, V_k$ --- соответствующие собственные подпространства. Согласно последней лемме, все $\lambda_i$ положительны. Положим $\mu_i \vcentcolon = \sqrt{\lambda_i}$. Рассмотрим оператор $\P$, действующий в пространства $V_i$ умножением на $\mu_i$. Тогда $\P^2 = \A$. Оператор $\P$ самосопряжён (т.\,к. он задаётся диагональной матрицей в ортонормированном базисе из собственных векторов оператора $\A$) и положителен в силу последней леммы.

    Осталось доказать единственность оператора $\P$. Пусть оператор $\P$ удовлетворяет условиям теоремы. Пусть $\mu_1, \ldots, \mu_k$ --- его различные собственные значения и $W_1, \ldots, W_k$ --- соответствующие собственные подпространства. Тогда оператор $\P^2 = \A$ действует в пространстве $W_i$ умножением на $\mu_i^2$, т.\,е. $\mu_i^2$ является собственным значением для $\A$. Следовательно, при подходящей нумерации имеем $\mu_i^2 = \lambda_i$ и $W_i = V_i$. Это показывает, что оператор $\P$ определён однозначно.
\end{proof}

\begin{definition}
    Оператора $\P$, построенный в предыдущей теореме, называется \textit{положительным корнем} из положительного оператора $\A$ и обозначается $\sqrt{\A}$.
\end{definition}

\begin{definition}
    Самосопряжённый оператор $\A$ называется \textit{неотрицательным}, если $(\A v, v) \geqslant 0$ $\forall v \in V$. Все утверждения выше переносятся без изменений на неотрицательные операторы.
\end{definition}

\begin{theorem}[Полярное разложение]
    Для любого невырожденного оператора $\A$ в евклидовом или эрмитовом пространстве существует единственное представление в виде
    \[
        \A = \P\U,
    \]
    где $\P$ --- положительный, а $\U$ --- ортогональный (унитарный) оператор.
\end{theorem}

\begin{proof}
    Если $\A = \P\U$, то $\A^\ast = \U^\ast\P$ и $\A\A^\ast = \P\U\U^\ast\P = \P^2$. Оператор $\A\A\ast$ очевидно самосопряжён; кроме того, он является положительным:
    \[
        (\A\A^\ast v, v) = (\A^\ast v, \A^\ast, v) > 0
    \]
    при $v \ne 0$, т.\,к. $\A^\ast v \ne \bs{0}$ в силу невырожденности $\A$. Поэтому положительный оператор $\P$, удовлетворяющий соотношению $\A\A^\ast = \P^2$, единственен в силу последней теоремы, а именно, $\P = \sqrt{\A\A^\ast}$. Тогда оператор $\P^{-1}\A$ также определён однозначно. Мы имеем $\U\U^\ast = \P^{-1}\A\A^\ast\P^{-1} = \P^{-1}\P^2\P^{-1} = \id$. Следовательно, $\U$ --- ортогональный (унитарный) оператор и $\A = \P\U$.
\end{proof}

Аналогично, рассмотрев положительный оператор $\A^\ast\A$ можно доказать существование второго полярного разложения $\A = \U^\prime\P^\prime$ (где $\P^\prime = \sqrt{\A^\ast\A}$).

По этой теме см. теоретические задачи $21$ и $22$.

В одномерном эрмитовом пространстве $\C$ положительные операторы --- это положительыне вещественные числа, а унитарные операторы --- это комплексные числа, по модулю равные $1$, т.\,е. вида $e^{i\varphi}$. Поэтому полярное разложение --- это представление комплексного числа $z$ в полярных координатах: $z = \rho e^{i\varphi}$, что объясняет название.

\begin{theorem}
    Для любого линейного отображения $\A: V \to V$ евклидова или эрмитова пространства $V$ существуют такие ортонормированные базисы $e_1, \ldots, e_n$ и $f_1, \ldots, f_n$ и положительные числа $\sigma^1 \geqslant \sigma^2 \geqslant \ldots \geqslant \sigma^r$, где $r = \dim\Im\A = \rk\A$, что
    \[
        \A e_i =
        \begin{cases}
            \sigma^if_i,& i \leqslant r,\\
            0,& i > r.
        \end{cases}
    \]

    Числа $\sigma_1, \sigma_2, \ldots, \sigma_r$ называют \textit{сингулярными числами} оператора $\A$. А базисы $e_1, \ldots, e_n$ и $f_1, \ldots, f_n$ называют \textit{сингулярными базисами} оператора $\A$.
\end{theorem}

\begin{proof}
    Оператор $\A^\ast\A$ самосопряжён и неотрицателен:
    \[
        (\A^\ast\A)^\ast = \A^\ast\A,\quad(\A^\ast\A v, v) = (\A v, \A v) \geqslant 0
    \]
    $\forall v \in V$. Поэтому существует ортонормированный базис собственных векторов $e_1, \ldots, e_m$ матрицы $\A^\ast\A$. Все её собственные значения неотрицательны. Таким образом,
    \[
        \A^\ast\A e_i = (\sigma^i)^2e_i,\quad i = 1, \ldots, n.
    \]

    Будем считать при этом, что $\sigma^1 \geqslant \sigma^2 \geqslant \ldots \geqslant \sigma^r > 0$, $\sigma^{r + 1} = \ldots = \sigma^n = 0$. Значит,
    \[
        (\A e_i, \A e_j) = (\A^\ast\A e_i, e_j) = (\sigma^i)^2(e_i, e_j)\delta_{ij} =
        \begin{cases}
            0,& i \ne j,\\
            \sigma_i^2,& i = j
        \end{cases}
    \]

    Значит, векторы $f_i \vcentcolon = (\sigma^i)^{-1}\A e_i$ образуют ортономированный базис в $V$, причём $\A e_i = \sigma^if_i$.
\end{proof}

