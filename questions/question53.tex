\section{Ортогональные аффинные преобразования (движения). Собственные и несобственные движения. Разложение движения в композицию параллельного переноса и ортогонального преобразования с неподвижной точкой}

\begin{definition}
    \textit{Движением} (\textit{изометрией}) аффинного евклидова пространства $(\A, V)$ называется всякое его аффинное преобразование, дифференциал которого является ортогональным оператором.
\end{definition}

Во-первых, ясно, что всякое движение биективно. А во-вторых,

\begin{theorem}
    Всякое движение сохраняет расстояния между точками и обратно, всякое аффинное преобразование, сохраняющее расстояния между точками, является движением.
\end{theorem}

\begin{proof}
    $\Rightarrow$. Пусть $f$ --- движение, тогда $\forall p, q \in \A$ имеем
    \[
        \rho(f(p), f(q)) = \abs{\overline{f(p)f(q)}} = \abs{\varphi(\overline{pq})} = \abs{\overline{pq}} = \rho(p, q),
    \]
    т.\,е. $f$ сохраняет расстояния.

    $\Leftarrow$. Обратно, пусть $f$ сохраняет расстояния. Тогда выполнено равенство
    \[
        \abs{\overline{pq}} = \abs{\overline{f(p)f(q)}} = \abs{\varphi(\overline{pq})}.
    \]

    Отсюда следует, что дифференциал сохраняет длины всех векторов. Значит, он ортогонален, и $f$ --- движение.
\end{proof}

Движения аффинного евклидова пространства образуют группу, обозначаемую $\mathrm{Isom}(\A, V)$.

\begin{definition}
    Движение $f$ называется \textit{собственным}, если $\det df = 1$, иначе --- \textit{несобственным}.
\end{definition}

Собственные движения образуют подгруппу в $\mathrm{Isom}(\A, V)$, обозначаемую $\mathrm{Isom}_+(\A, V)$.

\begin{example}
    Важным примером несобственного движения является (ортогональное) \textit{отражение} $r_H$ относительно гиперплоскости $H$. Пусть $e$ --- единичный вектор, ортогональный $H$. Всякую точку $p \in \A$ можно единственным образом представить в виде $p = q + \lambda e$ ($q \in H$). По определению, $r_Hp \vcentcolon = q - \lambda e$. Дифференциал отражения $r_H$ есть (ортогональное) отражение относительно направляющего подпространства гиперплоскости $H$ в пространстве $V$. Пусть $H_1$ и $H_2$ --- две гиперплоскости. Если они параллельны, то $dr_{H_1} = dr_{H_2}$ и, следовательно,
    \[
        d(r_{H_1}r_{H_2}) = dr_{H_1} \cdot dr_{H_2} = \id.
    \]

    В этом случае $r_{H_1}r_{H_2}$ --- параллельный перенос на удвоенный общий перпендикуляр плоскостей $H_1$ и $H_2$. Если же $H_1$ и $H_2$ пересекаются по $(n - 2)$-мерной плоскости $P$, то $r_{H_1}r_{H_2}$ --- поворот на вокруг $P$ на удвоенный угол между $H_1$ и $H_2$.
\end{example}

\begin{theorem}
    Пусть $\Phi: \A \to \A$ --- движение аффинного евклидова конечномерного пространства $(\A, V)$ с дифференциалом $\varphi$. Тогда найдётся вектор $u \in V$ такой, что $\varphi(u) = u$ и $\Phi = t_u\Psi$, где $\Psi$ --- движение с неподвижной точкой.
\end{theorem}

\begin{proof}
    Пусть $a \in \A$ --- произвольная точка. Рассмотрим вектор
    \[
        v \vcentcolon = \overline{a\Phi(a)}.
    \]

    Если существует собственный вектор линейного оператора $\varphi$ с собственным значением $1$, то обозначим через $W$ собственное подпространство линейного оператора $\varphi$, отвечающее собственному значению $1$, иначе $U \vcentcolon = \{\bs{0}\}$. Т.\,к. $\varphi$ --- ортогональный оператор, а $W$ --- инвариантное подпространство относительно $\varphi$, то по важной лемме подпространство $W^\perp$ тоже является инвариантным относительно $\varphi$. Кроме того, линейный оператор $\varphi - \id$ действует на $W^\perp$ невырожденным образом. Т.\,к. $V = W \oplus W^\perp$, вектор $v$ представляется в виде $v = \pr_Wv + \ort_Wv$, причём $\varphi(\pr_Wv) = \pr_Wv$. Положим $\Psi \vcentcolon = t_u^{-1}\Phi$ (где $u \vcentcolon = \pr_Wv$) и найдём для $\Psi$ неподвижную точку в виде $b = a + w$, $w \in W^\perp$. Имеем
    \begin{multline*}
        \Psi(a + w) = (t_{-u}\Phi)(a + w) = t_{-u}(\Phi(a) + \varphi(w)) = t_{-u}((a + v) + \varphi(w)) =\\ = a + (\pr_Wv + \ort_Wv) + \varphi(w) - \pr_Wv = a + \ort_Wv + \varphi(w) = a + w + \ort_Wv + (\varphi - \id)w.
    \end{multline*}

    Хотим найти такой вектор $w \in W^\perp$, что $\ort_Wv + (\varphi - \id)w = \bs{0}$. А такой вектор существует в силу невырожденности $(\varphi - \id)\big|_{W^\perp}$ и равен $-(\varphi - \id)^{-1}\ort_Wv$.
\end{proof}

