\section{Векторное пространство линейных отображений. Алгебра линейных операторов. Изоморфизм алгебры матриц и алгебры линейных операторов}

\renewcommand{\C}{\mathcal{C}}

\begin{definition}
    Множество всех линейных отображений $\A: V \to W$ с операциями сложения и умножения на скаляры
    \[
        (\A_1 + \A_2)(v) \vcentcolon = \A_1v + \A_2v,\quad (\lambda\A)(v) \vcentcolon = \lambda(\A v)
    \]
    является линейным пространством над полем $\F$. Оно называется \textit{пространством линейных отображений} из $V$ в $W$ и обозначается $\Hom_\F(V, W)$. Мы также будем пользоваться обозначением $\End(V) \vcentcolon = \Hom_\F(V, V)$.
\end{definition}

Напомним определение алгебры.

\begin{definition}
    \textit{Алгеброй} над полем $\F$ называется множество $\mathfrak{A}$ с операциями сложения, умножения и умножения на элементы поля $\F$, обладающими следующими свойствами:
    \begin{enumerate}[nolistsep]
        \item относительно сложения и умножения на элементы поля $\mathfrak{A}$ есть векторное пространство;
        \item относительно сложения и умножения $\mathfrak{A}$ есть кольцо;
        \item $(\lambda a)b = a(\lambda b) = \lambda(ab)$ $\forall \lambda \in \F$, $\forall a, b \in \mathfrak{A}$.
    \end{enumerate}
\end{definition}

На множестве $\End(V)$ всех линейных операторов векторного пространства $V$ над полем $\F$ можно также определить произведение (композицию) линейных операторов $(\A\B)(v) \vcentcolon = \A(\B v)$. Очевидно, оно тоже линейно. При этом выполнены все аксиомы алгебры.

Алгебра $\End(V)$ является ассоциативной и обладает единицей. Единицей этой алгебры является тождественный линейный оператор, который мы будем обозначать через $\id$.

\begin{definition}
    Две алгебры $\mathfrak{A}$ и $\mathfrak{A}^\prime$ называются \textit{изоморфными}, если существует изоморфизм колец из первой алгебры во вторую, являющийся одновременно и изоморфизмом векторных пространств. Пишем $\mathfrak{A} \simeq \mathfrak{A}^\prime$.
\end{definition}

\begin{theorem}
    Для любого $n$-мерного линейного пространства $V$ над полем $\F$ выполняется $\End(V) \simeq \underset{n \times n}{\Mat}(\F)$.
\end{theorem}

\begin{proof}
    Зафиксируем базис $e_1, \ldots, e_n$ в пространстве $V$ и рассмотрим отображение $\Phi: \End(V) \to \underset{n \times n}{\Mat}(\F)$, сопоставляющее каждому линейному оператору $\A: V \to V$ его матрицу $A$ в выбранном базисе.

    Т.\,к. любой линейный оператор однозначно определяется своей матрицей (в фиксированном базисе) формулой из предложения 1 в вопросе 9, а любая квадратная матрица порядка $n$ является матрицей порядка $n$ является матрицей некоторого линейного оператора (столбцы этой матрицы можно считать координатами образов базисных векторов из выбранного базиса), отображение $\Phi$ биективно.

    Далее пусть $\Phi(\A) = A = (a^j_i)$, $\Phi(\B) = B = (b^k_j)$, $\Phi(\C) = C = (c^k_i)$, где $\C = \B\A$. Тогда с одной стороны,
    \[
        \C e_i = \B(\A e_i) = \B(a^j_ie_j) = a^j_i\B e_j = a^j_ib^k_je_k,
    \]
    а с другой ---
    \[
        \C e_i = c^k_ie_k.
    \]
    Из единственности разложения вектора по базису получаем $c^k_i = b^k_ja^j_i$; в матричной записи $C = BA$. Таким образом, $\Phi(\B\A) = \Phi(\B) \cdot \Phi(\A)$. Аналогично проверяются остальные аксиомы: $\Phi(\lambda\A) = \lambda\Phi(\A)$ и $\Phi(\A + \B) = \Phi(\A) + \Phi(\B)$.
\end{proof}

\begin{corollary}
    $\dim\End(V) = (\dim V)^2$.
\end{corollary}

\renewcommand{\C}{\mathbb{C}}

