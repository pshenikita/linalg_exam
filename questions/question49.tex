\section{Взаимное расположение двух плоскостей. Пересечение и аффинная оболочка двух
аффинных плоскостей. Размерность аффинной оболочки}

Пусть $P_1 = p_1 + W_1$ и $P_2 = p_2 + W_2$ --- плоскости.

Очевидно, что если они пересекаются и $p_0$ --- одна из точек этого пересечения, то
\[
    P_1 \cap P_2 = p_0 + (W_1 \cap W_2).
\]

\begin{theorem}
    Плоскости $P_1$ и $P_2$ пересекаются тогда и только тогда, когда
    \[
        \overline{p_1p_2} \in W_1 + W_2.
    \]
\end{theorem}

\begin{proof}
    Плоскости $P_1$ и $P_2$ пересекаются тогда и только тогда, когда существуют векторы $w_1 \in W_1$ и $w_2 \in W_2$, что
    \[
        p_1 + w_1 = p_2 + w_2.
    \]
    Это равенство может быть переписано в виде
    \[
        \overline{p_1p_2} = w_1 - w_2.
    \]
    Поэтому существование таких векторов $w_1$ и $w_2$ как раз и означает, что $\overline{p_1p_2} \in W_1 + W_2$.
\end{proof}

\begin{definition}
    Плоскости $P_1$ и $P_2$ называются \textit{параллельными}, если $W_1 \subseteq W_2$ или $W_2 \subseteq W_1$ и \textit{скрещивающимися}, если $P_1 \cap P_2 = \varnothing$ и $W_1 \cap W_2 = \{\bs{0}\}$.
\end{definition}

\begin{theorem}
    Пусть $\pi_1 = p_1 + W_1$ и $\pi_2 = p_2 + W_2$ --- плоскости. Тогда
    \[
        \aff(P_1 \cup P_2) = p_1 + \langle\overline{p_1p_2}, W_1 + W_2\rangle,
    \]
    причём если $P_1 \cap P_2 \ne \varnothing$, то
    \[
        \dim\aff(P_1 \cup P_2) = \dim(W_1 + W_2),
    \]
    иначе
    \[
        \dim\aff(P_1 \cup P_2) = \dim(W_1 + W_2) + 1.
    \]
\end{theorem}

\begin{proof}
    Обозначим $P \vcentcolon = p_1 + \langle\overline{p_1p_2}, W_1 + W_2\rangle$. Понятно, что $P_1, P_2 \subseteq P$. Обратно, т.\,к. $p_1 \in \aff(P_1 \cup P_2)$, то $\aff(P_1 \cup P_2) = p_1 + X$ для некоторого подпространства $X \subseteq V$. Поскольку $p_2 \in \aff(P_1 \cup P_2)$, имеем $\overline{p_1p_2} \in X$. Т.\,к. $\forall w_1 \in W_1$ точка $p_1 + w_1 \in P_1 \subseteq \aff(P_1 \cup P_2)$, то $w_1 \in X$. Далее, для любого $w_2 \in W_2$ имеем $p_2 + w_2 \in P_2 \subseteq \aff(P_1 \cup P_2)$ и $p_2 + w_2 = a + ((\overline{p_1p_2}) + w_2) \in \aff(P_1 \cup P_2)$, отсюда $\overline{p_1p_2} + w \in X$. Поскольку $\overline{p_1p_2} \in X$, имеем $w \in X$. Получилось, что $\langle\overline{p_1p_2}, W_1 + W_2\rangle \subseteq X$, т.\,е. $P \subseteq \aff(P_1 \cup P_2)$, а значит, $P = \aff(P_1 \cup P_2)$, что и требовалось.

    Из теоремы 1, если $P_1 \cap P_2 \ne \varnothing$, то $\overline{p_1p_2} \in W_1 + W_2$ и $\dim\aff(P_1 \cup P_2) = \dim\langle\overline{p_1p_2}, W_1 + W_2\rangle = \dim(W_1 + W_2)$. А иначе $\overline{p_1p_2} \notin W_1 + W_2$ и к $\dim(W_1 + W_2)$ нужно будет прибавить $1$.
\end{proof}

