\section{Самосопряженные (эрмитовы) операторы в унитарном пространстве, их свойства. Теорема о существовании ортонормированного базиса из собственных векторов эрмитова оператора. Приведение эрмитовой квадратичной формы к главным осям}

См. вопросы 38 и 41 (про приведение к главным осям).

\begin{theorem}[Эрмитово разложение]
    Для любого оператора $\A$ в эрмитовом пространстве существует единственное представление в виде
    \[
        \A = \mathcal{R} + i\mathcal{I},
    \]
    где $\mathcal{R}$ и $\mathcal{I}$ --- эрмитовы операторы.
\end{theorem}

\begin{proof}
    Сначала докажем единственность. Если $\A = \mathcal{R} + i\mathcal{I}$ --- эрмитово разложение, то $\A^\ast = \mathcal{R}^\ast - i\mathcal{I}^\ast = \mathcal{R} - i\mathcal{I}$. Из этих двух соотношений получаем
    \[
        \mathcal{R} = \frac{1}{2}(\A^\ast + \A),\quad\mathcal{I} = \frac{i}{2}(\A^\ast - \A),
    \]
    т.\,е. операторы $\mathcal{R}$ и $\mathcal{I}$ определены однозначно и эрмитово разложение единственно.

    С другой стороны, операторы $\mathcal{R}$ и $\mathcal{I}$, задаваемыми предыдущими формулами, очевидно, эрмитовы (самосопряжены), так что эрмитово разложение существует.
\end{proof}

В одномерном эрмитовом пространстве $\C$ эрмитовы операторы --- это вещественные числа, а операторы $\mathcal{R}$ и $\mathcal{I}$ в эрмитовом разложении --- это вещественная и мнимая части комплексного числа.

