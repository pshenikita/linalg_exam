\section{Билинейные функции и их матрицы. Изменение матрицы билинейной формы при замене базиса. Ранг билинейной функции. Симметрические билинейные функции}

\begin{definition}
    Пусть $V$ --- линейное пространство над полем $\F$. Функция $\B: V \times V \to \F$ называется \textit{билинейной функцией}, если она линейна по каждому аргументу:
    \begin{gather*}
        \B(\lambda_1x_1 + \lambda_2x_2, y) = \lambda_1\B(x_1, y) + \lambda_2\B(x_2, y),\\
        \B(x, \mu_1y_1 + \mu_2y_2) = \mu_1\B(x, y_1) + \mu_2\B(x, y_2)
    \end{gather*}
    для любых $\lambda_1, \lambda_2, \mu_1, \mu_2 \in \F$ и $x, x_1, x_2, y, y_1, y_2 \in V$.
\end{definition}

\begin{definition}
    \textit{Матрицей билинейной функции} $\B$ в базисе $e_1, \ldots, e_n$ пространства $V$ называется квадратная матрица $B = (b_{ij})$ размера $n$, где $b_{ij} = \B(e_i, e_j)$.
\end{definition}

Зная матрицу $B = (b_{ij})$ билинейной функции, можно восстановить значение $\B(x, y)$ на любой паре векторов $x = x^ie_i$ и $y = y^je_j$:
\[
    \B(x, y) = \B(x^ie_i, y^je_j) = x^iy_j\B(e_i, e_j) = b_{ij}x^iy^j = x^tBy.
\]

\begin{definition}
    Выражение $B(x, y) = b_{ij}x^iy^j$ называется \textit{билинейной формой}.
\end{definition}

Билинейная форма представляет собой однородный многочлен степени $2$ от двух наборов переменных $x^1, \ldots, x^n$ и $y^1, \ldots, y^n$, который линее по $x$ при фиксированных $y$ и линеен по $y$ при фиксированных $x$.

\begin{theorem}[Закон изменения матрицы билинейной функции]
    Имеет место соотношение
    \[
        B^t = C^tBC,
    \]
    где $B$ --- матрица билинейной функции $\B: V \times V \to \F$ в базисе $e_1, \ldots, e_n$, $B^t$ --- матрица в базисе $e_{1^\prime}, \ldots, e_{n^\prime}$ и $C$ --- матрица перехода от базиса $e_1, \ldots, e_n$ к базису $e_{1^\prime}, \ldots, e_{n^\prime}$.
\end{theorem}

\begin{proof}
    Пусть $B = (b_{ij})$, $B^t = (b^\prime_{ij})$, $C = (c^i_{i^\prime})$. Мы имеем
    \[
        b^\prime_{ij} = \B(e_{i^\prime}, e_{j^\prime}) = \B(c^i_{i^\prime}e_i, c^j_{j^\prime}e_j) = c^i_{i^\prime}c^j_{j^\prime}\B(e_i, e_j) = c^i_{i^\prime}b_{ij}c^j_{j^\prime},
    \]
    что эквивалентно матричному соотношению $B^t = C^tBC$.
\end{proof}

\begin{corollary}
    Ранг матрицы билинейной функции не зависит от базиса.
\end{corollary}

\begin{proof}
    Т.\,к. матрица $C$ обратима, $\rk B^t = \rk(C^tBC) = \rk B$.
\end{proof}

\begin{definition}
    \textit{Рангом} билинейной функции $\B$ (обозначается $\rk\B$) называется ранг её матрицы в произвольном базисе. Билинейная функция $\B$ в пространстве $V$ называется \textit{невырожденной}, если $\rk\B = \dim V$.
\end{definition}

Множество $\operatorname{B}(V)$ всех билинейных функций в пространстве $V$ образует линейное пространство относительно операций сложения функций и умножения функций на скаляры. Сопоставление билинейной функции $\B$ её матрицы $B$ в фиксированном базисе $e_1, \ldots, e_n$ устанавливает изоморфизм между пространством $\operatorname{B}(V)$ и пространством квадратных матрицы $\underset{n \times n}\Mat(\F)$. Как и в случае пространства линейных операторов $\Hom(V, V)$, этот изоморфизм неканоничен, т.\,к. он зависит от выбора базиса.

\begin{theorem}
    Отображение $\varphi: \operatorname{B}(V) \to \Hom(V, V^\ast)$, сопоставляющее билинейной функции $\B$ линейной отображение $\widetilde{\B}: V \to V^\ast$, задаваемое формулой
    \[
        \widetilde{\B}(x) \vcentcolon = \B(x, \bs{\cdot})\text{ для $x \in V$},
    \]
    является каноническим изоморфизмом. Здесь $\B(x, \bs{\cdot}) \in V^\ast$ --- линейная функция, значение которой на векторе $y \in V$ есть $\B(x, y)$.
\end{theorem}

\begin{proof}
    Отображение $\varphi$ линейно, т.\,к. билинейная функция линейна по первому аргументу $x$. Кроме того, отображение $\varphi$ биективно: обратное отображение $\varphi^{-1}$ ставит в соответствие линейному отображению $\widetilde{\B}: V \to V^\ast$ билинейную функцию $\B$, заданную по формуле $\B(x, y) = \widetilde{\B}(x)(y)$. Следовательно, $\varphi$ --- изоморфизм. Этот изоморфизм каноничен, т.\,к. в его конструкции не использовался базис.
\end{proof}

\begin{definition}
    Билинейная функция $\B: V \times V \to \F$ называется \textit{симметрической}, если $\B(y, x) = \B(x, y)$, и \textit{кососимметрической}, если $\B(x, x) = 0$ для любых $x, y \in V$.
\end{definition}

\begin{remark}
    Именно такое определение кососимметрической билинейной функции правильное, потому что над полем характеристики $2$ стандартное определение $\B(x, y) = -\B(y, x)$ равносильно определению симметричности. А над полем характеристики не $2$ наше определение и стандартное равносильны. Действительно, если $\B(x, x) = 0$ $\forall x \in V$, то $\forall u, v \in V$ выполняется
    \[
        0 = \B(u + v, u + v) = \underbrace{\B(u, u)}_{0} + \B(u, v) + \B(v, u) + \underbrace{\B(v, v)}_{0} \Rightarrow \B(u, v) = -\B(v, u),
    \]
    а в обратную сторону очевидно.
\end{remark}

\begin{definition}
    \textit{Ядром} билинейной симметрической функции $\B$ называется множество векторов $x \in V$, для которых $\B(x, y) = 0$ $\forall y \in V$. Обозначается $\Ker\B$.
\end{definition}

Отметим, что $\Ker\B$ не зависит от порядка подстановки аргументов в функцию $\B$ из-за её симметричности.

