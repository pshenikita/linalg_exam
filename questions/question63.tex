\section{Коммутирующие операторы на конечномерном комплексном векторном пространстве}

\begin{theorem}
    Произведение эрмитовых операторов $\A\B$ является эрмитовым тогда и только тогда, когда $\A$ и $\B$ коммутируют.
\end{theorem}

\begin{proof}
    $\A\B = \B\A \Leftrightarrow (\A\B)^\ast = (\B\A)^\ast = \A^\ast\B^\ast = \A\B$.
\end{proof}

\begin{lemma}
    Пусть $\A$, $\B$ --- коммутирующие операторы на комплексном пространстве $V$. Тогда $\A$ и $\B$ имеют общий собственный вектор.
\end{lemma}

\begin{proof}
    Пусть $\lambda$ --- собственное значение оператора $\A$, а $V_\lambda$ --- соответствующее ему собственное подпространство. Как известно, оно инвариантно относительно $\A$. Тогда оно инвариантно и относительно оператора $\B$. Действительно, пусть $v \in V_\lambda$, тогда
    \[
        \A(\B v) = \B(\A v) = \B(\lambda x) = \lambda(\B v),
    \]
    следовательно, $\B v \in V_\lambda$. Оператор $\B\big|_{V_\lambda}$ имеет собственный вектор $u \in V_\lambda$: $\B u = \mu u$, $\mu$ --- собственное значение $\B$. Таким образом, $\A u = \lambda u$, $\B u = \mu u$, т.\,е. $u$ --- общий собственный вектор.
\end{proof}

\begin{theorem}
    Два эрмитовых (унитарных) оператора $\A$, $\B$ в эрмитовом пространстве $V$ одновременно приводятся к диагональному виду в некотором ортонормированном базисе тогда и только тогда, когда они коммутируют.
\end{theorem}

\begin{proof}
    $\Rightarrow$. Предположим, чо $\A$ и $\B$ диагонализируемы в общем ортонормированном базисе, мы приходим к выводу о перестановочности их матриц $A$, $B$ в этом базисе. Но очевидно, что коммутирующие матрицы коммутируют в любом базисе. Это устанавливается цепочкой равенств
    \[
        C^{-1}AC \cdot C^{-1}BC = C^{-1}ABC = C^{-1}BAC = C^{-1}BC \cdot C^{-1}AC.
    \]
    Таким образом, коммутируют и сами операторы.

    $\Leftarrow$. Обратно, пусть $\A\B = \B\A$. Тогда по предыдущей лемме операторы $|A$ и $\B$ имеют общий собственный вектор $v$. Без ограничения общности, можно считать $\abs{v} = 1$. Подпространство $W = \langle v\rangle^\perp$ размерности $n - 1$ инвариантно относительно $\A$ и $\B$, т.\,к. они эрмитовы или унитарны (важная лемма). Ограничения $\A$ и $\B$ на $W$ будут коммутирующими эрмиовыми (унитарными) операторами. Индукция по размерности приводит к явной конструкци ортонормированного базиса, в котором $\A$ и $\B$ запишутся в диагональной форме.
\end{proof}

