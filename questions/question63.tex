\section{Коммутирующие операторы на конечномерном комплексном векторном пространстве}

\begin{theorem}
    Произведение эрмитовых операторов $\A\B$ является эрмитовым тогда и только тогда, когда $\A$ и $\B$ коммутируют.
\end{theorem}

\begin{proof}
    $\A\B = \B\A \Leftrightarrow (\A\B)^\ast = (\B\A)^\ast = \A^\ast\B^\ast = \A\B$.
\end{proof}

\begin{theorem}
    Два эрмитовых (унитарных) оператора $\A$, $\B$ в эрмитовом пространстве $V$ одновременно приводятся к диагональному виду в некотором ортонормированном базисе тогда и только тогда, когда они коммутируют.
\end{theorem}

\begin{proof}
    $\Rightarrow$. Предположим, что $\A$ и $\B$ диагонализируемы в общем ортонормированном базисе, мы приходим к выводу о перестановочности их матриц $A$, $B$ в этом базисе. Но очевидно, что коммутирующие матрицы коммутируют в любом базисе. Это устанавливается цепочкой равенств
    \[
        C^{-1}AC \cdot C^{-1}BC = C^{-1}ABC = C^{-1}BAC = C^{-1}BC \cdot C^{-1}AC.
    \]
    Таким образом, коммутируют и сами операторы.

    $\Leftarrow$. Обратно, пусть $\A\B = \B\A$. Тогда по теоретической задаче 11 операторы $\A$ и $\B$ имеют общий собственный вектор $v$. Без ограничения общности, можно считать $\abs{v} = 1$. Подпространство $W = \langle v\rangle^\perp$ размерности $n - 1$ инвариантно относительно $\A$ и $\B$, т.\,к. они эрмитовы или унитарны (важная лемма). Ограничения $\A$ и $\B$ на $W$ будут коммутирующими эрмитовыми (унитарными) операторами. Индукция по размерности приводит к явной конструкции ортонормированного базиса, в котором $\A$ и $\B$ запишутся в диагональной форме.
\end{proof}

На самом деле, верно следующее обобщение упомянутой выше задачи.

\begin{theorem}
    Любое семейтво $\{\A_i : i \in I\}$ попарно коммутирующих операторов в конечномерном векторном пространстве над алгебраически замкнутым полем имеет общий собственный вектор.
\end{theorem}

\begin{proof}
    Если все операторы скалярные (в частности, при $n = 1$), то любой ненулевой вектор собственный для них всех. В общем случае применим индукцию по $n = \dim V$. Пусть $\A_{i_0}$ --- нескалярный оператор, $\lambda_0$ --- его собственное значение. Тогда $V_{\lambda_0}$ --- собственное подпространство для $\A_{i_0}$ --- имеет размерность $< n$. Оно инвариантно для всех $\A_i$ (т.\,к. они коммутируют, эта простая выкладка уже проделывалась).

    По предположению индукции, $\exists v_1 \in V_{\lambda_0}$ --- общий собственный вектор для $\A_i$ $\forall i \in I$.
\end{proof}

