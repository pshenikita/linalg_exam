\section{Положительно и отрицательно определённые вещественные квадратичные формы. Критерий Сильвестра}

\begin{definition}
    Симметрическая билинейная функция $\B$ называется \textit{положительно определённой}, если $\B(x, x) > 0$ при $x \ne \bs{0}$. Соответствующая квадратичная форма $Q(x)$ удовлетворяет условию $Q(x) > 0$ при $x \ne 0$ и также называется \textit{положительно определённой}.
\end{definition}

Положительно определённая симметрическая билинейная функция задаёт в пространстве $V$ скалярное произведения, т.\,е. превращает $V$ в евклидово пространство.

\begin{theorem}[Критерий Сильвестра]
    Симметрическая билинейная функция (квадратичная форма) положительно определена тогда и только тогда, когда все угловые миноры её матрицы в некотором базисе положительны.
\end{theorem}

\begin{proof}
    Пусть все угловые миноры $\det Q_k$ матрицы квадратичной формы $Q(x)$ положительны. Тогда в силу теоремы Якоби квадратичная форма приводится к виду $Q(u) = q^\prime_{11}(u^1)^2 + \ldots + q^\prime_{nn}(u^n)^2$, где $n = \rk Q = \dim V$, а $q^\prime_{kk} = \frac{\det Q_k}{\det Q_{k - 1}} > 0$. Такая квадратичная форма положительно определена, т.\,к. $Q(u) > 0$ при $u \ne 0$.

    Обратно, пусть $Q(x)$ положительно определена. Т.\,к. в любом базисе мы имеем $q_{ii} = \B(e_i, e_i) > 0$, всегда применимо основное преобразование метода Лагранжа. Тогда последовательно применяя основное преобразование, мы приведём квадратичную форму к виду $Q(u) = q^\prime_{11}(u^1)^2 + \ldots + q^\prime_{nn}(u^n)^2$, где $q^\prime_{ii} > 0$. Следовательно, $\det Q_k = \det Q^\prime_k = q^\prime_{11}\ldots q^\prime_{kk} > 0$ для любого $k$.
\end{proof}

\begin{remark}
    С аналогичным доказательством, симметрическая билинейная функция отрицательно определена тогда и только тогда, когда последовательность угловых миноров её матрицы в некотором базисе знакочередующаяся и $\det Q_1 < 0$.
\end{remark}

Про приведение квадратичных форм к каноническому виду \href{http://halgebra.math.msu.su/staff/klyachko/teaching/lin.al/QU.PDF}{есть файл} Антона Александровича Клячко. 

