\section{Псевдоевклидово пространство. Группы Лоренца и Пуанкаре}

\begin{definition}
    \textit{Псевдоевклидовым пространством} называется пространство с обобщённым скалярным произведением, задаваемым невырожденной симметрической билинейной функцией (не обязательно положительно определённой). Псевдоевклидово пространство размерности $p + q$ с симметрической билинейной функцией сигнатуры $p - q$ обозначается $\R^{p, q}$.
\end{definition}

\begin{definition}
    Группа операторов, сохраняющих псевдоевклидово скалярное произведение в $\R^{p, q}$, называется \textit{псевдоортогональной группой} и обозначается $O(p, q)$. В ортонормированном базисе матрица $A$ псевдоортогонально оператора удовлетворяет соотношению $A^tGA = G$, где
    \[
    G =
    \begin{pmatrix}
        \underset{p \times p}{E} & 0\\
        0 & -\underset{q \times q}{E}
    \end{pmatrix}.
    \]
    Такие матрицы называются \textit{псевдоортогональными}.
\end{definition}

Отметим, что любой оператор $\f \in O(p, q)$ переводит изотропные вектора в изотропные, потому что $(\f x, \f x) = (x, x) = 0$, если $x$ --- изотропный.

\begin{definition}
    Пространство $\R^{1, 3}$ называется \textit{пространством Минковского}, а группа $\bs{L} \vcentcolon = O(1, 3)$ называется \textit{группой Лоренца}.
\end{definition}

Такое четырёхмерное вещественное пространство с невырожденной симметричной метрикой с индексами инерции $(1, 3)$ занимает особое место в специальной теории относительности. Стандартными являются обозначения
\begin{align*}
    &V = \langle e_0, e_1, e_2, e_3\rangle,\\
    &x = te_0 + x_1e_1 + x_2e_2 + x_3e_3,\\
    &(x, x) = t^2 - x_1^2 - x_2^2 - x_3^2.
\end{align*}

Достаточно интересным является частный случай <<одномерной группы Лоренца>> $\bs{L}_1$ автоморфизмов двумерного пространства, сохраняющих метрику
\[
    (u, u) = t^2 - x^2.
\]

Группа $\bs{L}_1$ описывает физическое движение по прямой (у нас теперь $x$ --- не вектор, а координата вектора $u = te_0 + xe_1$). Ясно, что все изотропные векторы пропорциональны векторам $e_0 + e_1$ и $e_0 - e_1$. Поэтому для оператора $f \in O(1, 1)$ в силу его невырожденности имеются две возможности:
\[
    \f(e_0 + e_1) = \alpha(e_0 + e_1),\quad\f(e_0 - e_1) = \beta(e_0 - e_1)
\]
или
\[
    \f(e_0 + e_1) = \alpha(e_0 - e_1),\quad\f(e_0 - e_1) = \beta(e_0 + e_1).
\]

Рассмотрим одну из этих возможностей, например, первую. Имеем
\begin{align*}
    &\f e_0 = \frac{\alpha + \beta}{2}e_0 + \frac{\alpha - \beta}{2}e_1,\\
    &\f e_1 = \frac{\alpha - \beta}{2}e_0 + \frac{\alpha + \beta}{2}e_1.
\end{align*}
Матрица
\[
    F \vcentcolon =
    \begin{pmatrix}
        \frac{\alpha + \beta}{2} & \frac{\alpha - \beta}{2}\\
        \frac{\alpha - \beta}{2} & \frac{\alpha + \beta}{2}
    \end{pmatrix}
\]
оператора $\f$ имеет определитель $\det F = \alpha\beta$. Ограничимся собственным преобразованием Лоренца, т.\,е. будем считать $\alpha\beta = 1$. Для преобразования координат получим
\[
    \begin{pmatrix}
        t^\prime\\
        x^\prime
    \end{pmatrix} =
    \begin{pmatrix}
        \frac{\alpha^{-1} + \alpha}{2} & \frac{\alpha^{-1} - \alpha}{2}\\
        \frac{\alpha^{-1} - \alpha}{2} & \frac{\alpha^{-1} + \alpha}{2}
    \end{pmatrix} \cdot 
    \begin{pmatrix}
        t\\
        x
    \end{pmatrix},
\]
откуда
\begin{align*}
    &t^\prime = \frac{\alpha^{-1} + \alpha}{2}\br{t + \frac{\alpha^{-1} - \alpha}{\alpha^{-1} + \alpha}x},\\
    &x^\prime = \frac{\alpha^{-1} + \alpha}{2}\br{\frac{\alpha^{-1} - \alpha}{\alpha^{-1} + \alpha}t + x}.
\end{align*}

Введём обозначение
\[
    \frac{\alpha - \alpha^{-1}}{\alpha + \alpha^{-1}} = \vcentcolon v = \frac{\alpha^2 - 1}{\alpha^2 + 1}.
\]

Здесь $v$ --- скаляр. Рассматриваемая величина сответствует физической скорости, а её принято обозначать буквой $v$. Заметим, что всегда $\abs{v} < 1$, и поэтому имеют смысл выражения, вытекающие из последнего соотношения:
\begin{gather*}
    \alpha^2 = \frac{1 - v}{1 + v},\quad\alpha = \sqrt{\frac{1 - v}{1 + v}},\\
    \frac{\alpha + \alpha^{-1}}{2} = \frac{1}{\sqrt{1 - v^2}}.
\end{gather*}

Наконец, получаем
\[
    t^\prime = \frac{t - vx}{\sqrt{1 - v^2}},\quad x^\prime = \frac{x - vt}{\sqrt{1 - v^2}}.
\]

\begin{definition}
    Эти формулы называют \textit{преобразованиями Лоренца}.
\end{definition}

Преобразования выше записаны в масштабе, когда скорость света принята за $1$ и в целях лаконичности в таком виде и будет нами использоваться. Заметим, что при скорости, близкой к нулю (малой по сравнению со скоростью света), выведенные нами преобразования принимают вид преобразования Галилея:
\[
    t^\prime = t,\quad x^\prime = x - vt.
\]

Однако в общем случае положение точки характеризуется двумя координатами $(t, x)$ --- временной и пространственной. Положениям $(x_1, t_1)$, $(x_2, t_1)$ с одним и тем же $t = t_1$, в первой (неподвижной) системе координат соответствуют различные $t_1^\prime$ и $t_2^\prime$. Отсюда получается, например, закон изменения длин:
\[
    x_1^\prime - x_2^\prime = \frac{x_1 - vt_1}{\sqrt{1 - v^2}} - \frac{x_2 - vt_1}{\sqrt{1 - v^2}} = \frac{x_1 - x_2}{\sqrt{1 - v^2}}.
\]
Наоборот, при $x_1 = x_2$, $t_1 \ne t_2$ получим закон изменения времени.



