\section{Теорема Якоби}

Теорема Якоби позволяет (при выполнении некоторого дополнительного условия) найти нормальный вид квадратичной формы без нахождения преобразования.

Напомним, что \textit{угловым минором порядка $k$} матрицы $Q$ называется минор (определитель подматрицы), составленный из первых $k$ строк и первых $k$ столбцов. Угловой минор порядка $k$ будем обозначать через $\det Q_k$.

\begin{theorem}[Якоби]
    Предположим, что все угловые миноры матрицы $Q$ квадратичной формы отличны от нуля до порядка $r = \rk Q$. Тогда существует замена координат, приводящая данную квадратичную форму к виду
    \[
        \det Q_1(x^1)^2 + \frac{\det Q_2}{\det Q_1}(x^2)^2 + \ldots + \frac{\det Q_r}{\det Q_{r - 1}}(x^r)^2.
    \]
\end{theorem}

\begin{definition}
    Скажем, что для квадратичной формы имеет место \textit{регулярный случай}, если она приводится к диагональному виду последовательным применением исключительно основного преобразования Лагранжа.
\end{definition}

\begin{lemma}
    Для квадратичной формы $Q(x)$ имеет место регулярный случай тогда и только тогда, когда все угловые миноры её матрицы $Q$ отличны от нуля до порядка $r = \rk Q$.
\end{lemma}

\begin{proof}
    Пусть угловые миноры до порядка $r$ отличны от нуля. Тогда $q_{11} = \det Q_1$ --- угловой минор порядка $1$, который не равен нулю по предположению. Значит, на первом шаге применимо основное преобразование метода Лагранжа.

    Предположим теперь, что после $k$-кратного применения основного преобразования метода Лагранжа матрица квадратичной формы принимает вид
    \[
        Q^\prime =
        \left(
        \begin{array}{c c c | c c}
            q^\prime_{11} &  & 0 &  & \\
            & \ddots & & \bigzero & \\
            0 & & q^\prime_{kk} & &\\
            \hline
            & & & q^\prime_{k + 1, k + 1} & \cdots\\
            & \bigzero & & \vdots & \ddots
        \end{array}
        \right).
    \]

    Заметим, что матрица замены координат для основного преобразования есть
    \[
        C =
        \begin{pmatrix}
            1 & -\frac{q_{12}}{q_{11}} & \ldots & -\frac{q_{1n}}{q_{11}}\\
            0 & 1 & \ldots & 0\\
            \vdots & \vdots & \ddots & \vdots\\
            0 & 0 & \ldots & 1
        \end{pmatrix}
    \]
    (см. формулы из метода замены базисов). Для угловых подматриц $Q_k$ мы имеем $Q^\prime_k = C^t_kQ_kC_k$, где $C_k$ --- угловая подматрица матрицы $C$. Т.\,к. $\det C_k = 1$, мы получаем $\det Q^\prime_k = \det Q_k$, т.\,е. угловые миноры матрицы квадратичной формы не меняются при основном преобразовании метода Лагранжа.

    Возвращаясь к матрице $Q^\prime$, мы получаем $\det Q_{k + 1} = \det Q^\prime_{k + 1} = q^\prime_{11}\ldots q^\prime_{kk}q^\prime_{k + 1, k + 1} \ne 0$ при $k < r$ по предположению. Следовательно, $q^\prime_{k + 1, k + 1} \ne 0$, и мы снова можем применить основное преобразование.

    После $r$-кратного применения основного преобразования мы получаем матрицу $Q^\prime$, где $k = r$ и матрица в правом нижнем углу нулевая. Следовательно, квадратичная форма приведена к диагональному виду последовательным применением основного преобразования метода Лагранжа, и мы имеем регулярный случай.

    Пусть теперь имеет место регулярный случай, т.\,е. форма приведена к диагональному виду с ненулевыми числами $q^\prime_{11}, \ldots q^\prime_{rr}$ на диагонали последовательным применением основного преобразования. Тогда, т.\,к. угловые миноры не меняются при основном преобразовании, мы имеем $\det Q_k = \det Q^\prime_k = q^\prime_{11}\ldots q^\prime_{kk} \ne 0$ при $k \leqslant r$.
\end{proof}

Теперь докажем теорему Якоби.

\begin{proof}
    В силу предыдущей леммы, мы можем привести квадратичную форму к диагональному виду
    \[
        q^\prime_{11}(u^1)^2 + \ldots + q^\prime_{rr}(u^r)^2,
    \]
    используя лишь основное преобразование метода Лагранжа. Тогда $\det Q_k = \det Q^\prime_k = q^\prime_{11}\ldots q^\prime_{kk}$ при $k \leqslant r$, т.\,е. $q^\prime_{kk} = \frac{\det Q_k}{\det Q_{k - 1}}$, что и требовалось.
\end{proof}

