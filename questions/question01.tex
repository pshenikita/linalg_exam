\section{Векторные пространства. Линейная зависимость и независимость векторов. Базис, размерность. Примеры}

\begin{definition}
    \textit{Линейным} (или \textit{векторным}) \textit{пространством} над полем $\F$ называется множество $V$ с заданными на нём операциями \textit{сложения} $+: (u, v) \in V \times V \mapsto (u + v) \in V$ и \textit{умножения} элементов $V$ на элементы $\F$ $\bs{\cdot}: (\lambda, v) \in \F \times V \mapsto (\lambda \cdot v) \in V$, удовлетворяющие следующим аксиомам:
    \begin{enumerate}
        \item $v + u = u + v$ $\forall u, v \in V$;
        \item $(u + v) + w = u + (v + w)$ $\forall u, v, w \in V$;
        \item $\exists \bs{0} \in V: v + \bs{0} = v$ $\forall v \in V$;\hspace{2.9cm}\smash{\raisebox{.5\dimexpr\baselineskip+\itemsep+\parskip}{$\left.\rule{0pt}{.5\dimexpr4\baselineskip+3\itemsep+3\parskip}\right\}\text{$(V, +)$ --- абелева группа}$}}
        \item $\forall v \in V\;\exists (-v) \in V: (-v) + v = \bs{0}$;
        \item $\lambda \cdot (u + v) = \lambda \cdot u + \lambda \cdot v$ $\forall u, v \in V$, $\forall \lambda \in \F$;
        \item $(\lambda + \mu) \cdot v = \lambda \cdot v + \mu \cdot v$ $\forall v \in V$, $\forall \lambda, \mu \in \F$;
        \item $\lambda \cdot (\mu \cdot v) = (\lambda\mu) \cdot v$ $\forall v \in V$, $\forall \lambda, \mu \in \F$;\hspace{1cm}\smash{\raisebox{.5\dimexpr\baselineskip+\itemsep+\parskip}{$\left.\rule{0pt}{.5\dimexpr4\baselineskip+3\itemsep+3\parskip}\right\}\text{$\F$ линейно действует на $V$}$}}
        \item $1 \cdot v = v$ $\forall v \in V$.
    \end{enumerate}
\end{definition}

\begin{definition}
    Элементы множества $V$ называются \textit{векторами}, элемент $\bs{0}$ называется \textit{нулевым вектором}, а элемент $(-v)$ называется \textit{противоположным} к $v$. Элементы $\F$ называют \textit{скалярами}.
\end{definition}

\begin{proposal}
    \begin{enumerate}[nolistsep]
        \item[$1^\circ.$] $\bs{0} \cdot v = \lambda \cdot \bs{0} = \bs{0}$ $\forall v \in V$, $\forall \lambda \in \F$;
        \item[$2^\circ.$] $(-1) \cdot v = -v$ $\forall v \in V$;
        \item[$3^\circ.$] $\lambda v = \bs{0} \Rightarrow \lambda = 0$ или $v = \bs{0}$.
    \end{enumerate}
\end{proposal}

\begin{proof}
    \begin{enumerate}[nolistsep]
        \item[$1^\circ.$] $0 \cdot v + 0 \cdot v = (0 + 0) \cdot v = 0 \cdot v \Rightarrow 0 \cdot v = \bs{0}$. Аналогично, $\lambda \cdot \bs{0} + \lambda \cdot \bs{0} = \lambda(\bs{0} + \bs{0}) = \lambda \cdot \bs{0} \Rightarrow \lambda \cdot \bs{0} = \bs{0}$;
        \item[$2^\circ.$] $v + (-1) \cdot v = 1 \cdot v + (-1) \cdot v = (1 - 1) \cdot v = 0 \cdot v = \bs{0} \Rightarrow -v = (-1) \cdot v$.
        \item[$3^\circ.$] $\lambda \ne 0 \Rightarrow \bs{0} = \lambda^{-1}\lambda v = v$.
    \end{enumerate}
\end{proof}

\begin{problem}[А.\,А. Клячко\footnotemark]
    Найдите лишнюю аксиому и докажите, что она лишняя, а все остальные не лишние.
\end{problem}

\footnotetext{Ещё в первом семестре дело было\ldots}

\begin{solution}
    Лишней является аксиома коммутативности сложения (первая), докажем это. Заметим, что предложение 1 было доказано без использования коммутативности, значит, им здесь можно пользоваться. Нам будет нужен ещё два пункта (везде над переходами стоят номера пунктов или аксиом, которые использовались при этом переходе):
    \begin{enumerate}[nolistsep]
        \item[$4^\circ.$] $v - v \overset{8, 2^\circ}{=\joinrel=} 1 \cdot v + (-1) \cdot v \overset{6}{=} (1 - 1) \cdot v = 0 \cdot v \overset{1^\circ}{=} \bs{0} \Rightarrow$ \fbox{$v - v = \bs{0}$};
        \item[$5^\circ.$] $\bs{0} + v \overset{4^\circ}{=} (v - v) + v \overset{2}{=} v + (\underbrace{-v + v}_{{} = \bs{0}}) \overset{3, 4}{=\joinrel=} v \Rightarrow$ \fbox{$\bs{0} + v = v$}.
    \end{enumerate}

    Теперь мы можем вывести первую аксиому из остальных:
    \begin{multline*}
        u + v \overset{3}{=} (u + v) + \bs{0} \overset{4}{=} (u + v) + (-(v + u) + (v + u)) \overset{2}{=} ((u + v) - (v + u)) + (v + u) \overset{2}{=} {}\\\displaystyle {} = (u + \underbrace{v - v}_{{} = \bs{0}} - u) + (v + u) \overset{4^\circ}{=} (\underbrace{u - u}_{{} = \bs{0}}) + (v + u) \overset{4^\circ}{=} \bs{0} + (v + u) \overset{5^\circ}{=} v + u.
    \end{multline*}

Антон Александрович сказал, что остальные не выводятся. Но мы с Костей Зюбиным не смогли доказать это для 5-ой аксиомы. Доказательство независимости от остальных для каждой аксиомы проводится так: нужно привести пример такой структуры, в которой выполняются все аксиомы, кроме выбранной.
    \begin{itemize}
        \item \textbf{2 аксиома}. Рассмотрим множество $M = \{e, a, b\}$ с операцией $\ast$, заданной таблицей:
            $$
            \begin{array}{c || c | c | c}
                \ast & e & a & b\\
                \hline\hline
                e & e & a & b\\
                \hline
                a & a & a & e\\
                \hline
                b & b & e & b
            \end{array}
            $$

            Заметим, что операция $\ast$ коммутативна, в $M$ существует нейтральный элемент $e$ по $\ast$ и каждый элемент имеет обратный по $\ast$. Однако эта операция не ассоциативна:
            $$
            (b \ast a) \ast a = e \ast a = a,\quad b \ast (a \ast a) = b \ast a = e.
            $$

            Теперь возьмём алгебраическую систему $V = (\R \times M, +, \star)$ с операциями, определёнными по следующим правилам:
            $$
            u + v = (a, x) + (b, y) \vcentcolon = (a + b, x \ast y),\quad \lambda \star v = \lambda \star (a, x) \vcentcolon = (\lambda a, x).
            $$

            Аксиома 2 не выполнена, т.\,к. $\ast$ неассоциативна. Выполнение аксиом 1, 3, 4 следует из того, что они выполняются для $+$ над $\R$ и $\ast$ над $M$. Проверим выполнение остальных аксиом:
            $$\footnotesize
            \begin{array}{rl}
                5:\; & \lambda \star (a + b, x \ast y) = (\lambda(a + b), x \ast y) = (\lambda a + \lambda b, x \ast y) = (\lambda a, x) + (\lambda b, y) = \lambda\star((a, x) + (b, y))\\
                6:\; & (\lambda + \mu) \cdot (a, x) = ((\lambda + \mu)a, x) = (\lambda a + \mu a, x \ast x) = (\lambda a, x) + (\mu a, x) = \lambda \star (a, x) + \mu \star (a, x)\\
                7:\; & (\lambda\mu) \star (a, x) = (\lambda\mu a, x) = \lambda \star (\mu a, x)\\
                8:\; & 1 \star (a, x) = (1 \cdot a, x) = (a, x)
            \end{array}
            $$
        \item \textbf{3 аксиома}. Без третьей аксиомы нельзя ввести четвёртую, поэтому её удаление не имеет смысла.
        \item \textbf{4 аксиома}. Рассмотрим алгебраическую систему $V = (\R \cup \{\infty\}, +, \boldsymbol{\cdot})$. Доопределим сложение и умножение для $\infty$ следующим образом:
            $$
            \infty + a = a + \infty \vcentcolon = \infty,\quad \lambda \cdot \infty \vcentcolon = \infty.
            $$

            Выполнение аксиом 1-3 сразу вытекает из определения. Аксиома 4 не выполнена, т.\,к. у $\infty$ нет обратного по $+$. Выполнение аксиом 5-8 проверяется перебором нескольких случаев.
        \item \textbf{6 аксиома}. Рассмотрим алгебраическую систему $V = (\R, +, \star)$, в которой сложение определено так же, как в действительных числах, а умножение так:
            $$
            \lambda \star v \vcentcolon = v.
            $$

            Аксиомы 1-4 выполнены, т.\,к. они выполнены для $\R$ и $+$. Выполнение аксиом 5, 7 и 8 сразу вытекает из определения. Аксиома 6 не выполнена:
            $$
            u + u = 1 \star u + 1 \star u,\quad (1 + 1) \star u = u.
            $$
        \item \textbf{7 аксиома}. Рассмотрим $\R$ как векторное пространство над полем $\Q$ с базисом $M \supset \{1, \sqrt{2}\}$. Пусть отображение $f: \R \rightarrow \R$ задаётся своими значениями на числах из $M$, а для других чисел определяется соотношением
            $$
            f(q_1v_1 + q_2v_2 + \ldots + q_nv_n) = q_1f(v_1) + q_2f(v_2) + \ldots + q_nf(v_n),
            $$
            где $v_i$ --- некоторые векторы из базиса $M$. Такое отображение является линейным, сохраняющим все рациональыне числа.

            Теперь возьмём алгебраическую систему $V = (\R, +, \star)$, в которой сложение $+$ определено естественным образом, а умножение $\star$ определяется через $f$ и естественное умножение $\boldsymbol{\cdot}$:
            $$
            \lambda \star u \vcentcolon = f(\lambda) \cdot u.
            $$

            Аксиомы 1-4 выполнены, т.\,к. они выполнены для $\R$ и $+$. Выполнение аксиомы 5 проверяется непосредственно. Аксиома 6 выполнена, т.\,к. отображение $f$ линейно. Проверим, что аксиома 7 не выполнена:
            $$
            \sqrt{2} \star (\sqrt{2} \star u) = \sqrt{2} \star u = u,\quad (\sqrt{2} \star \sqrt{2}) \star u = 2u.
            $$

            Выполнение аксиомы 8 вытекает из определения.
        \item \textbf{8 аксиома}. Рассмотрим алгебраическую систему $V = (\R, +, \star)$, в которой сложение определено естественным образом, а умножение так:
            $$
            \lambda \star u \vcentcolon = \bs{0}.
            $$

            Аксиомы 1-4 выполнены, т.\,к. они выполнены для $\R$ и $+$. Выполнение аксиом 5-7 проверяется непосредственно. Аксиома 8 не выполнена.
    \end{itemize}
\end{solution}

\begin{example}
    \begin{enumerate}
        \item Множество $\{\bs{0}\}$ из одного элемента является линейным пространством над любым полем.
        \item Множества геометрических векторов на прямой, плоскости или пространстве являются линейными пространствами над полем $\R$.
        \item Поле $\F$ является векторным пространством над самим собой.
        \item Поле $\C$ является линейным пространством над полем $\R$, а поле $\R$ является линейным пространством над полем $\Q$.
        \item Пусть
            $\ds
                \F^n \vcentcolon =
                \left\{
                    \begin{pmatrix}
                        x_1\\
                        x_2\\
                        \vdots\\
                        x_n
                    \end{pmatrix} : x_i \in \F
                \right\}
            $
            --- множество столбцов фиксированной длины $n$ из элементов поля $\F$. Операции покоординатного сложения и умножения на скаляры
            \[
                \begin{pmatrix}
                    x_1\\
                    x_2\\
                    \vdots\\
                    x_n
                \end{pmatrix} +
                \begin{pmatrix}
                    y_1\\
                    y_2\\
                    \vdots\\
                    y_n
                \end{pmatrix} \vcentcolon=
                \begin{pmatrix}
                    x_1 + y_1\\
                    x_2 + y_2\\
                    \vdots\\
                    x_n + y_n
                \end{pmatrix},\quad
                \lambda\cdot
                \begin{pmatrix}
                    x_1\\
                    x_2\\
                    \vdots\\
                    x_n
                \end{pmatrix} \vcentcolon=
                \begin{pmatrix}
                    \lambda x_1\\
                    \lambda x_2\\
                    \vdots\\
                    \lambda x_n
                \end{pmatrix}.
            \]
            задают на $\F^n$ структуру линейного пространства над $\F$. Его часто называют \textit{координатным}.
    \end{enumerate}
\end{example}

Пусть $V$ --- линейное пространство над полем $\F$.

\begin{definition}
    \textit{Линейной комбинацией} системы векторов $\{v_i: i \in I\}$ называется формальная сумма вида $\sum\limits_{i \in I}\lambda_iv_i$, в которой лишь конечное число скаляров $\lambda_i$ отличны от нуля.
\end{definition}

\begin{remark}
    Линейную комбинацию системы $\{v_i : i \in I\}$ можно также определить как функцию $i \in I \mapsto \lambda_i \in \F$, которая принимает ненулевое значение только на конечном числе индексов.
\end{remark}

\begin{definition}
    Линейная комбинация $\sum\limits_{i \in I}\lambda_iv_i$ называется \textit{тривиальной}, если $\lambda_i = 0$ $\forall i \in I$.
\end{definition}

\begin{definition}
    Система векторов $\{v_i : i \in I\}$ называется \textit{линейно зависимой}, если существует нетривиальная линейная комбинация, представляющая нулевой вектор. В противном случае система называется \textit{линейно независимой}.
\end{definition}

\begin{lemma}
    Если система векторов $\{v_i : i \in I\}$ линейно зависима, то в ней найдётся вектор, представленный линейной комбинацией всех остальных.
\end{lemma}

\begin{proof}
    Пусть $\sum\limits_{i \in I}\lambda_iv_i = \bs{0}$, причём $\exists\lambda_j \ne 0$. Тогда $\ds v_j = \sum_{i \in I \setminus \{j\}}\frac{-\lambda_i}{\lambda_j}v_i$.
\end{proof}

\begin{definition}
    \textit{Базисом} пространства $V$ называется линейно независимая система $\{v_i : i \in I\}$, порождающая всё пространство $V$, т.\,е. такая, что каждый вектор из $V$ представляется какой-то линейной комбинацией системы $\{v_i : i \in I\}$.
\end{definition}

\begin{definition}
    Линейное пространство называется \textit{конечномерным}, если в нём существует конечная система векторов, порождающая его. В противном случае пространство называется \textit{бесконечномерным}.
\end{definition}

\begin{proposal}
    Представление любого вектора линейного пространства в виде линейной комбинации базисных векторов единственно.
\end{proposal}

\begin{proof}
    Действительно, если $v = \sum\limits_{i \in I}\lambda_iv_i = \sum\limits_{i \in I}\mu_iv_i$ (где $\{v_i : i \in I\}$ --- базис), то получаем $\bs{0} = \sum\limits_{i \in I}(\lambda_i - \mu_i)v_i$. Из линейной независимости базиса, линейная комбинация в правой части тривиальна и $\lambda_i = \mu_i$ $\forall i \in I$ и два представления $v$ совпадают.
\end{proof}

\begin{definition}
    \textit{Линейная оболочка} системы векторов $\langle v_i : i \in I \rangle$ есть множество всевозможных линейных комбинаций $\sum\limits_{i \in I}\lambda_iv_i$.
\end{definition}

\begin{theorem}
    В конечномерном пространстве все базисы состоят из одного числа элементов.
\end{theorem}

Доказательство этой теоремы будет опираться на следующую лемму.

\begin{lemma}[О линейной зависимости]
    Пусть $e_1, \ldots, e_m$ и $f_1, \ldots, f_n$ --- две (конечные) линейно независимые системы. Тогда $\{f_1, \ldots, f_n\} \subseteq \langle e_1, \ldots, e_m\rangle \Rightarrow n \leqslant m$.
\end{lemma}

\begin{proof}
    Пусть $f_j = a_{1j}e_1 + \ldots + a_{mj}e_m$, $a_{ij} \in \F$, $j = 1, \ldots, n$. Т.\,к. $f_1, \ldots, f_n$ --- линейно независимая система векторов, то $x_1f_1 + \ldots + x_nf_n = \bs{0} \Leftrightarrow x_1 = \ldots = x_n = 0$. Подставляя сюда выражения $f_i$ через $e_1, \ldots, e_m$, получаем
    \begin{multline*}
        \bs{0} = x_1(a_{11}e_1 + \ldots + a_{m1}e_m) + \ldots x_n(a_{1n}e_1 + \ldots + a_{mn}e_m) =\\ = (a_{11}x_1 + \ldots + a_{1n}x_n)e_1 + \ldots + (a_{m1}x_1 + \ldots + a_{mn}x_n)e_m.
    \end{multline*}
    Т.\,к. $e_1, \ldots, e_m$ --- линейно независимая система, то последнее равенство равносильно
    \[
        \begin{cases}
            a_{11}x_1 + \ldots + a_{1n}x_n = 0,\\
            \dotfill\\
            a_{m1}x_1 + \ldots + a_{mn}x_n = 0.
        \end{cases}
    \]

    Если $n > m$, то эта система имеет ненулевое решение, что противоречит линейной независимости системы $f_1, \ldots, f_n$.
\end{proof}

Теперь докажем теорему 1:

\begin{proof}
    Пусть $V$ --- конечномерное пространство и в $V$ существует базис $e_1, \ldots, e_m$. Пусть $\{f_i : i \in I\}$ --- другой базис. Если этот базис бесконечен, то в нём содержится конечная линейно независимая система векторов $f_1, \ldots, f_n$, где $n > m$. При этом, т.\,к. $e_1, \ldots, e_m$ --- базис, мы имеем $\{f_1, \ldots, f_n\} \subseteq \langle e_1, \ldots, e_m\rangle$, что противоречит лемме о линейной зависимости. Следовательно, базис $\{f_i : i \in I\}$ конечен, т.\,е. имеет вид $f_1, \ldots, f_n$. Тогда $\{f_1, \ldots, f_n\} \subseteq \langle e_1, \ldots, e_m \rangle$ и $\{e_1, \ldots, e_m\} \subseteq \langle f_1, \ldots, f_m\rangle$. Отсюда $n = m$.
\end{proof}

\begin{lemma}
    В конечномерном пространстве любую линейно независимую систему можно дополнить до базиса.
\end{lemma}

\begin{proof}
    Пусть $\{e_1, \ldots, e_k\}$ --- конечная подсистема в $V$. Тогда, если эта система максимальна по включению, то она базис. Иначе существует $e_{k + 1} \in V$ такой, что система $\{e_1, \ldots, e_k, e_{k + 1}\}$ линейно независима. Продолжная процесс далее, за конечное число шагов получим базис (в силу конечномерности пространства $V$).
\end{proof}

\begin{remark}
    Ниже изложен удобный алгоритм дополнения линейно независимой системы до базиса. Пусть 
    \[
        u_1 = 
        \begin{pmatrix}
            u_{11}\\
            \vdots\\
            u_{1n}
        \end{pmatrix},\quad
        u_2 = 
        \begin{pmatrix}
            u_{21}\\
            \vdots\\
            u_{2n}
        \end{pmatrix},\quad\ldots,\quad
        u_m = 
        \begin{pmatrix}
            u_{m1}\\
            \vdots\\
            u_{mn}
        \end{pmatrix}
    \] линейно независимы. Обозначим
    \[
        U = 
        \begin{pmatrix}
            u_{11} & u_{21} & \ldots & u_{m1}\\
            \vdots & \vdots & \ddots & \vdots\\
            u_{1n} & u_{2n} & \ldots & u_{mn}\\
        \end{pmatrix}.
    \]
    Отметим, что $\rk U = m$, а $\rk(U \mid E) = n$. Так что нужно привести матрицу $(U \mid E)$ элементарными преобразованиями строк к ступенчатому виду и дополнить столбцы матрицы $U$ единичными столбцами, вошедшими в базис матрицы $(U \mid E)$ (в первом семестре была теорема, что при элементарных преобразованиях строк линейные зависимости между столбцами не меняются).
\end{remark}

\begin{lemma}
    Всякое конечномерное линейное пространство $V$ обладает базисом. Более точно, из всякого конечного порождающего множества $S \subseteq V$ можно выбрать базис пространства $V$.
\end{lemma}

\begin{proof}
    Если множество $S$ линейно зависимо, то по лемме 1 в нём найдётся вектор, линейно выражающийся через остальные. Выкидывая этот вектор, мы получаем порождающее множество из меньшего числа векторов. Продолжая так дальше, мы в конце концов получим линейно независимое порождающее множество, т.\,е. базис.
\end{proof}

\begin{remark}
    Чтобы сделать это на практике, выписываем векторы в матрицу по столбцам, приводим её к ступенчатому виду и те столбцы, в которых стоят лидеры, будут базисными.
\end{remark}

\begin{definition}
    \textit{Размерностью} конечномерного линейного пространства $V$ (обозначается $\dim V$) называется число элементов в базисе $V$. Если $V$ бесконечномерно, то пишут $\dim V = \infty$.
\end{definition}

\begin{remark}
    В нулевом пространстве $\{\bs{0}\}$ базисом естественно считать пустое множество $\varnothing$. Поэтому $\dim\{\bs{0}\} = 0$.
\end{remark}

