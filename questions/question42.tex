\section{Полуторалинейные функции и формы. Эрмитовы формы и эрмитовы матрицы.
Канонический вид эрмитовой квадратичной формы. Критерий Сильвестра}

Я сразу писал и про евклидовы, и про эрмитовы пространства, поэтому следующих трёх билетах мне писать почти нечего, могу лишь сослаться на билеты, где это уже написано.

\begin{definition}
    Пусть $V$ --- линейное пространство над полем $\C$. Функция $\S: V \times V \to \C$ называется \textit{полуторалинейной}, если она линейна по второму аргументу и \textit{антилинейна} (или \textit{полулинейна}) по первому аргументу:
    \begin{gather*}
        \S(\lambda_1x_1 + \lambda_2, y) = \overline{\lambda_1}\S(x_1, y) + \overline{\lambda_2}\S(x_2, y),\\
        \S(x, \mu_1y_1 + \mu_2y_2) = \mu_1\S(x, y_1) + \mu_2\S(x, y_2)
    \end{gather*}
    для любых $\lambda_1, \lambda_2, \mu_1, \mu_2 \in \C$ и $x, x_1, x_2, y, y_1, y_2 \in V$. \textit{Матрица полуторалинейной функции $\S$} в базисе $e_1, \ldots, e_n$ определяется как $S = (s_{ij})$, где $s_{ij} = \S(e_i, e_j)$.
\end{definition}

Значение $\S(x, y)$ выражается через матрицу $S = (s_{ij})$ и координаты векторов $x = x^ie_i$ и $y = y^je_j$ следующим образом:
\[
    \S(x, y) = \S(x^ie_i, y^je_j) = \overline{x^i}y^j\S(e_i, e_j) = s_{ij}\overline{x^i}y^j = \overline{x}^tSy.
\]

\begin{definition}
    Выражение $S(x, y) = s_{ij}\overline{x^i}y_j = \overline{x}^tSy$ называется \textit{полуторалинейной формой}.
\end{definition}

Примером полуторалинейной функции может служить эрмитово скалярное произведение.

Матрицы $S$ и $S^\prime$ полуторалинейной функции $\S$ в разных базисах связаны соотношением
\[
    S^\prime = \overline{C}^tSC,
\]
которое доказывается аналогично соотношению для билинейных функций. Отсюда следует, что ранг матрицы полуторалинейной функции не зависит от выбора базиса.

\begin{definition}
    Полуторалинейная функция $\S$ называется \textit{эрмитовой}, если для любых векторов $x, y \in V$ выполняется $\S(x, y) = \overline{\S(y, x)}$.
\end{definition}

Функция $\S$ является эрмитовой тогда и только тогда, когда её матрица $B$ в произвольном базисе удовлетворяла условию $\overline{B^t} = B$.

\begin{definition}
    Такие матрицы называются \textit{эрмитовыми}.
\end{definition}

Очевидно, на диагонали эрмитовой матрицы стоят вещественные числа. Также стоит отметить, что определитель эрмитовой матрицы всегда вещественен, т.\,к.:
\[
    \det B = \det B^t = \det \overline{B} = \overline{\det B}.
\]

\begin{definition}
    Каждой эрмитовой полуторалинейной функции $\S$ соответствует \textit{эрмитова квадратичная функция} $q(x) \vcentcolon = \S(x, x)$.
\end{definition}

\begin{proposal}
    Всякая эрмитова квадратичная функция $q$ принимает только вещественные значения.
\end{proposal}

\begin{proof}
    Пусть $\S$ --- эрмитова полуторалинейная функция, соответствующая $q$. Тоода для любого вектора $x \in V$ имеем
    \[
        q(x) = \S(x, x) = \overline{\S(x, x)} = \overline{q(x)}.
    \]
    Значит, $q(x) \in \R$.
\end{proof}

Стоит заметить, что всякая эрмитова полуторалинейная функция $\S$ однозначно определяется своей эрмитовой квадратичной функцией $q$:
\[
    \S(x, y) = \frac{1}{4}(q(x + y) + iq(x + iy) - q(x - y) - iq(x - iy)).
\]

Как и в случае симметрической билинейной функции, доказывается аналоги теоремы 1 из вопроса 26 и предложения 1 из вопроса 27, из которых аналогично вещественному случаю выводится, что квадратичная любая форма заменой базиса может быть приведена к \textit{нормальному виду}
\[
    q(x) = \abs{x_1}^2 + \ldots + \abs{x_p}^2 - \abs{x_{p + 1}}^2 - \ldots - \abs{x_{p + q}}^2.
\]

При этом верны аналоги закона инерции, теоремы Якоби и критерия Сильвестра.

