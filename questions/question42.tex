\section{Полуторалинейные функции и формы. Эрмитовы формы и эрмитовы матрицы.
Канонический вид эрмитовой квадратичной формы. Критерий Сильвестра}

Я сразу писал и про евклидовы, и про эрмитовы пространства, поэтому в этом и следующих трёх билетах мне писать нечего, могу лишь сослаться на билеты, где это уже написано.

\begin{definition}
    Пусть $V$ --- линейное пространство над полем $\C$. Функция $\S: V \times V \to \C$ называется \textit{полуторалинейной}, если она линейна по второму аргументу и \textit{антилинейна} (или \textit{полулинейна}) по первому аргументу:
    \begin{gather*}
        \S(\lambda_1x_1 + \lambda_2, y) = \overline{\lambda_1}\S(x_1, y) + \overline{\lambda_2}\S(x_2, y),\\
        \S(x, \mu_1y_1 + \mu_2y_2) = \mu_1\S(x, y_1) + \mu_2\S(x, y_2)
    \end{gather*}
    для любых $\lambda_1, \lambda_2, \mu_1, \mu_2 \in \C$ и $x, x_1, x_2, y, y_1, y_2 \in V$. \textit{Матрица полуторалинейной функции $\S$} в базисе $e_1, \ldots, e_n$ определяется как $S = (s_{ij})$, где $s_{ij} = \S(e_i, e_j)$.
\end{definition}

Значение $\S(x, y)$ выражается через матрицу $S = (s_{ij})$ и координаты векторов $x = x^ie_i$ и $y = y^je_j$ следующим образом:
\[
    \S(x, y) = \S(x^ie_i, y^je_j) = \overline{x^i}y^j\S(e_i, e_j) = s_{ij}\overline{x^i}y^j = \overline{x}^tSy.
\]

\begin{definition}
    Выражение $S(x, y) = s_{ij}\overline{x^i}y_j = \overline{x}^tSy$ называется \textit{полуторалинейной формой}.
\end{definition}

Примером полуторалинейной функции может служить эрмитово скалярное произведение.

Матрицы $S$ и $S^\prime$ полуторалинейной функции $\S$ в разных базисах связаны соотношением
\[
    S^\prime = \overline{C}^tSC,
\]
которое доказывается аналогично соотношению для билинейных функций. Отсюда следует, что ранг матрицы полуторалинейной функции не зависит от выбора базиса.

Критерий Сильвестра уже был в вопросе 29.

