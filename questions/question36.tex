\section{Изоморфизм евклидовых пространств одинаковой размерности. Изоморфизм евклидова пространства и его сопряжённого}

\begin{definition}
    Два евклидовых или эрмитовых пространства $V$ и $W$ называются \textit{изоморфными}, если существует изоморфизм линейных пространств $\A: V \to W$, сохраняющий скалярное произведение, т.\,е. $(\A u, \A v) = (u, v)$ $\forall u, v \in V$.
\end{definition}

\begin{proposal}
    Два евклидовых или эрмитовых пространства $V$ и $W$ изоморфны тогда и только тогда, когда их размерности совпадают.
\end{proposal}

\begin{proof}
    Если $V$ и $W$ изоморфны как евклидовы (эрмитовы) пространства, то они изоморфны как линейные пространства, а потому $\dim V = \dim W$.
    
    Доказательство обратного утверждения аналогично доказательству соответствующего утверждения для линейных пространств: изоморфизм между евклидовыми (эрмитовыми) пространствами размерности $n$ устанавливается при помощи биекции между базисами $e_1, \ldots, e_n$ в $V$ и $f_1, \ldots, f_n$ в $W$. Для того, чтобы получаемый изоморфизм линейных пространств $\A: V \to W$ сохранял скалярное произведение, базисы необходимо выбрать ортонормированными. В этом случае мы имеем $f_i = \A e_i$ и $(e_i, e_j) = (f_i, f_j) = \delta_{ij}$. Поэтому для любых векторов $u = u^ie_i$ и $v = v^je_j$ мы имеем
    \[
        (\A u, \A v) = \overline{u^i}v^j(\A e_i, \A e_j) = \overline{u^i}v^j(f_i, f_j) = \overline{u^i}v^j\delta_{ij} = \sum_{i = 1}^n\overline{u^i}v^i = (u, v),
    \]
    т.\,е. изоморфизм $\A$ сохраняет скалярное произведение.
\end{proof}

Т.\,к. пространства $V$ и $V^\ast$ имеют одну размерность (в конечномерном случае), то они изоморфны. Однако построение изоморфизма между ними требует выбора базисов и в этом смысле неканонично. Оказывается, что в присутствии скалярного произведения можно установить изоморфизм $V \to V^\ast$ каноничным образом, т.\,е. не прибегая к выбору базисов.

Пусть $V$ --- евклидово пространство. Каждому вектору $u \in V$ сопоставим линейную функцию $\xi_u \vcentcolon = (u, \bs{\cdot})$.

\begin{theorem}
    Пусть $V$ --- евклидово пространство. Отображение $u \mapsto \xi_u$ устанавливает канонически изоморфизм $\A: V \to V^\ast$.
\end{theorem}

\begin{proof}
    Линейность отображения $u \mapsto \xi_u$ вытекает из линейности скалярного произведения по первому аргументу. Т.\,к. $\dim V = \dim V^\ast$, чтобы проверить, что $\A: V \to V^\ast$ --- изоморфизм, достаточно проверить, что $\Ker\A = \{\bs{0}\}$. Пусть $v \in \Ker\A$, т.\,е. $\A v = \xi_v = 0$. Тогда $\xi_v(w) = (u, w) = 0$ $\forall w \in W$. Но тогда и $(v, v) = 0$, значит, $v = 0$ и ядро отображения $\A$ нулевое.
\end{proof}

Аналогичным образом для эрмитова пространства $V$ устанавливается канонический изоморфизм $\overline{V} \to V^\ast$, $u \mapsto (u, \bs{\cdot})$, где $\overline{V}$ --- \textit{комплексно сопряжённое пространство} (с умножением на скаляры, определённым по формуле $\lambda \cdot v \vcentcolon = \overline{\lambda}v$). Это позволяет отождествить два понятия <<сопряжённого>> пространства для эрмитова пространства $V$.

