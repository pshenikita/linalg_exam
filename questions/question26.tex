\section{Квадратичные формы и их матрицы. Восстановление симметрической билинейной функции по даной квадратичной функции. Диагональный вид квадратичной формы. Алгоритм Лагранжа}

\begin{definition}
    \textit{Квадратичной формой} над $\F$ называется однородный многочлен второй степени от $n$ переменных $x = (x^1, \ldots, x^n)$, т.\,е. многочлен вида
    \[
        Q(x) = Q(x^1, \ldots, x^n) = q_{ij}x^ix^j = \sum_{i = 1}^nq_{ii}(x^i)^2 + \sum_{i < j}2q_{ij}x^ix^j,
    \]
    где $q_{ji} = q_{ij} \in \F$. Симметричная матрица $Q = (q_{ij})$ размера $n \times n$ называется \textit{матрицей квадратичной формы}.
\end{definition}

Если $B(x, y) = b_{ij}x^iy^j$ --- симметрическая билинейная форма, то $B(x, x) = b_{ij}x^ix^j$ является квадратичной формой с матрицей $B$. Таким образом, квадратичная форма $B(x, x)$ полностью определяет симметрическую билинейную форму $B(x, y)$, а значит, и симметрическую билинейную функцию $\B(x, y)$. Это можно увидеть и не прибегая к выбору базиса: для симметрической билинейной функции имеет место соотношение
\[
    \B(x, y) = \frac{1}{2}(\B(x + y, x + y) - \B(x, x) - \B(y, y)).
\]

\begin{remark}
    Заметим, что здесь мы должны уметь делить на $2$, поэтому формула верна только для полей с характеристикой не $2$.
\end{remark}

\begin{definition}
    Функцию $V \to \F$, $x \mapsto \B(x, x)$ называют \textit{квадратичной функцией}.
\end{definition}

\begin{theorem}
    Для симметрической билинейной функции $\B$ над полем характеристики, отличной от $2$, существует базис, в котором её матрица диагональна. Другими словами, любую квадратичную форму $Q(x)$ линейной заменой координат $x = Cy$ можно привести к виду
    \[
        Q(y) = r_{11}(y^1)^2 + \ldots + r_{nn}(y^n)^2.
    \]
\end{theorem}

Мы приведём два доказательства этого факта. В первом случае мы будем работать с квадратичными формами и координатами, а во втором --- с симметрическими билинейными функциями и базисами. Каждое из доказательств будет проведено таким образом, что его можно будет использовать как алгоритм.

\smallskip
\noindent{\bfseries Первое доказательство {\mdseries (метод Лагранжа)}.}
Пусть $Q(x) = q_{ij}x^ix^j$ --- квадратичная форма. Доказательство заключается в последовательном упрощении $Q(x)$, использующем основное и два вспомогательных преобразования.

\textbf{Основное преобразование} производится, если в квадратичной форме $Q(x) = q_{ij}x^ix^j$ первый коэффициент $q_{11}$ не равен нулю. Тогда имеем
\begin{multline*}
    Q(x^1, \ldots, x^n) = q_{11}(x^1)^2 + 2q_{12}x^1x^2 + \ldots + 2q_{1n}x^1x^n + \sum_{i, j > 1}q_{ij}x^ix^j =\\ =
    q_{11}\br{x^1 + \frac{q_{12}}{q_{11}}x^2 + \ldots + \frac{q_{1n}}{q_{11}}x^n}^2 - q_{11}q_{11}\br{\frac{q_{12}}{q_{11}}x^2 + \ldots + \frac{q_{1n}}{q_{11}}x^n}^2 + \sum_{i, j > 1}q_{ij}x^ix^j =\\ = q_{11}\br{x^1 + \frac{q_{12}}{q_{11}}x^2 + \ldots + \frac{q_{1n}}{q_{11}}x^n}^2 + Q^\prime(x^2, \ldots, x^n),
\end{multline*}
где $Q^\prime(x^2, \ldots, x^n)$ --- некоторая квадратичная форма от $n - 1$ переменных. Теперь сделаем замену координат
\begin{gather*}
    u^1 = x^1 + \frac{q_{12}}{q_{11}}x^2 + \ldots + \frac{q_{1n}}{q_{11}}x^n,\\
    u^2 = x^2,\quad \ldots,\quad u^n = x^n.
\end{gather*}

В результате $Q(x)$ преобразуется к виду
\[
    Q(u^1, \ldots, u^n) = q_{11}(u^1)^2 + Q^\prime(u^2, \ldots, u^n).
\]

Если в форме $Q^\prime(u^2, \ldots, u^n)$ первый коэффициент (т.\,е. $q^\prime_{22}$) не равен нулю, то мы снова можем применить основное преобразование, и т.\,д.

\textbf{Первое вспомогательное преобразование} производится, если $q_{11} = 0$, но существует $q_{ii} \ne 0$. В этом случае мы делаем замену $u^1 = x^i, u^i = x^1$, а остальные координаты без изменений. В результате получаем $q^\prime_{11} \ne 0$.

\textbf{Второе вспомогательное преобразование} производится, если все коэффициенты $q_{ii}$ при квадратах равны нулю, но при этом есть хотя бы один ненулевой коэффициент (в противном случае $Q(x) \equiv 0$ уже имеет нужный вид). Пусть $q_{ij} \ne 0$, где $i < j$. Произведём замену координат
\[
    x^i = u^i,\quad x^j = u^i + u^j,\quad x^k = u^k\text{ при $k \ne i, j$}.
\]

В результате форма $Q(x)$ преобразуется к виду
\[
    Q(x) = 2q_{ij}x^ix^j + \ldots = 2q_{ij}u^i(u^i + u^j) + \ldots = 2q_{ij}(u^i)^2 + \ldots,
\]
где $\ldots$ означает члены, не содержащие квадратов. Далее мы можем применить предыдущие преобразования.

Последовательно применяя основное преобразование и (если нужно) вспомогательные преобразования, мы приводим форму $Q(x)$ к диагональному виду.
\hfill$\blacksquare$\par\smallskip

\smallskip
\noindent{\bfseries Первое доказательство {\mdseries (метод поиска базиса)}.}
Пусть $B = (b_{ij})$ --- матрица билинейной функции $\B$ в исходном базисе $e_1, \ldots, e_n$.

\textbf{Основное преобразование} производится, если $b_{11} = \B(e_1, e_1) \ne 0$. Выберем новый базис следующим образом:
\begin{align*}
    e_{1^\prime} &= e_1,\\
    e_{2^\prime} &= e_2 - \frac{\B(e_1, e_2)}{\B(e_1, e_1)}e_1 = e_2 - \frac{b_{12}}{b_{11}}e_1,\\
    \vdots\\
    e_{n^\prime} &= e_n - \frac{\B(e_1, e_n)}{\B(e_1, e_1)}e_1 = e_n - \frac{b_{1n}}{b_{11}}e_1.
\end{align*}

В результате получим $\B(e_{1^\prime}, e_{1^\prime}) = 0$ при $i > 1$. Таким образом, матрица билинейной функции $\B$ в новом базисе принимает вид
\[
    B^\prime =
    \left(
    \begin{array}{c | c c c}
        b_{11} & 0 & \ldots & 0\\
        \hline
        0 & & & \\
        \vdots & & \widetilde{B}^\prime & \\
        0 & & & 
    \end{array}
    \right),
\]
где $\widetilde{B}^\prime$ --- матрица размера $(n - 1) \times (n - 1)$ билинейной функции $\B$ на подпространстве $\langle e_{2^\prime}, \ldots, e_{n^\prime} \rangle$. Далее мы работаем уже с этой матрицей $\widetilde{B}^\prime$.

Первое вспомогательное преобразование производится, если $b_{11} = 0$, но имеется $b_{ii} \ne 0$. Тогда делаем замену, меняющую местами первый и $i$-ый базисный векторы.

Второе вспомогательное преобразование производится, если все $b_{ii}$ равны нулю, но при этом билинейная функция $\B$ не является тождественно нулевой, т.\,е. $b_{ij} = \B(e_i, e_j) \ne 0$ для некоторых $i < j$. Произведём замену базиса
\[
    e_{i^\prime} = e_i + e_j,\quad e_{j^\prime} = e_j,\quad e_{k^\prime} = e_k\text{ при $k \ne i, j$}.
\]

Далее можем применить предыдущие преобразования.

Последовательно применяя основное преобразование и дополняя его в необходимых случаях вспомогательными преобразованиями, мы получаем базис $f_1, \ldots, f_n$, в котором матрица билинейной функции $\B$ имеет диагональный вид.
\hfill$\blacksquare$\par\smallskip

Обратим внимание, что основное и вспомогательное преобразование в обоих доказательствах --- это одно и то же преобразование, просто в первом случае оно записано через координаты, а во втором --- через базисы. Так что диагональные матрицы, получаемые первым и вторым методом, совпадают, как и все промежуточные матрицы.

\begin{remark}
    Если при приведении матрицы билинейной функции (квадратичной формы) к диагональному виду использовалось лишь основное преобразование, то матрица перехода от исходного базиса к базису, в котором матрица имеет диагональный вид, является верхнетреугольной.
\end{remark}

\begin{problem}
    Над полем $\Z_2$ симметрическая билинейная функция с матрицей
    $
    \begin{pmatrix}
        0 & 1\\
        1 & 0
    \end{pmatrix}
    $
    не приводится к диагональному виду заменой базиса.
\end{problem}

Приведём здесь два решения. Первое моё, второе --- Антона Александровича.

\begin{solution}
    Пусть можно, и матрица замены базисов имеет вид
    $C = 
    \begin{pmatrix}
        a & b\\
        c & d
    \end{pmatrix}
    $. Произведём замену:
    \[
        C^t
        \begin{pmatrix}
            0 & 1\\
            1 & 0
        \end{pmatrix}
        C =
        \begin{pmatrix}
            a & c\\
            b & d
        \end{pmatrix}
        \begin{pmatrix}
            0 & 1\\
            1 & 0
        \end{pmatrix}
        \begin{pmatrix}
            a & b\\
            c & d
        \end{pmatrix} =
        \begin{pmatrix}
            c & a\\
            d & b
        \end{pmatrix}
        \begin{pmatrix}
            a & b\\
            c & d
        \end{pmatrix} =
        \begin{pmatrix}
            2ac & bc + ad\\
            bc + ad & 2bd
        \end{pmatrix}.
    \]
    Тут можно рассуждать по-разному:
    \begin{enumerate}
        \item Над $\Z_2$ имеем $2 = 0$, поэтому на диагонали у такой матрицы стоят нули. Поэтому если она диагональная, то нулевая, а нулевая матрица --- она нулевая в любом базисе.
        \item Вне диагонали должны стоять нули, поэтому $bc + ad = 0$. Над $\Z_2$ имеем $-1 = 1$, поэтому $0 = ad - bc = \det C$, но матрица перехода между базисами должна быть невырожденной.
    \end{enumerate}
\end{solution}

\begin{solution}
    Пусть $\B$ --- билинейная функция с такой матрицей в базисе $e_1, e_2$. Тогда
    \[
        \B(x, x) = \B(x^1e_1 + x^2e_2, x^1e_1 + x^2e_2) = x^1\underbrace{\B(e_1, e_1)}_{0} + x^2\underbrace{\B(e_2, e_2)}_{0} + x^1x^2(\underbrace{\B(e_1, e_2) + \B(e_1, e_2)}_{\text{Над $\Z_2$ это $0$}}) = 0.
    \]
    Значит, при любой замене базиса на диагонали будут стоять нули. А ещё при любой замене базиса матрица должна быть симметрической. Поэтому она либо нулевая, либо такая, как есть. Ранг должен сохраняться, поэтому она такая, как есть, т.\,е. не диагональная.
\end{solution}

