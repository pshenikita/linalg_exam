\section{Инвариантные подпространства линейного оператора. Ограничение линейного оператора на инвариантное подпространство. Вид матрицы линейного оператора при наличии инвариантных подпространств}

\begin{definition}
    Подпространство $W \subseteq V$ называется \textit{инвариантным} относительно оператора $\A: V \to V$, если $\A(W) \subseteq W$.
\end{definition}

\begin{example}
    $\Ker\A$ и $\Im\A$ являются инвариантными подпространствами. 
\end{example}

Пусть $W \subseteq V$ --- инвариантное подпространство для оператора $\A: V \to V$. Выберем базис $e_1, \ldots, e_k$ в $W$ и дополним его до базиса $e_1, \ldots, e_k, e_{k + 1}, \ldots, e_n$ в $V$. Пусть $A = (a^i_j)$ --- матрица оператора $\A$ в этом базисе. Тогда $\A e_j = a^1_je_1 + \ldots a^k_je_k$ при $j = 1, \ldots, k$. Это означает, что матрица $A$ имеет вид
$
A =
\left(\begin{array}{c | c}
    \ast & \ast\\
    \hline
    0 & \ast
\end{array}
\right)
$, где в левом нижнем углу стоит матрица размера $(n - k) \times k$ из нулей. Аналогично, если имеет место разложение $V = W_1 \oplus W_2$ в прямую сумму инвариантных подпространств, $\A(W_1) \subseteq W_1$, $\A(W_2) \subseteq W_2$, то в подходящем базисе матрица оператора $\A$ будет иметь блочно-диагональный вид:
$
A =
\left(\begin{array}{c | c}
    \ast & 0\\
    \hline
    0 & \ast
\end{array}
\right)
$.

\begin{definition}
    Пусть $W \subseteq V$ --- инвариантное подпространство для оператора $\A: V \to V$. Тогда оператор $\widehat{\A}: W \to W$, определённый равенством $\widehat{\A}w \vcentcolon = \A w$ для $w \in W$, называется \textit{ограничением} оператора $\A$ на подпространство $W$ и обозначается как $\A\big|_W$.
\end{definition}

\begin{definition}
    Линейный оператор $\widetilde{A}: V / W \to V / W$, определённый на классах смежности по правилу $\widetilde{A}(v + W) = \A v + W$, называется \textit{фактор-оператором}.
\end{definition}

Определение фактор-оператора корректно. Действительно, если $v + W = u + W$, то $v - u \in W$, $\A(v - u) \in W$, и мы имеем $\widetilde{A}(v + W) = \A(u + v - u) + W = \A u + \A(v - u) + W = \A u + W = \widetilde{\A}(u + W)$.

\begin{proposal}
    Пусть $W \subseteq V$ --- инвариантное подпространство для оператора $\A: V \to V$. Пусть $e_1, \ldots, e_k$ --- базис в $W$ и $e_1, \ldots, e_k, e_{k + 1}, \ldots, e_n$ --- базис в $V$. Тогда матрица оператора $\A$ в этом базисе имеет вид
    $
    A =
    \left(
    \begin{array}{c | c}
        \widehat{A} & \ast\\
        \hline\vspace{-4mm}\\
        0 & \widetilde{A}
    \end{array}
    \right)
    $, где $\widehat{A}$ --- матрица ограничения $\A\big|_W$ в базисе $e_1, \ldots, e_k$, а $\widetilde{A}$ --- матрица фактор-оператора в базисе $e_{k + 1} + W, \ldots, e_n + W$ факторпространства $V / W$.
\end{proposal}

\begin{proof}
    Пусть 
    $
    A =
    \begin{pmatrix}
        a^1_1 & a^1_2 & \ldots & a^1_n\\
        a^2_1 & a^2_2 & \ldots & a^2_n\\
        \vdots & \vdots & \ddots & \vdots\\
        a^n_1 & a^n_2 & \ldots & a^n_n
    \end{pmatrix}
    $. Тогда
    $
    \widehat{A} =
    \begin{pmatrix}
        a^1_1 & \ldots & a^1_k\\
        \vdots & \ddots & \vdots\\
        a^k_1 & \ldots & a^k_k
    \end{pmatrix}
    $ и
    $
    \widetilde{A} =
    \begin{pmatrix}
        a^{k + 1}_{k + 1} & \ldots & a^{k + 1}_n\\
        \vdots & \ddots & \vdots\\
        a^n_{k + 1} & \ldots & a^n_n
    \end{pmatrix}
    $. Разложим вектор по базису: $v = v^1e_1 + \ldots + v^ke_k + \ldots + v^{k + 1}e_{k + 1} + \ldots + v^ne_n$. Отметим, что в силу инвариантности подпространства $W$ имеет место
    $
    A \cdot (v^1, \ldots, v^k, 0, \ldots, 0)^t = (\underbrace{\ast, \ldots, \ast}_{k}, 0, \ldots, 0)^t
    $. Итак,
    \begin{multline*}
        \A v = A \cdot v =
        \begin{pmatrix}
            a^1_1 & a^1_2 & \ldots & a^1_n\\
            a^2_1 & a^2_2 & \ldots & a^2_n\\
            \vdots & \vdots & \ddots & \vdots\\
            a^n_1 & a^n_2 & \ldots & a^n_n
        \end{pmatrix} \cdot
        \begin{pmatrix}
            v_1\\
            \vdots\\
            v_k\\
            v_{k + 1}\\
            \vdots\\
            v_n
        \end{pmatrix} = 
        \begin{pmatrix}
            a^1_1 & a^1_2 & \ldots & a^1_n\\
            a^2_1 & a^2_2 & \ldots & a^2_n\\
            \vdots & \vdots & \ddots & \vdots\\
            a^n_1 & a^n_2 & \ldots & a^n_n
        \end{pmatrix} \cdot \left(
        \begin{pmatrix}
            v_1\\
            \vdots\\
            v_k\\
            0\\
            \vdots\\
            0
        \end{pmatrix}
        +
        \begin{pmatrix}
            0\\
            \vdots\\
            0\\
            v_{k + 1}\\
            \vdots\\
            v_n
        \end{pmatrix}
        \right) =\\
        =
        \left(
        \begin{array}{c | c}
            \widehat{A} & 0\\
            \hline
            0 & 0
        \end{array}
        \right) \cdot v + 
        \left(
        \begin{array}{c | c}
            0 & \ast\\
            \hline\vspace{-4mm}\\
            0 & \widetilde{A}
        \end{array}
        \right) \cdot v =
        \left(
        \begin{array}{c | c}
            \widehat{A} & \ast\\
            \hline\vspace{-4mm}\\
            0 & \widetilde{A}
        \end{array}
        \right) \cdot v.
    \end{multline*}
    Отсюда и следует, что в выбранном базисе матрица $A$ имеет указанный вид.
\end{proof}

\begin{remark}
    В конспектах Т.\,Е. Панова написано, что это утверждение примерно очевидно и доказывать мы его не будем. Поэтому изложенное выше доказательство я проделал сам, оно может содержать ошибки.
\end{remark}

\begin{corollary}
    Если пространство $V$ представлено в виде прямой суммы ненулевых инвариантных относительно преобразования $\A$ подпространств $V = W_1 \oplus \ldots \oplus W_k$, то существует базис, в котором матрица преобразования имеет блочно-диагональный вид
    $
        A = \operatorname{diag}(A_1, \ldots, A_k) =
        \begin{pmatrix}
            A_1 & \ldots & 0\\
            \vdots & \ddots & \vdots\\
            0 & \ldots & A_k
        \end{pmatrix}
    $, где $A_i$ --- матрица ограничения $\A\big|_{W_i}$ преобразования $\A$ на подпространство $W_i$, $i = 1, \ldots, k$.
\end{corollary}

