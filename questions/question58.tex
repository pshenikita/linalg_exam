\section{Определение тензора типа $(p, q)$ на векторном пространстве. Линейные операции. Отождествление тензоров малых валентностей с геометрическими объектами}

Пусть $V$ --- линейное пространство над полем $\F$ нулевой характеристики (обычно $\F = \R$ или $\C$) и $V^\ast$ --- двойственное пространство. Элементы $v \in V$ --- это, как обычно, векторы, а элементы $\xi \in V^\ast$ здесь мы будем называть \textit{ковекторами}.

\begin{definition}
    \textit{Тензором} типа $(p, q)$ (и \textit{валентности} $p + q$) называется функция
    \[
        \T: \underbrace{V \times \ldots \times V}_p \times \underbrace{V^\ast \times \ldots \times V^\ast}_q \to \F
    \]
    от $p$ векторных и $q$ ковекторных аргументов, которая линейна по каждому аргументу.
\end{definition}

Тензоры типа $(p, q)$ образуют линейное пространство над полем $\F$, в котором сумма и умножение на скаряны определены по формуле
\[
    (\lambda\T + \mu\S)(v_1, \ldots, v_p, \xi^1, \ldots, \xi^q) \vcentcolon = \lambda\T(v_1, \ldots, v_p) + \mu\S(v_1, \ldots, v_p, \xi^1, \ldots, \xi^q).
\]

Мы будем обозначать это пространство через $\Ten_p^q(V)$.

\begin{example}
    Геометрическая интерпреация тензоров малых валентностей:
    \begin{enumerate}
        \item Тензоры типа $(0, 0)$ --- это скаляры из поля $\F$.
        \item Тензор типа $(1, 0)$ --- линейная функция на $V$, т.\,е. элемент $V^\ast$.
        \item Тензор типа $(0, 1)$ --- линейная функция на $V^\ast$, т.\,е. элемент из $V^{\ast\ast}$. Однако пространства $V$ и $V^{\ast\ast}$ канонически изоморфны. Имея в виду этот изоморфизм, тензор типа $(0, 1)$ можно считать вектором, т.\,е. элементом из $V$.
        \item Тензор типа $(2, 0)$ --- это билинейная функция на $V$.
        \item Тензор типа $(1, 1)$ --- это линейный оператор $\A: V \to V$. Действительно, каждому тензору $\T(v, \xi)$ можно сопоставить оператор $\A(x) \vcentcolon = \xi(x)$. Обратно, каждому оператору можем сопоставить тензор $\T(x, \xi) \vcentcolon = \xi(\A x)$ типа $(1, 1)$.
    \end{enumerate}
\end{example}

