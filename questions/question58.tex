\section{Определение тензора типа $(p, q)$ на векторном пространстве. Линейные операции. Отождествление тензоров малых валентностей с геометрическими объектами}

Здесь я основывался в основном на лекциях Панова, так что определения слегка странноваты. Но мне понравилось, так что здесь написано то, что написано.

Пусть $V$ --- линейное пространство над полем $\F$ нулевой характеристики (обычно $\F = \R$ или $\C$) и $V^\ast$ --- двойственное пространство. Элементы $v \in V$ --- это, как обычно, векторы, а элементы $\xi \in V^\ast$ здесь мы будем называть \textit{ковекторами}.

\begin{definition}
    \textit{Полилинейной функцией} типа $(p, q)$ (и \textit{валентности} $p + q$) называется функция
    \[
        \T: \underbrace{V \times \ldots \times V}_p \times \underbrace{V^\ast \times \ldots \times V^\ast}_q \to \F
    \]
    от $p$ векторных и $q$ ковекторных аргументов, которая линейна по каждому аргументу.
\end{definition}

Тензоры типа $(p, q)$ образуют линейное пространство над полем $\F$, в котором сумма и умножение на скаляры определены по формуле
\[
    (\lambda\T + \mu\S)(v_1, \ldots, v_p, \xi^1, \ldots, \xi^q) \vcentcolon = \lambda\T(v_1, \ldots, v_p, \xi^1, \ldots, \xi^q) + \mu\S(v_1, \ldots, v_p, \xi^1, \ldots, \xi^q).
\]

Мы будем обозначать это пространство через $\mathbb{P}_p^q(V)$.

\begin{example}
    Геометрическая интерпретация полилинейных функций малых валентностей:
    \begin{enumerate}
        \item Полилинейная функция типа $(0, 0)$ --- это скаляр из поля $\F$.
        \item Полилинейная функция типа $(1, 0)$ --- линейная функция на $V$, т.\,е. элемент $V^\ast$.
        \item Полилинейная функция типа $(0, 1)$ --- линейная функция на $V^\ast$, т.\,е. элемент из $V^{\ast\ast}$. Однако пространства $V$ и $V^{\ast\ast}$ канонически изоморфны. Имея в виду этот изоморфизм, полилинейную функцию типа $(0, 1)$ можно считать вектором, т.\,е. элементом из $V$.
        \item Полилинейная функция типа $(2, 0)$ --- это билинейная функция на $V$.
        \item Полилинейная типа $(1, 1)$ --- это линейный оператор $\A: V \to V$. Об этом говорит следующее предложение.
    \end{enumerate}
\end{example}

\begin{proposal}
    Сопоставление линейному оператору $\A$ полилинейной функции $\T_\A(v, \xi) \vcentcolon = \xi(\A v)$ типа $(1, 1)$ устанавливает канонический изоморфизм $\End(V) \simeq \mathbb{P}_1^1(V)$.
\end{proposal}

\begin{proof}
    Прежде всего заметим, что данное отображение $\End(V) \to \mathbb{P}_1^1(V)$ линейно. Пусть $\dim V = \vcentcolon n$. Тогда размерность пространства $\mathbb{P}_1^1(V)$ равна $n^2$ --- это доказывается при помощи выбора базиса так же, как и для билинейных функций. Поэтому размерности пространств $\End(V)$ и $\mathbb{P}_1^1(V)$ равны, а значит, достаточно доказать, что $\End(V) \to \mathbb{P}_1^1(V)$ --- инъекция. Будем делать это по критерию инъективности. Пусть $\T_\A$ --- тождественно нулевая полилинейная функция, т.\,е. $\xi(\A v) = 0$ $\forall v \in V, \xi \in V^\ast$. Получаем, что любая линейная функция обращается в нуль на векторе $\A v$, т.\,е. $\A v = \bs{0}$. Т.\,к. это верно $\forall v \in V$, получаем $\A = \O$. Итак, отображение $\A \mapsto \T_\A$ инъективно, а значит, задаёт изоморфизм.
\end{proof}

