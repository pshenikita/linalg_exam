\documentclass[a4paper, 11pt]{article}
\usepackage{preamble}
\usepackage{import}

\begin{document}
\let\hypercontentsline=\contentsline
\renewcommand{\contentsline}[4]{\hypertarget{toc.#4}{}\hypercontentsline{#1}{#2}{#3}{#4}}

\title{\bfseries\scshape Экзамен по линейной алгебре}
\date{1 курс$\quad\bullet\quad$Весенний семестр 2024\,г.}
\author{Лектор: Чубаров И.\,А.$\quad\bullet\quad$ Автор: Пшеничный Никита\thanks{\texttt{Telegram: \href{https://t.me/pshenikita}{@pshenikita}. Последняя компиляция: \DTMnow}}, группа 109}

\maketitle

\begin{abstract}
    При подготовке данного файла я использовал курсы лекций И.\,А. Чубарова, \href{http://higeom.math.msu.su/people/taras/#teaching}{Т.\,Е. Панова} и \href{http://halgebra.math.msu.su/wiki/lib/exe/fetch.php/28_lectures_02_06_20_.pdf}{О.\,В. Куликовой}, курс семинаров А.\,А. Клячко и книги <<Курс алгебры>> Э.\,Б. Винберга, <<Введение в алгебру. Часть \rnum{2}: Линейная алгебра>> А.\,И. Кострикина, <<Сборник задач по аналитической геометрии и линейной алгебре>> под редакцией Ю.\,М. Смирнова и <<Задачи по линейной алгебре>> А.\,А. Гайфуллина, А.\,В. Пенского и С.\,В. Смирнова.

    В конце есть раздел с условиями (всех) и решениями (некоторых) теоретических задач Т.\,Е. Панова, а по тексту и в приложениях разбросаны задачи из Винберга и семинаров А.\,А. Клячко. Так же разбросаны по тексту ссылки на файлы Антона Александровича с крутыми методами решения почти стандартных задач по линейной алгебре. Правда, там он пишет всё очень неподробно, думаю, здесь когда-нибудь появятся комментарии к этим файлам.

    Всё, что выделено {\scshape вот таким шрифтом} (кроме этой записи) является ссылкой (заголовки вопросов и содержание ссылаются друг на друга). Нумерация задач сквозная по всем вопросам и отдельная для приложений и теоретических задач.

    Обо всех ошибках и опечатках пишите мне, исправлю.
\end{abstract}

\tableofcontents

\newpage

\import{questions/}{question01.tex}
\import{questions/}{question02.tex}
\import{questions/}{question03.tex}
\import{questions/}{question04.tex}
\import{questions/}{question05.tex}
\import{questions/}{question06.tex}
\import{questions/}{question07.tex}
\import{questions/}{question08.tex}
\import{questions/}{question09.tex}
\import{questions/}{question10.tex}
\import{questions/}{question11.tex}
\import{questions/}{question12.tex}
\import{questions/}{question13.tex}
\import{questions/}{question14.tex}
\import{questions/}{question15.tex}
\import{questions/}{question16.tex}
\import{questions/}{question17.tex}
\import{questions/}{question18.tex}
\import{questions/}{question19.tex}
\import{questions/}{question20.tex}
\import{questions/}{question21.tex}
\import{questions/}{question22.tex}
\import{questions/}{question23.tex}
\import{questions/}{question24.tex}
\import{questions/}{question25.tex}
\import{questions/}{question26.tex}
\import{questions/}{question27.tex}
\import{questions/}{question28.tex}
\import{questions/}{question29.tex}
\import{questions/}{question30.tex}
\import{questions/}{question31.tex}
\import{questions/}{question32.tex}
\import{questions/}{question33.tex}
\import{questions/}{question34.tex}
\import{questions/}{question35.tex}
\import{questions/}{question36.tex}
\import{questions/}{question37.tex}
\import{questions/}{question38.tex}
\import{questions/}{question39.tex}
\import{questions/}{question40.tex}
\import{questions/}{question41.tex}
\import{questions/}{question42.tex}
\import{questions/}{question43.tex}
\import{questions/}{question44.tex}
\import{questions/}{question45.tex}

\renewcommand{\A}{\mathfrak{A}}
\renewcommand{\B}{\mathfrak{B}}

\import{questions/}{question46.tex}
\import{questions/}{question47.tex}
\import{questions/}{question48.tex}
\import{questions/}{question49.tex}
\import{questions/}{question50.tex}
\import{questions/}{question51.tex}
\import{questions/}{question52.tex}
\import{questions/}{question53.tex}
\import{questions/}{question54.tex}
\import{questions/}{question55.tex}
\import{questions/}{question56.tex}
\import{questions/}{question57.tex}

\renewcommand{\A}{\mathcal{A}}
\renewcommand{\B}{\mathcal{B}}

\import{questions/}{question58.tex}
\import{questions/}{question59.tex}
\import{questions/}{question60.tex}
\import{questions/}{question61.tex}
\import{questions/}{question62.tex}
\import{questions/}{question63.tex}
\import{questions/}{question64.tex}
% \import{questions/}{question65.tex}

\newpage

\setcounter{problem}{0}

\section{Приложения}

Здесь описаны некоторые сюжеты, которые я у кого-то видел, но которые, к сожалению, не вошли в программу экзамена.

\import{appendix/}{projectors.tex}
\import{appendix/}{mnk.tex}
\import{appendix/}{spector_decomposition.tex}
\import{appendix/}{poly_matrix.tex}

\renewcommand{\A}{\mathfrak{A}}
\renewcommand{\B}{\mathfrak{B}}

\import{appendix/}{barycentric.tex}

\renewcommand{\A}{\mathcal{A}}
\renewcommand{\B}{\mathcal{B}}

\newpage

\setcounter{problem}{0}

\import{}{problems.tex}

\end{document}

