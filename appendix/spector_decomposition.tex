\subsection{Спектральное разложение самосопряжённого оператора}

\setcounter{definition}{0}
\setcounter{proposal}{0}
\setcounter{lemma}{0}
\setcounter{theorem}{0}

\begin{definition}
    Проектор $\P: V \to V$ на $U$ вдоль $W$ называется \textit{ортогональным}, если $W = U^\perp$. Такой проектор будем обозначать через $\pr_U$.

    Это обозначение вполне согласуется с предыдущими: если $V = U \oplus U^\perp$, то $\forall v \in V$ мы имеем $v = \pr_Uv + \ort_Uv$.
\end{definition}

\begin{proposal}
    Проектор $\P: V \to V$ является самосопряжённым оператором тогда и тлько тогда, когда он ортогонален.
\end{proposal}

\begin{proof}
    Пусть $\P = \pr_U$ --- ортогональный проектор. Выберем ортонормированный базис в $U$ и дополним его до ортонормированного базиса в $V$. Тогда в этом ортонормированном базисе матрица оператора $\pr_U$ диагональна (с единицами и нулями на диагонали), а значит, оператор $\pr_U$ самосопряжён.

    Обратно, пусть $\P$ --- самосопряжённый проектор на $U$ вдоль $W$. Возьмём произвольные векторы $u \in U = \Im\P$ и $w \in W = \Ker\P$. Тогда $u = \P v$ для некоторого $v \in V$ и $\P w = \bs{0}$. Мы имеем
    \[
        (u, w) = (\P v, w) = (v, \P w) = 0,
    \]
    откуда получаем $W = U^\perp$ и $\P = \pr_U$.
\end{proof}

\begin{example}
    Пусть $u = (u^1, \ldots, u^n)^t \in \R^n$ --- ненулевой вектор-столбец и $\langle u\rangle$ --- одномерное подпространство. Тогда матрица оператора $\pr_{\langle u\rangle}$ в стандартном базисе $\R^n$ есть
    \[
        \frac{1}{\abs{u}^2}uu^t = \frac{1}{\abs{u}^2}
        \begin{pmatrix}
            u^1\\
            u^2\\
            \vdots\\
            u^n
        \end{pmatrix} \cdot 
        \begin{pmatrix}
            u^1 & u^2 & \ldots & u^n
        \end{pmatrix} = \frac{1}{\abs{u}^2}
        \begin{pmatrix}
            u^1u^1 & u^1u^2 & \ldots & u^1u^n\\
            u^2u^1 & u^2u^2 & \ldots & u^2u^n\\
            \vdots & \vdots & \cdots & \vdots\\
            u^nu^1 & u^nu^2 & \ldots & u^nu^n\\
        \end{pmatrix}.
    \]
\end{example}

Напомним, что спектром оператора $\A$ называется множество его собственный значений. Для каждого собственного значения $\lambda$ рассмотрим ортогональный проектор $\pr_{V_\lambda}$ на соответствующее собственное подпространство $V_\lambda$.

\begin{theorem}[Спектральное разложение]
    Пусть $\A$ --- самосопряжённый оператор. Тогда имеет место разложение
    \[
        \A = \sum_\lambda\lambda\pr_{V_\lambda},
    \]
    где сумма берётся по всем собственным значениям. При этом проекторы $\pr_{V_\lambda}$ удовлетворяют соотношениям $\pr_{V_\lambda}\A = \A\pr_{V_\lambda} = \lambda\pr_{V_\lambda}$ и $\pr_{V_\lambda}\pr_{V_\mu} = \O$ при $\lambda \ne \mu$.
\end{theorem}

\begin{proof}
    Оператор $\A$ коммутирует с проектором $\pr_{V_\lambda}$, потому что $\Im\pr_{V_\lambda} = V_\lambda$ и $\Ker\pr_{V_\lambda} = V_\lambda^\perp$ являются инвариантными подпространствами для $\A$ (задача 1 в приложении про проекторы). Действительно, первое очевидно, а второе следует из самосопряжённости оператора: если $v \in V_\lambda^\perp$, а $u \in V_\lambda$, то
    \[
        (\A v, u) = (v, \A u) = (v, \lambda u) = \lambda (v, u) = 0.
    \]

    А композиция проекторов на разные собственные подпространства тождественно нулевая, потому что (из решения задачи 2 в приложении про проекторы) их действие на вектор $v \in V$ есть его компонента из подпространства $\Im\pr_{V_\lambda} \cap \Im\pr_{V_\mu} = \{\bs{0}\}$. Коммутируют они по той же задаче (там очень вырожденный случай получается).

    Разложению $V = \bigoplus\limits_\lambda V_\lambda$ в прямую сумму собственных подпространств соответствует разложение тождественного оператора в сумму ортогональных проекторов:
    \[
        \id = \sum_\lambda\pr_{V_\lambda}.
    \]

    Умножив это соотношение слева на $\A$ и использовав соотношение $\A\pr_{V_\lambda} = \lambda\pr_{V_{\lambda}}$, получим требуемое.
\end{proof}

