\section{Барицентрические координаты}

Пусть $p_0, p_1, \ldots, p_n$ --- аффинно независимые точки $n$-мерного аффинного пространства $(\A, V)$. Тогда каждая точка $p \in \A$ единственным образом представляется в виде
\[
p = \sum_{i = 0}^nx^ip_i,\quad\text{где $\ds\sum_{i = 0}^nx^i = 1$}.
\]

В самом деле, это равенство можно переписать в виде
\[
    \overline{p_0p} = \sum_{i = 1}^nx^i\overline{p_0p_i},
\]
откуда следует, что в качестве $x^1, \ldots, x^n$ можно (и должно) взять координаты вектора $\overline{p_0p}$ в базисе $\overline{p_0p_1}, \ldots, \overline{p_0p_n}$; после этого определить $x^0$ равенством $x^0 = 1 - \sum\limits_{i = 1}^nx^i$

\begin{definition}
    Числа $x_0, x_1, \ldots, x_n$ называются \textit{барицентрическими координатами} точки $p$ относительно $p_0, p_1, \ldots, p_n$.
\end{definition}

\begin{proposal}
    Пусть $f: \A \to \A^\prime$ --- аффинное отображение. Тогда
    \[
        f\br{\sum_i\lambda_ip_i} = \sum_i\lambda_if(p_i)
    \]
    для любой барицентрической комбинации $\sum\limits_i\lambda_ip_i$ точек $p_1, \ldots, p_k$.
\end{proposal}

\begin{proof}
    Векторизуем пространство $\A$. Тогда получим
    \[
        f\br{\sum_i\lambda_ip_i} = \varphi\br{\sum_i\lambda_ip_i} + b = \sum_i\lambda_i(\varphi(p_i) + b) = \sum_i\lambda_if(p_i).
    \]
\end{proof}

В частности, центр тяжести системы точек при аффинном отображении переходит в центр тяжести их образов.

\begin{proposal}
    Барицентрические координаты суть аффинно-линейные функции.
\end{proposal}

\begin{proof}
    Пусть $x^0, x^1, \ldots, x^n$ --- барицентрические координаты относительно аффинно-независимых точек $p_0, p_1, \ldots, p_n$. Если векторизовать пространство $\A$, приняв точку $p_0$ за начало отсчёта, то $x^1, \ldots, x^n$ будут обычными координатами относительно базиса $\overline{p_0p_1}, \ldots, \overline{p_0p_n}$. Следовательно, $x^1, x^2, \ldots, x^n$ --- аффинно-линейные функции. А т.\,к. $x^0 = 1 - \sum\limits_{i = 1}^nx^i$, то $x^0$ --- также аффинно-линейная функция.
\end{proof}

