\subsection{Нормальные операторы}

Игорь Андреевич на консультации сказал, что он не успел рассказать эту тему. Не знаю, планировал ли он рассказать больше, чем здесь, но что-то здесь есть.

\begin{definition}
    Оператор $\A$ в евклидовом или эрмитовом пространстве называется \textit{нормальным}, если он коммутирует с сопряжённым, т.\,е. $\A^\ast\A = \A\A^\ast$.
\end{definition}

Все специальные классы операторов, рассмотренные выше (от самосопряжённых до унитарных), являются нормальными.

\begin{lemma}
    Пусть $v$ --- собственный вектор нормального оператора $\A$ с собственным значением $\lambda$. Тогда $v$ также является собственным вектором для сопряжённого оператора $\A^\ast$, с собственным значением $\overline{\lambda}$.
\end{lemma}

\begin{proof}
    Если оператор $\A$ нормален, то
    \[
        (\A v, \A v) = (\A^\ast\A v, v) = (\A\A^\ast v, v) = (\A^\ast v, \A^\ast v),
    \]
    т.\,е. $\abs{\A v} = \abs{\A^\ast v}$ для любого вектора $v$. Поскольку вместе с оператором $\A$ нормален и каждый оператор вида $\A - \lambda\id$, отсюда следует, что
    \[
        \abs{(\A - \lambda\id)v} = \abs{(\A^\ast - \overline{\lambda}\id)v}
    \]
    для любого $\lambda$. Поэтому, если $(\A - \lambda\id) = \bs{0}$, то $(\A^\ast - \overline{\lambda}\id) = \bs{0}$.
\end{proof}

В эрмитовом пространстве класс нормальных операторов --- это в точности операторы, диагонализируемые в ортонормированном базисе.

\begin{theorem}
    Оператор в эрмитовом пространстве диаонализируем в ортонормированном базисе тогда и только тогда, когда он нормален.
\end{theorem}

\begin{proof}
    Пусть матрица оператора $\A$ в некотором ортонормированном базисе диагональна с числами $\lambda_1, \ldots, \lambda_n$ на диагонали. Тогда сопряжённый оператор $\A^\ast$ в том же базисе имеет диагональную матрицу с числами $\overline{\lambda_1}, \ldots, \overline{\lambda_n}$. Т.\,к. диагональные матрицы коммутируют, мы получаем $\A^\ast\A = \A\A^\ast$, т.\,е. $\A$ нормален.

    Обратно, пусть $\A^\ast\A = \A\A^\ast$. Доказательство диагонализируемости в ортонормированном базисе будем вести по индукции по размерности пространства $V$. При $\dim V = 1$ доказывать нечего. Предположим, что утверждение доказано для пространств размерности не больше $n - 1$, и докажем его для размерности $n$.

    Выберем собственный вектор $v$ с собственным значением $\lambda$ для $\A$, т.\,е. одномерное инвариантное подпространство $W = \langle v\rangle$. Докажем, что ортогональное дополнение $W^\perp$ также инвариантно относительно $\A$. Пусть $u \in W^\perp$, т.\,е. $(u, v) = 0$. Тогда
    \[
        (\A u, v) = (u, \A^\ast v) = (u, \overline{\lambda}v) = \overline{\lambda}(u, v) = 0
    \]
    (где мы воспользовались предыдущей леммой леммой). Следовательно, $\A u \in W^\perp$ и пространство $W^\perp$ инвариантно относительно оператора $\A$.

    С другой стороны, пространство $W^\perp$ инвариантно относительно оператора $\A^\ast$ в силу леммы 1 из вопроса 37. Т.\,к. $W^\perp$ инвариантно и относительно $\A$, и относительно $\A^\ast$, ограничение $\A\big|_{W^\perp}$ является нормальным оператором. Т.\,к. $\dim W^\perp = n - 1$, по предположению индукции в пространстве $W^\perp$ имеется ортонормированный базис $e_1, \ldots, e_{n - 1}$ из собственных векторов оператора $\A\big|_{W^\perp}$. Тогда $e_1, \ldots, e_{n - 1}, \frac{v}{\abs{v}}$ --- ортонормированный базис из собственных векторов оператора $\A$.
\end{proof}

В евклидовом пространстве аналог этой теоремы не имеет места: оператор
$
\begin{pmatrix}
    a & -b\\
    b & a
\end{pmatrix}
$ нормален, но не диагонализируем при $b \ne 0$. На самом деле, мы знаем из теоремы 1 и предложения 2 в вопросе 38, что в евклидовом пространстве класс операторов, диагонализируемых в ортонормированном базисе, --- это в точности самосопряжённые операторы. Здесь можно посмотреть теоретическую задачу 24.

