\subsection{Проекторы}

\setcounter{definition}{0}
\setcounter{proposal}{0}
\setcounter{lemma}{0}
\setcounter{theorem}{0}

\begin{definition}
    Пусть пространство $V$ представимо в виде прямой суммы двух подпространств: $V = V_1 \oplus V_2$. Тогда $\forall v \in V$ имеется единственное разложение $v = v_1 + v_2$, $v_1 \in V_1$, $v_2 \in V_2$. Оператор $\P: V \to V$, переводящий вектор $v = v_1 + v_2$ в вектор $v_1$, называется \textit{проектором} на $V_1$ вдоль $V_2$.
\end{definition}

Для такого проектора $\P$ мы очевидно имеем $\Im\P = V_1$ и $\Ker\P = V_2$.

\begin{theorem}
    Оператор $\A: V \to V$ является проектором тогда и только тогда, когда $\A^2 = \A$.
\end{theorem}

\begin{proof}
    $\Rightarrow$. Если $\A$ --- проектор на $V_1$ вдоль $V_2$, то для $v = v_1 + v_2$ имеем
    \[
        \A^2v = \A(\A(v_1 + v_2)) = \A v_1 = v_1 = \A v,
    \]
    т.\,е. $\A^2 = \A$.

    $\Leftarrow$. Положим $V_1 \vcentcolon = \Im\A$, $V_2 \vcentcolon = \Ker\A$. Мы покажем, что $\A$ --- проектор на $V_1$ вдоль $V_2$.

    Сначала докажем, что $V = V_1 \oplus V_2$, т.\,е. что $V = V_1 + V_2$ и $V_1 \cap V_2 = \{\bs{0}\}$. Пусть $v \in V_1 \cap V_2$. Тогда $v \in V_1 = \Im\A$, т.\,е. существует $u \in V$ такой, что $\A u = v$, и $v \in V_2 = \Ker\A$, т.\,е. $\A v = \bs{0}$. Тогда
    \[
        v = \A u = \A^2 u = \A(\A u) = \A v = \bs{0},
    \]
    т.\,е. $v = \bs{0}$. Итак, $V_1$ и $V_2$ действительно образуют прямую сумму. Кроме того,
    \[
        \dim(V_1 \oplus V_2) = \dim V_1 + \dim V_2 = \dim\Im\A + \dim\Ker\A = \dim V,
    \]
    т.\,е. $V = V_1 \oplus V_2$.

    Рассмотрим теперь произвольный вектор $v \in V$ и представим его в виде $v = v_1 + v_2$, $v_1 \in V_1 = \Im\A$, $v_2 \in V_2 = \Ker\A$. Тогда существует $u \in V$ такой, что $v_1 = \A u$, а $\A v_2 = \bs{0}$. Мы имеем
    \[
        \A v = \A v_1 + \A v_2 = \A(\A u) = \A^2u = \A u = v_1.
    \]
    Итак, $\A$ --- действительно проектор на $V_1$ вдоль $V_2$.
\end{proof}

Приведём здесь один простой факт из теории множеств.

\begin{proposal}
    Отображение $f: X \to X$ тождественно на своём образе тогда и только тогда, когда $f^2 = f$.
\end{proposal}

\begin{proof}
    $\Rightarrow$. $\forall x \in X$ $f(\underbrace{f(x)}_{{} \in \Im f}) = f(x) \Rightarrow f^2 = f$.

    $\Leftarrow$. $\forall x \in X$ $f(f(x)) = f(x)$. Таким образом, $\forall (y \vcentcolon = f(x)) \in \Im f$ $f(y) = y \Rightarrow f\big|_{\Im f} = \id$.
\end{proof}

Таким образом, линейный оператор $\A: V \to V$ явлется проектором тогда и только тогда, когда он тождественен на своём образе.

Матрица проектора на $V_1$ вдоль $V_2$ в базисе, составленном из базисов пространств $V_1$ и $V_2$, имеет вид
$
\br{
    \begin{array}{c | c}
        E & 0\\
        \hline
        0 & 0
    \end{array}
}
$, где $E$ --- единичная матрица размера $\dim V_1$.

\begin{problem}[А.\,А. Клячко]
    Докажите, что оператор $\B$ коммутирует с проектором $\A$ тогда и только тогда, когда $\Im\A$ и $\Ker\A$ являются инвариантными подпространствами для $\B$.
\end{problem}

\begin{solution}
    $\Rightarrow$. Пусть $v \in \Im\A$. Тогда существует $u \in V$ такой, что $\A u = v$. При этом
    \[
        \B v = \B(\A u) = \A(\B u) \in \Im\A,
    \]
    т.\,е. подпространство $\Im\A$ инвариантно для $\B$. Теперь пусть $w \in \Ker\A$. Тогда
    \[
        \A(\B w) = \B(\A w) = \B\bs{0} = \bs{0} \in \Ker\A,
    \]
    т.\,е. подпространство $\Ker\A$ инвариантно для $\B$.

    $\Leftarrow$. Т.\,к. $\A$ --- проектор, то $V = \Im\A + \Ker\B$, т.\,е. $\forall v \in V$ можем записать $v = v_1 + v_2$, где $v_1 \in \Im\A$, $v_2 \in \Ker\A$. Так что силу линейности опрераторов нам достаточно проверить коммутирование отдельно на векторах из $\Im\A$ и из $\Ker\A$.

    Пусть $u \in \Im\A$, тогда $\B u \in \Im\A$. Как обсуждалось выше, проектор тождественен на своём образе, так что
    \[
        \A(\B u) = \B u,\quad \B(\A u) = \B u.
    \]

    Теперь пусть $w \in \Ker\A$, тогда $\B w \in \Ker\A$. Поэтому
    \[
        \A(\B w) = \bs{0},\quad \B(\A w) = \B\bs{0} = \bs{0}.
    \]
\end{solution}

\begin{problem}[А.\,А. Клячко]
    Докажите, что два проектора $\A$ и $\B$ коммутируют тогда и только тогда, когда пространство $V$ раскладывается в прямую сумму своих подпространств следующим образом:
    \[
        V = \Ker\A \cap \Ker\B \oplus \Im\A \cap \Ker\B \oplus \Ker\A \cap \Im\B \oplus \Im\A \cap \Im\B.
    \]
\end{problem}

\begin{solution}
    $\Rightarrow$. Проверим, что для каждого из подпространств в прямой сумме выполнено, что его пересечение с суммой остальных тривиально (см. теорему 2 в вопросе 5, пункт 2). Для простоты изложения введём обозначения для векторов соответствующих подпространств:
    \[
        v_1 \in \Ker\A \cap \Ker\B,\quad v_2 \in \Im\A \cap \Ker\B,\quad v_3 \in \Ker\A \cap \Im\B,\quad v_4 \in \Im\A \cap \Im\B.
    \]

    Предположим, что $v_1 = v_2 + v_3 + v_4$ (то есть, что $v_1$ лежит в первом подпространстве и в сумме трёх остальных). В первом равенстве подействуем оператором $\A\B$ на обе части. При этом обнулится всё, кроме $\A(\B v_4)$ (остальные векторы лежат в ядре одного из операторов $\A$ и $\B$). Поэтому равенство принимает вид $\A(\B v_4) = \bs{0}$, т.\,е. $v_4 \in \Ker\A$, но $v_4 \in \Im\A$, причём $\Ker\A \cap \Im\A = \{\bs{0}\}$. Значит, $v_4 = \bs{0}$. Теперь равенство принимает вид $v_1 = v_2 + v_3$. Подействуем оператором $\A$ на обе его части, получим $\A v_2 = \bs{0}$. Таким образом, $v_2 \in \Ker\A \cap \Im\A \Rightarrow v_2 = \bs{0}$. Итак, выражение приняло вид $v_1 = v_3$. Но тогда получаем, что вектор $v_1$ лежит одновременно и в $\Ker\B$, и в $\Im\B$, то есть, он тоже нулевой. И наконец, $v_1 = \bs{0}$, что и требовалось.

    Остальные случаи разбираются аналогично.

    $\Leftarrow$. Заметим, что $v_2 \in \Im\A$, а проектор $\A$ тождественен на своём образе, так что $\A v_2 = v_2$. Отсюда, $\B(\A v_2) = \B v_2 = \bs{0}$. Аналогично, $\A(\B v_3) = \bs{0}$. А вектор $v_4$ лежит в образах обоих проекторов, поэтому $\A(\B v_4) = \B(\A v_4) = v_4$.
    \begin{gather*}
        \A(\B(v_1 + v_2 + v_3 + v_4)) = \A(\B v_3 + \B v_4) = \overbrace{\A(\B v_3)}^{\bs{0}} + \A(\B v_4) = \A(\B v_4) = v_4,\\ \B(\A(v_1 + v_2 + v_3 + v_4)) = \B(\A v_2 + \A v_4) = \underbrace{\B(\A v_2)}_{\bs{0}} + \B(\A v_4) = \B(\A v_4) = v_4.
    \end{gather*}

    Таким образом, $\A$ и $\B$ коммутируют.
\end{solution}

