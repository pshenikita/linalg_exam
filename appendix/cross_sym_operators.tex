\subsection{Кососимметрические и косоэрмитовы операторы. Канонический вид. Эрмитово разложение}

\setcounter{definition}{0}
\setcounter{proposal}{0}
\setcounter{lemma}{0}
\setcounter{theorem}{0}

\begin{definition}
    Оператор $\A: V \to V$ в евклидовом (эрмитовом) пространстве называется \textit{кососимметрическим} (соответственно, \textit{косоэрмитовым}), если $\A^\ast = -\A$, т.\,е. $\forall u, v \in V$ выполнено соотношение
    \[
        (\A u, v) = -(u, \A v).
    \]
\end{definition}

\begin{proposal}
    Матрица $A$ кососимметрического (косоэрмитова) оператора $\A$ в ортонормированном базисе евклидова (эрмитова) пространства кососимметрична (косоэрмитова), т.\,е. $A^t = A$ (соответственно, $\overline{A}^t = A$).

    Если матрица оператора $\A$ в некотором ортонормированном базисе кососимметрична (косоэрмитова), то оператор $\A$ кососимметричен (косоэрмитов).
\end{proposal}

\begin{proof}
    То же, что и для самосопряжённых операторов.
\end{proof}

\begin{theorem}
    Для косоэрмитова оператора $\A$ существует ортонормированный базис, в котором его матрица диагональна с чисто мнимыми числами на диагонали. Другими словами, для косоэрмитова оператора существует ортонормированный базис из собственных векторов, а все собственные значения --- чисто мнимые.
\end{theorem}

\begin{proof}
    Доказательство, как и в случае самосопряжённого оператора, основано на важной лемме. Будем вести индукцию по размерности пространства $V$. При $\dim V = 1$ доказывать нечего. Предположим, что утверждение доказано для операторов в пространствах размерности $n - 1$, и докажем его для размерности $n$.

    Выберем собственный вектор $v$ для $\A$, т.\,е. одномерное инвариантное подпрсотранство $W = \langle v\rangle$. В силу выжной леммы ортогональное дополнение $W^\perp$ инвариантно относительно оператора $\A^\ast$, а значит, оно инвариантно и относительно $\A = -\A^\ast$. Т.\,к. $\dim W^\perp = n - 1$, в пространстве $W^\perp$ имеется ортонормированный базис $e_1, \ldots, e_{n - 1}$ из собственных векторов оператора $\A\big|_{W^\perp}$. Тогда $e_1, \ldots, e_{n - 1}, \frac{v}{\abs{v}}$ --- ортонормированный базис из собственных векторов оператора $\A$.

    Пусть $D$ --- диагональная матрица косоэрмитова оператора $\A$ в ортонормированном базисе из собственных векторов. Т.\,к. $\A^\ast = -\A$, получаем $\overline{D}^t = -D$. Следовательно, диагональные элементы матрицы $D$ (собственные числа) удовлетворяют соотношению $\overline{\lambda} = -\lambda$, т.\,е. являются чисто мнимыми.
\end{proof}

\begin{theorem}
    Для кососимметрического оператора $\A$ существует ортонормированный базис, в котором его матрица блочно-диагональная с блоками размера $1$ или $2$, причём блоки размера $1$ нулевые, а блоки размера $2$ имеют вид
    $
    \begin{pmatrix}
        0 & -a\\
        a & 0
    \end{pmatrix}
    $ с ненулевыми $a \in \R$.
\end{theorem}

\begin{proof}
    В пространстве размерности $1$ или $2$ доказывать нечего, т.\,к. кососимметрическая матрица и так имеет там требуемый вид. Предположим, что утверждение доказано для операторов в пространствах размерности не больше $n - 1$, и докажем его для пространства $V$ размерности $n$ (где $n \geqslant 3$).

    В силу теоремы теоремы 1 в вопросе 19 для оператора $\A$ существует одномерное или двумерное инвариантное подпространство $W \subset V$. Как и в случае косоэрмитовых операторов, из важной леммы следует, что ортогональное дополнение $W^\perp$ также инвариантно.

    По предположению индукции, в пространстве $W^\perp$ имеется терубемый базис для оператора $\A\big|_{W^\perp}$. Выбрав произвольный ортонормированный базис в $W$ и взяв объединение базисов $W^\perp$ и $W$, мы получим ортонормированный базис пространства $V$, в котором матрица оператора $\A$ состоит из блоков требуемого вида и ещё одного блока размера $1$ или $2$ --- матрицы оператора $\A\big|_W$. Этот последний блок --- кососимметрическая матрица размера $1$ или $2$, т.\,е. она тоже имеет требуемый вид.
\end{proof}

\begin{theorem}[Эрмитово разложение]
    Для любого оператора $\A$ в эрмитовом пространстве существует единственное представление в виде
    \[
        \A = \mathcal{R} + i\mathcal{I},
    \]
    где $\mathcal{R}$ и $\mathcal{I}$ --- эрмитовы операторы.
\end{theorem}

\begin{proof}
    Сначала докажем единственность. Если $\A = \mathcal{R} + i\mathcal{I}$ --- эрмитово разложение, то $\A^\ast = \mathcal{R}^\ast - i\mathcal{I}^\ast = \mathcal{R} - i\mathcal{I}$. Из этих двух соотношений получаем
    \[
        \mathcal{R} = \frac{1}{2}(\A^\ast + \A),\quad\mathcal{I} = \frac{i}{2}(\A^\ast - \A),
    \]
    т.\,е. операторы $\mathcal{R}$ и $\mathcal{I}$ определены однозначно и эрмитово разложение единственно.

    С другой стороны, операторы $\mathcal{R}$ и $\mathcal{I}$, задаваемыми предыдущими формулами, очевидно, эрмитовы (самосопряжены), так что эрмитово разложение существует.
\end{proof}

В одномерном эрмитовом пространстве $\C$ эрмитовы операторы --- это вещественные числа, а операторы $\mathcal{R}$ и $\mathcal{I}$ в эрмитовом разложении --- это вещественная и мнимая части комплексного числа.

