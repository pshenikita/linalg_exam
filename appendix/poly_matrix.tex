\subsection{Вычисление многочленов и функций от матриц}

Одним из важных применений жордановой формы является эффектвное вычисление многочленов и функций от операторов (матриц).

Прежде всего получим формулу для многочлена от жордановой клетки

\begin{proposal}
    Пусть $f \in \F[t]$. Тогда его значение на жордановой клетке
    \[
        J_\lambda =
        \begin{pmatrix}
            \lambda & 1 & & \\
            & \lambda & \ddots & \\
            & & \ddots & 1 \\
            & & & \lambda
        \end{pmatrix}
    \]
    размера $n$ вычисляется по формуле
    \[
        f(J_\lambda) =
        \begin{pmatrix}
            f(\lambda) & \frac{f^\prime(\lambda)}{1!} & \cdots & \frac{f^{(n - 1)}(\lambda)}{(n - 1)!}\\
            & \ddots & \ddots & \vdots\\
            \scalebox{2}{$0$} & & f(\lambda) & \frac{f^\prime(\lambda)}{1!}\\
            & & & f(\lambda)
        \end{pmatrix}.
    \]
\end{proposal}

\begin{proof}
    Вначале по индукции проверим эту формулу для многочлена $f(t) = t^m$, т.\,е. для $m$-ой степени жордановой клетки. Пусть $(J_\lambda)^m = (a_{ij}^m)$, т.\,е. $(ij)$-ый элемент матрицы $(J_\lambda)^m$ есть $a_{ij}^m$. По предположению индукции для $f(t) = t^{m - 1}$ имеем
    \[
        a_{ij}^{m - 1} = \frac{f^{(j - i)}(\lambda)}{(j - i)!} = C_{m - 1}^{j - i}\lambda^{m - 1 - j + i},
    \]
    где $C_k^i = \frac{k!}{i!(k - i)!}$ --- биномиальный коэффициент; мы считаем $C_k^i = 0$ при $i < 0$ или $i > k$. Тогда из соотношения $(J_\lambda)^m = (J_\lambda)^{m - 1}J_\lambda$ по правилу умножения матриц вычисляем
    \begin{multline*}
        a_{ik}^m = \sum_{j = 1}^na_{ij}^{m - 1}a_{jk}^1 = \sum_{j = 1}^nC_{m - 1}^{j - i}\lambda^{m - 1 - j + i}C_1^{k - j}\lambda^{1 - k + j} =\\ =C_{m - 1}^{k - 1 - i}\lambda^{m + i - k} + C_{m - 1}^{k - i}\lambda^{m + i - k} = C_m^{k - i}\lambda^{m - k + i},
    \end{multline*}
    что и требовалось. Осталось заметить, что эта формула линейна по $f$, а значит, она верна для любого многочлена $f(t)$.
\end{proof}

На основе этой формулы мы можем также вычислять многочлены от матриц в жордановой форме $J$, т.\,к. многочлен применяется к такой матрице поблочно.

Теперь если $A$ --- произвольная матрица, то мы можем привести её к жордановой форме, т.\,е. найти жорданову матрицу $J$, для которой $A = C^{-1}JC$. При возведении матрицы $A$ в степень мы получаем:
\[
    A^m = (C^{-1}JC)^m = C^{-1}JCC^{-1}JC\ldots C^{-1}JC = CJ^mC^{-1}.
\]
Тогда аналогичная формула верна для произвольного многочлена $f(t)$:
\[
    f(A) = C^{-1}f(J)C.
\]

При помощи этой формулы мы можем вычислять любой многочлен от матрицы $A$, зная её жорданову форму и жорданов базис.

\begin{definition}
    Пусть $\A$ --- оператор в комплексном пространстве с собственными значениями $\lambda_1, \ldots, \lambda_k$ кратностей $r_{\lambda_1}, \ldots, r_{\lambda_k}$ соответственно. Говорят, что два многочлена $f(t)$ и $g(t)$ \textit{совпадают на спектре оператора $\A$}, если
    \[
        f^{(j)}(\lambda_i) = g^{(j)}(\lambda_i)\quad\text{при $i = 1, \ldots, k$, $j = 0, \ldots, r_{\lambda_i} - 1$.}
    \]
\end{definition}

\begin{proposal}
    Если два многочлена $f(t)$ и $g(t)$ совпадают на спектре оператора $\A$, то $f(\A) = g(\A)$.
\end{proposal}

\begin{proof}
    При вычислении многочлена $f(A)$ по формулам выше используются производные многочлена $f(t)$ в точка $\lambda_i$ порядка не выше $r_{\lambda_i} - 1$.
\end{proof}

Это утверждение позволяет находить многочлен $f(t)$ большой степени от матрицы $A$ (например, возводить матрицу в большую степень), вовсе не вычисляя её жордановой формы, воспользовавшись следующим утверждением.

\begin{proposal}
    Пусть $\A$ --- оператор в $n$-мерном пространстве. Для любого многочлена $f(t)$ существует многочлен $g(t)$ степени меньше $n$, совпадающий с $f(t)$ на спектре оператора $\A$.
\end{proposal}

\begin{proof}
    Пусть $\chi_\A(t)$ --- характеристический многочлен для $\A$. Тогда $\chi_\A(t) = \prod\limits_{i = 1}^k(\lambda_i - t)^{r_{\lambda_i}}$ --- многочлен степени $n$, причём $\chi_\A(\A) = \O$. Разделим $f(t)$ на $\chi_\A(t)$ с остатком:
    \[
        f(t) = q(t)\chi_\A(t) = g(t),
    \]
    где $g(t)$ --- многочлен-остаток, имеющий степень $< n$. Подставив $\lambda_i$ в соотношение и дифференцируя требуемое число раз, получим требуемые соотношения. Кроме того, подставляя в соотношение выше оператор $\A$, получим $f(\A) = g(\A)$. Это показывает, что $g(t)$ --- требуемый многочлен.
\end{proof}

На практике указанный многочлен находится с помощью интерполяции или методом неопределённых коэффициентов. Тогда мы имеем $f(A) = g(A)$. Если же нам известна жорданова форма матрицы $A$, то можно ещё снизить степень многочлена $g$, заменив $r_{\lambda_i}$ на числа $m_{\lambda_i}$ --- размеры максимальных жордановых клеток с $\lambda_i$ на диагонали (жорданов базис для этого знать не обязательно).



